\chapter{Endomorfier}
Vi vil i dette kapitel etablere nogle vigtige resultater vedrørende endomorfier på elliptiske kurver, som viser sig nødvendige i beviset for Hasses sætning.

\section{Endomorfier på elliptiske kurver}
Lad $K$ være et legeme og $\clk$ dens tilhørende algebraiske aflukning. I det følgende vil vi med en elliptisk kurve $E$ mene en kurve på formen $y^2 = x^3 + Ax + B$. 
Vi begynder da med følgende definition:

\begin{definition}
En endomorfi på $E$ er en homomorfi $\alpha : E(\clk) \to E(\clk)$ givet
ved rationale funktioner.
\end{definition}

Med en rational funktion forstår vi en kvotient af polynomier. Det vil altså sige, at 
en endomorfi $\alpha$ skal opfylde, at $\alpha(P_1 + P_2) = \alpha(P_1) + \alpha(P_2)$ 
og der skal findes rationale 
funktioner $R_1(x, y)$ og $R_2(x, y)$, begge med koefficienter i $\clk$, så
\begin{align*}
	\alpha(x, y) = (R_1(x, y), R_2(x, y)),
\end{align*}
for alle $(x, y) \in E(\clk)$. da $\alpha$ er en homomorfi gælder der, at $\alpha(\infty)=\infty$. Den trivielle endomorfi angives med $0$ og er den endomorfi, som sender ethvert punkt til $\infty$. Vi vil fremover antage, at $\alpha$ ikke er den trivielle endomorfi, hvilket betyder at der findes $(x, y)$ sådan at $\alpha(x, y) \neq \infty$.

\begin{example}
Lad $E$ være en elliptisk kurve og lad $\alpha$ være givet ved, at $\alpha(P)=2P$. Da er $\alpha$ en homomorfi og
$\alpha(x, y) = (R_1(x, y), R_2(x, y))$, hvor
\begin{align*}
	R_1(x, y) &= \left( \frac{3x^2 + A}{2y} \right)^2 - 2x, \\
	R_2(x, y) &= \left( \frac{3x^2 + A}{2y} \right) \left(x - \left( \left( \frac{3x^2 + A}{2y} \right)^2 - 
	2x \right) \right) - y \\
	&= \left( \frac{3x^2 + A}{2y} \right) \left(3x - \left( \frac{3x^2 + A}{2y} \right)^2 \right) - y.
\end{align*}
Da både $R_1$ og $R_2$ er rationale funktioner er $\alpha$ en endomorfi for $E$.
\end{example}


Vi ønsker nu, at finde en standard repræsentation for de rationale funktioner, som en endomorfi er givet ved. Følgende sætning gør dette muligt for os:

\begin{thm}
\label{end_rep_theorem}
Lad $E$ være en elliptisk kurve over et legeme $K$. En endomorfi $\alpha$ kan da skrives som
\begin{align*}
	\alpha(x, y) = (r_1(x), r_2(x)y) = \left( \frac{p(x)}{q(x)}, \frac{s(x)}{t(x)}y \right),
\end{align*}
hvor $p, q$ henholdsvis $s, t$ ikke har nogen fælles faktor.
\end{thm}
\begin{proof}
For et punkt $(x, y) \in E(\clk)$ gælder der, at $y^2 = x^3 + Ax + B$ så vi har også, at 
\begin{align*}
	y^{2k} = (x^3 + Ax + B)^k \quad \text{og} \quad y^{2k+1}= y^{2k}y = (x^3 + Ax + B)^k y, \quad k \in \mathbb{N}.
\end{align*}
Vi kan altså erstatte en lige potens af $y$ med et polynomium der kun afhænger af $x$, og en ulige potens med $y$ ganget med et polynomium der kun afhænger af $x$. For en rational funktion $R(x, y)$ kan vi da beskrive en anden rational funktion, som stemmer overens med denne på punkter fra $E(\clk)$. Vi kan altså antage, at
\begin{align}
	\label{rational_1}
	R(x, y) = \frac{p_1(x) + p_2(x)y}{p_3(x)+p_4(x)y}.
\end{align}
Vi kan endda gøre det endnu simplere ved at gange udtrykket i \eqref{rational_1} med $p_3(x)-p_4(x)$, hvilket giver 
\begin{align*}
	(p_3(x) - p_4(x)y)(p_3(x)+p_4(x)y) = p_3(x)^2 - p_4(x)^2 y^2,
\end{align*}
hvorefter vi kan erstatte $y^2$ med $x^3 + Ax + B$. Dette giver os altså, at 
\begin{align}
	\label{rational_final}
	R(x, y) = \frac{q_1(x) + q_2(x)y}{q_3(x)}.
\end{align}
Da $\alpha$ er en endomorfi er den givet ved
\begin{align*}
	\alpha(x, y) = (R_1(x, y), R_2(x, y)),
\end{align*}
hvor $R_1$ og $R_2$ er rationale funktioner. Da $\alpha$ specielt er en homomorfi bevarer den strukturen for en elliptisk kurve så vi har, at
\begin{align*}
	\alpha(x, -y) = \alpha(-(x, y)) =  -\alpha(x, y).
\end{align*}
Dette medfører, at 
\begin{align*}
	R_1(x, -y) = R_1(x, y) \quad \text{og} \quad R_2(x, -y) = -R_2(x, y).
\end{align*}
Skriver vi $R_1$ på samme form som i \eqref{rational_final} må $q_2(x) = 0$, og ligeledes må vi for $R_2$ have at $q_1(x) = 0$. Vi kan altså antage, at
\begin{align*}
		\alpha(x, y) = (r_1(x), r_2(x)y),
\end{align*}
hvor $r_1(x)$ og $r_2(x)$ er rationale funktioner. Skriv da
\begin{align*}
	r_1(x) = \frac{p(x)}{q(x)} \quad \text{og} \quad r_2(x)=\frac{s(x)}{t(x)}y,
\end{align*}
hvor $p, q$ henholdsvis $s, t$ ikke har nogen fælles faktorer. Hvis $q(x)=0$ for et punkt $(x, y)$ lader vi 
$\alpha(x, y) = \infty$. Hvis $q(x) \neq 0$ giver (ii) i lemma \ref{rational_lemma}, at $r_2(x)$ da også vil være defineret.
\end{proof}

\begin{lemma}
\label{rational_lemma}
Lad $\alpha$ være en endomorfi givet ved
\begin{align*}
	\alpha(x, y) = \left( \frac{p(x)}{q(x)}, \frac{s(x)}{t(x)}y \right),
\end{align*}
for en elliptisk kurve $E$. Lad $p, q$ henholdsvis $s, t$ være sådan, at de ikke har nogen fælles rødder. Da har vi, at
\begin{enumerate}[(i)]
	\item For et polynomium $u(x)$, som ikke har en fælles rod med $q(x)$ har vi, at
	\begin{align*}
		\frac{(x^3+Ax+B)s(x)^2}{t(x)^2} = \frac{u(x)}{q(x)^3}.
	\end{align*}
	\item $t(x_0)=0$ hvis og kun hvis $q(x_0)=0$.
\end{enumerate}
\end{lemma}
\begin{proof}
(i) For et punkt $(x, y) \in E(K)$ har vi også, at $\alpha(x, y) \in E(K)$, da $\alpha$ er en endomorfi. Derfor har vi, at
\begin{align*}
	\frac{(x^3 + Ax + B)s(x)^2}{t(x)^2} &= \frac{y^2 s(x)^2}{t(x)^2} = \left( \frac{s(x)}{t(x)}y \right)^2 \\
	&= \left( \frac{p(x)}{q(x)} \right)^3 + A \left( \frac{p(x)}{q(x)} \right) + B \\
	&= \frac{p(x)^3 + A p(x) q(x)^2 + B q(x)^3}{q(x)^3} = \frac{u(x)}{q(x)^3},
\end{align*}
hvor $u(x) = p(x)^3 + A p(x) q(x)^2 + B q(x)^3$. Antag nu, at $q(x_0)=0$. Hvis nu også $u(x_0)=0$ følger det, at
\begin{align*}
	u(x_0)=p(x_0)^3 + A p(x_0) q(x_0)^2 + B q(x_0)^3 = 0 &\Rightarrow p(x_0)^3 = 0 \\ &\Rightarrow p(x_0)=0,
\end{align*}
men $p$ og $q$ havde pr. antagelse ingen fælles rødder. Så hvis $q(x_0) = 0$ må $u(x_0) \neq 0$ og de har dermed ingen fælles rødder.

(ii) Vi ved fra (i), at 
\begin{align}
	\label{lemma_rational_sup}
	(x^3 + Ax + B)s(x)^2 q(x)^3 = u(x) t(x)^2.
\end{align}
Hvis $q(x_0) = 0$ følger det direkte fra \eqref{lemma_rational_sup}, at 
\begin{align*}
	u(x_0)t(x_0)^2 = 0.
\end{align*}
Da $q$ og $u$ ikke har nogen fælles rødder følger det, at $t(x_0) = 0$. Antag nu, at $t(x_0)=0$, da har vi fra
\eqref{lemma_rational_sup}, at
\begin{align*}
	({x_0}^3 + Ax_0 + B)s(x_0)^2 q(x_0)^3 = 0.
\end{align*}
Da $s$ og $t$ pr. antagelse ikke har nogen fælles rødder giver det yderligere, at 
\begin{align*}
	({x_0}^3 + Ax_0 + B)q(x_0)^3 = 0.
\end{align*}
Hvis ${x_0}^3 + Ax_0 + B \neq 0$ er $q(x_0)^3 = 0$ og dermed må $q(x_0) = 0$. Hvis vi derimod har, at 
${x_0}^3 + Ax_0 + B = 0$ er det klart, at $(x-x_0) \mid (x^3 + Ax + B)$. Med andre ord findes et polynomium $Q(x)$ sådan, at
\begin{align*}
	(x^3 + Ax + B) = (x-x_0)Q(x),
\end{align*}
hvor $Q(x_0) \neq 0$, da $x^3 + Ax + B$ ikke har nogen dobbeltrødder. Da $t(x_0)=0$ findes der også et polynomium $T(x)$ sådan, at
\begin{align*}
	t(x) = (x-x_0)T(x).
\end{align*}
Udtrykket fra \eqref{lemma_rational_sup} kan da skrives, som
\begin{align*}
	(x-x_0)Q(x)s(x)^2q(x)^3 = u(x) ((x-x_0)T(x))^2,
\end{align*}
hvilket med division med $(x-x_0)$ giver os, at
\begin{align*}
	Q(x)s(x)^2 q(x)^3 =  u(x) (x-x_0) T(x)^2.
\end{align*}
I tilfældet, hvor $x=x_0$ har vi så, at
\begin{align*}
	Q(x_0)s(x_0)^2q(x_0)^3 = 0,
\end{align*}
men da $Q(x_0) \neq 0$ og $s(x_0) \neq 0$ må $q(x_0)^3 = 0$ så $q(x_0)=0$.
\end{proof}


Med den nu etablerede standard repræsentation for endomorfier, er vi i stand til at give en definition for graden af en endomorfi:

\begin{definition}
Graden af en endomorfi $\alpha$ er givet ved
\begin{align*}
	\deg(\alpha) = \max \{ \deg p(x), \deg q(x) \},
\end{align*}
når $\alpha$ ikke er den trivielle endomorfi, altså for $\alpha \neq 0$. For $\alpha = 0$ lader vi $\deg (\alpha) = 0$.
\end{definition}

En endomorfi siges at være \textit{separabel} hvis den afledede $r_1'(x) \neq 0$.

Den følgende proposition er essentiel idet, at det tilknytter graden af en endomorfi til antallet af elementer i kernen for selvsamme endomorfi. Dette faktum benyttes direkte i beviset for Hasses sætning.

\begin{proposition}
\label{deg_to_ker}
Lad $E$ være en elliptisk kurve. Lad $\alpha \neq 0$ være en separabel 
endomorfi for $E$. Da er 
\begin{align*}
	\deg \alpha = \# \ker (\alpha),
\end{align*}
hvor $\ker (\alpha) = $ angiver kernen for homomorfien 
$\alpha : E(\clk) \to E(\clk)$. I tilfældet hvor $\alpha \neq 0$ ikke
er separabel gælder der, at 
\begin{align*}
	\deg \alpha > \# \ker (\alpha).
\end{align*}
\end{proposition}
\begin{proof}
Vi skriver $\alpha$ på standardformen, som vi introducerede tidligere, altså
sættes
\begin{align*}
	\alpha(x, y) = (r_1(x), r_2(x) y)),
\end{align*}
hvor $r_1(x) = p(x)/q(x)$. Da $\alpha$ er antaget til at være separabel er 
$r_1' \neq 0$ og dermed er $pq'-p'q$ ikke nulpolynomiet. Lad nu
\begin{align*}
	S = \{ x \in \clk \mid (pq' - p'q)(x)q(x) = 0 \}.
\end{align*}
Lad da $(a, b) \in E(\clk)$ være valgt sådan, at følgende er opfyldt
\begin{enumerate}
	\item $a \neq 0$, $b \neq 0$ og $(a, b) \neq \infty$,
	\item $\deg (p(x) - aq(x)) = \max \{ \deg p(x), \deg q(x) \} = \deg \alpha$,
	\item $a \notin r_1(S)$,
	\item $(a, b) \in \alpha(E(\clk))$.
\end{enumerate}
Da $pq'-p'q$ ikke er 
nulpolynomiet er $S$ en endelig mængde, hvilket dermed også betyder, at 
$\alpha(S)$ er en endelig mængde. Funktionen $r_1(x)$ antager uendeligt 
mange forskellige værdier når $x$ gennemløber $\clk$, da en algebraisk aflukning indeholder uendeligt mange elementer.
%da en algebraisk aflukning indeholder uendeligt mange elementer.
Da der for hvert $x$ er et punkt $(x, y) \in E(\clk)$ følger det, at 
$\alpha(E(\clk))$ er en uendelig mængde. Det er altså muligt, at vælge et
punkt $(a, b) \in E(\clk)$ med egenskaberne ovenfor.

Vi ønsker at vise, at der netop er $\deg \alpha$ punkter $(x_1, y_1) \in E(\clk)$ sådan,
at $\alpha(x_1, y_1) = (a, b)$. For et sådan punkt gælder der, at 
\begin{align*}
	\frac{p(x_1)}{q(x_1)} = a, \quad r_2(x_1) y_1 = b.
\end{align*}
Da $(a, b) \neq \infty$ er $q(x_1) \neq 0$. % Øvelse giver, at r_2(x_1) er defineret
Da $b \neq 0$ har vi også, at $y_1 = b / r_2(x_1)$. Dette betyder, at $y_1$ er bestemt
ved $x_1$, så vi behøver kun at tælle værdier for $x_1$. Fra antagelse (2) har vi, at 
$p(x) - aq(x) = 0$ har $\deg \alpha$ rødder talt med multiplicitet. Vi skal altså vise,
at $p-aq$ ikke har nogen multiple rødder. Antag for modstrid, at $x_0$ er en multipel
rod. Da har vi, at
\begin{align*}
	p(x_0) - aq(x_0) = 0 \quad \text{og} \quad p'(x_0) - aq'(x_0) = 0.
\end{align*}
Dette kan omskrives til ligningerne $p(x_0)=aq(x_0)$ og $aq'(x_0)=p'(x_0)$, 
som vi ganger med hinanden og får, at 
\begin{align*}
	a p(x_0)q'(x_0) = ap'(x_0)q(x_0).
\end{align*} 
Da $a \neq 0$ pr. (1) giver det os, at $x_0$ er en rod i $pq'-p'q$ så $x_0 \in S$.
Altså er $a=r_1(x_0) \in r_1(S)$, hvilket er i modstrid med (3). Dermed har 
$p-aq$ netop $\deg \alpha$ forskellige rødder. Da der er præcist $\deg \alpha$ 
punkter $(x_1, y_1)$ så $\alpha(x_1, y_1) = (a, b)$ har kernen for $\alpha$ netop
$\deg \alpha$ elementer.
\end{proof}

Dette bevis konkluderer undersøgelse af endomorfier på generelle legemer $K$. I kapitel 4
vil vi fortsat behandle endomorfier, men der vil det være for endelige legemer.


















