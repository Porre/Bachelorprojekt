% Vi benytter os af MEMOIR (pr. daleif's anbefaling)
\documentclass[a4paper,11pt,oneside]{memoir}

% Dette er hovedfilen, hvor jeg definerer min pre-amble.
% Idéen er at sprede de enkelte kapitler ud i hvert deres
% dokument for at gøre det hele lidt mere overskueligt.
% Eventuelle makroer vil også fremgå heri.
% Inden 6 måneder skulle det gerne ligne et bachelorprojekt. 
% 24.01.2013

% Matematikpakker
\usepackage{amsmath}
\usepackage{amsfonts}
\usepackage{amssymb}
\usepackage{amsthm}

% Image stuff
\usepackage{graphicx}

% Dansk babel og tillad øæå osv.
\usepackage[danish]{babel}
\usepackage[utf8]{inputenc}

% Til referencer, Biber giver os UTF-8 understøttelse
\usepackage[backend=biber]{biblatex}
\usepackage{csquotes} 

% Filen med vores referencer
\addbibresource{referencer.bib}


% Styling for sætninger osv., er valgt efter normal anbefaling
\theoremstyle{plain}
\newtheorem{theorem}{Sætning}
\newtheorem{proposition}{Proposition}
\newtheorem{lemma}{Lemma}
\newtheorem{corollary}{Korollar}

\theoremstyle{definition}
\newtheorem{definition}{Definition}
\newtheorem{remark}{Bemærkning}
\newtheorem{example}{Eksempel}
\newtheorem{algorithm}{Algoritme}

% section som chapterstyle er mindre voldsomt til et "lille" projekt
% \chapterstyle{section}

% Sørg for at niveauet til og med subsection nummereres
\setcounter{secnumdepth}{2}
\setcounter{tocdepth}{1}

% TikZ bruges til tegninger
\usepackage{tikz}

% -- START PÅ MAKROER -- 
% Angiver punktet i uendelighed
\newcommand{\infp}{\mathcal{O}}

% Skriver LCM med tekst
\newcommand{\lcm}{\mathrm{LCM}}

% Hurtig måde at lave modulo
\newcommand{\modu}[1]{\ (\mathrm{mod} \ #1)}

% Lukninger af mængder
\newcommand{\clfq}{\overline{\mathbb{F}}_q}
\newcommand{\clk}{\overline{K}}

% Karakteristik
\newcommand{\cha}{\text{char}}

% -- SLUT PÅ MAKROER --

% Bruges kun under skrivningen af projektet
\includeonly{gruppeloven}

\begin{document}
%\tableofcontents
\chapter{Etablering af gruppestrukturen}
I dette kapitel vil vi introducere elliptiske kurver.
Det viser sig, at være muligt at påføre de elliptiske kurver en gruppestruktur ved en geometrisk addition af punkter fra en sådan kurve. Vi vil indføre denne additionslov og vise, at det resulterer i en abelsk gruppe. For at kunne gøre dette skal vi desuden anvende projektiv geometri, som også vil blive introduceret.

\section{Elliptiske kurver}
For at kunne snakke om elliptiske kurver skal vi først og fremmest have defineret, hvad en elliptisk kurve er. Vi vil i denne tekst benytte følgende definition:

\begin{definition}
En elliptisk kurve $E$ er grafen for en ligning
\begin{align}
	\label{elliptic_curve}
	y^2 = x^3 + Ax + B,
\end{align}
hvor $A, B \in K$ er konstanter og $4A^3 + 27B^2 \neq 0$. Denne type elliptisk kurve siges at være på Weierstrass normalform.
\end{definition}

Da $\Delta = -16(4A^3 + 27B^2)$ er diskriminanten for \eqref{elliptic_curve} betyder det, at vi ikke tillader multiple rødder for en elliptisk kurve. Altså har kurvens rødder alle multiplicitet $1$. Der findes mere generelle definitioner af elliptiske kurver, men når vi arbejder over legemer som ikke har karakteristik $2$ eller $3$, kan vi altid skrive en elliptisk kurve på Weierstrass normalform.






\section{Det projektive plan}
Som tidligere nævnt vil vi etablere en gruppestruktur på de elliptiske
kurver. For at kunne gøre dette får vi brug for det projektive plan 
$\mathbb{P}^2$. Rent intuitivt kan man se det projektive plan, som
værende den affine plan 
\begin{align*}
	\mathbb{A}^2(K) = \{ (x, y) \in K \times K \},
\end{align*}
hvor $K$ et et legeme, med en ekstra linje "i uendelig". 
Vi ønsker at formalisere dette begreb. 
For $x, y, z \in K$ ikke alle nul og $\lambda \in K$, $\lambda \neq 0$, 
definerer vi en ækvivalensrelation. To tripler $(x_1, y_1, z_1)$ og 
$(x_2, y_2, z_2)$ siges at være ækvivalente hvis 
\begin{align*}
	(x_1, y_1, z_1) = (\lambda x_2, \lambda y_2, \lambda z_2),
\end{align*}
og vi skriver $(x_1, y_1, z_1) \sim (x_2, y_2, z_2)$. Vi vil fremover skrive
$(x:y:z)$ for en sådan ækvivalensklasse. I de tilfælde hvor $z \neq 0$ har vi, at
\begin{align*}
	(x, y, z) = (x/z, y/z, 1),
\end{align*}
hvilket er de punkter vi kalder for de "endelige" punkter i $\mathbb{P}^2(K)$.
Vi er nemlig i stand til at associere et punkt fra $\mathbb{A}^2(K)$ med et sådan
punkt. Vi har en afbildning (en inklusion for at være mere præcis) 
$\mathbb{A}^2(K) \to \mathbb{P}^2(K)$ givet ved
\begin{align*}
	(x, y) \mapsto (x, y, 1).
\end{align*}
Dette kan vi selvfølgelig ikke gøre, når $z=0$ og vi ser det som at vi
har $\infty$ i enten $x$- eller $y$-koordinaten. Vi kalder dermed punkterne
$(x, y, 0)$ for punkterne i "uendelig".








\section{Gruppeloven}
Lad nu $E$ være en elliptisk kurve over $K$ som i \ref{elliptic_curve}. Mængden
af punkter på $E$ med koordinater i $K$ er givet ved
\begin{align*}
	E(K) = \{ \infp \} \cup \{ (x, y) \in K \times K \mid y^2 = x^3 + Ax + B \},
\end{align*}
hvor $\infp$ er punktet i uendelighed, som vi vil definere senere. Vi definerer
da en binær operator/funktion $+$ på $E(K)$ ved følgende algoritme:

\begin{definition}[Gruppeloven for elliptiske kurver]
Givet to punkter $P_1, P_2 \in E(K)$, $P_i = (x_i, y_i)$. Et tredje punkt
$R = P_1 + P_2 = (x_3, y_3)$ findes da som følger
\begin{enumerate}
	\item Hvis $x_1 \neq x_2$ da er 
	\begin{align*}
		x_3 = m^2 - x_1 - x_2, \quad y_3 = m(x_1 - x_3) - y_1,
	\end{align*}		
	hvor $m = (y_2 - y_1)/(x_2 - x_1)$.
	\item Hvis $x_1 = x_2$, men $y_1 \neq y_2$ da er $R = P_1 + P_2 = \infp$.
	\item Hvis $P_1 = P_2$ og $y_1 \neq 0$ da er 
	\begin{align*}
		x_3 = m^2 - 2x_1, \quad y_3=m(x_1 - x_3) - y_1,
	\end{align*}
	hvor $m=(3{x_1}^2 + A)/2y_1$.
	\item Hvis $P_1 = P_2$ og $y_1 = 0$ da er $R = P_1 + P_2 = \infp$.
\end{enumerate}
Vi definerer desuden, at 
\begin{align*}
	P + \infp = \infp,
\end{align*}
for alle $P \in E(K)$.
\end{definition}
Vælg to punkter 
\begin{align*}
	P = (x_p, y_p), \quad Q = (x_q, y_q)
\end{align*}
på en elliptisk kurve $E$. Vi kan da trække en ret linje, $L$, igennem
punkterne $P$ og $Q$, som vil skære kurven for $E$ i et tredje 
punkt $P*Q$. Reflektér dette punkt og vi definerer $P+Q$ til at være
dette punkt. Lad desuden $\infp$ betegne punktet i uendelighed. 

Vi skal nu udlede formlerne for additionen af disse punkter. Lad først
$P \neq Q$ og lad $P$ og $Q$ være forskellige fra $\infp$. Da har vi,
at hældningen for linjen igennem $P$ og $Q$ er 
\begin{align*}
	m = \frac{y_q - y_p}{x_q - x_p}.
\end{align*}
Hvis $x_p = x_q$ er linjen lodret, hvilket er et tilfælde vi behandler
senere. Så lad $x_p \neq x_q$, da får vi videre at 
\begin{align*}
	y_q = m(x_q - x_p) + y_p.
\end{align*}
Vi indsætter dette i ligningen for $E$ og får, at 
\begin{align*}
	(m(x - x_p) + y_p)^2 = x^3 + Ax + B.
\end{align*}
Skriver vi dette ud får vi, at
\begin{align*}
	0 &= x^3 + Ax + B - 2y_p m(x-x_p) - m^2 (x - x_p)^2 - y_{p}^{2} \\
	&= x^3 + Ax + B - 2y_p m x - 2y_ p m x_p - m^2 (x^2 -2xx_p + x_{p}^{2}) - y_{p}^{2} \\
	&= x^3 - m^2 x^2 + (A-2my_p +2m^2x_p)x -2m y_p x_p -m^2 x_{p}^{2} - y_{p}^{2} + B. 
\end{align*}
Denne har tre rødder, som netop er de tre punkter, hvor $L$ skærer $E$.
Pr. vores konstruktion kender vi allerede de to rødder $x_p$ og $x_q$,
og vi ønsker at finde den tredje. Generelt for et kubisk polynomium 
$x^3 + ax^2 + bx + c$, med rødder $r, s, t$, har vi at 
\begin{align*}
	x^3 + ax^2 + bx + c = (x - r)(x - s)(x - t) = x^3 - (r + s + t)x^2 + \ldots,
\end{align*}
hvilket giver os, at $-a = r + s + t$. Hvis de to rødder vi kender er $r$ og $s$
kan vi finde den sidste som
\begin{align*}
	t = -a - r - s.
\end{align*}
I vores tilfælde er $a=-m^2$ så vi har, at 
\begin{align*}
	x = m^2 - x_p - x_q.
\end{align*}
Vi mangler da blot at reflektere dette punkt for at have fundet punktet 
punktet $P+Q=(x, y)$. Vi reflekterer over $x$-aksen og finder, at 
\begin{align*}
	x = m^2 - x_p - x_q, \quad y = m(x_p - x) - y_p.
\end{align*}
Vi vender nu tilbage til tilfældet, hvor $x_p = x_q$. Da vil linjen igennem 
$P$ og $Q$ være lodret, så den skærer $E$ i $\infp$. Vi husker, at når $\infp$ 
reflekteres over $x$-aksen får vi igen $\infp$. Vi får altså, at $P+Q=\infp$.

Tilfældet hvor $P=Q=(x, y)$ kræver lidt flere overvejelser, da ikke ligeså
let kan udvælge en linje. For to punkter der ligger tæt på hinanden vil linjen
igennem punkterne nærme sig tangenten til et af punkterne. Derfor vælger vi i
dette tilfælde, at lade linjen der går igennem punkterne være deres tangentlinje.

Blah blah blah.

Hvis $P=\infp$ er linjen igennem $P$ og $Q$ en lodret linje der skærer $E$ i
refleksionen af $Q$. Derfor får vi, at 
\begin{align*}
	\infp + Q = Q.
\end{align*}
Der gælder derfor også, at $\infp + \infp = \infp$.

Vi har nu dækket de mulige tilfælde og kan opstille gruppeloven som følger.




\chapter{Endomorfier og torsionpunkter}

Ordenen af et element $P$ fra en gruppe over en elliptisk kurve er det mindste heltal $n$ sådan at $nP = \infty$. Hvis der ikke findes et sådan $n$ siges ordenen af $P$ at være uendelig. Torsionspunkterne er netop de punkter, som har endelig orden og de viser sig at have relevans for elliptiske kurver over endelig legemer, som vi vil undersøge i næste kapitel 3.

Vi skal desuden etablere nogle vigtige resultater vedrørende endomorfier på elliptiske kurver, som viser sig nødvendige i beviset for Hasses sætning.

\section{Endomorfier}
Lad $K$ være et legeme og $\clk$ dens tilhørende algebraiske aflukning. I det følgende vil vi med en elliptisk kurve $E$ mene en kurve på formen Når vi skriver om en elliptisk kurve $E$ menes den at være på Weierstrass normalform, altså $y^2 = x^3 + Ax + B$. 

Vi begynder da med følgende definition:

\begin{definition}
En endomorfi på $E$ er en homomorfi $\alpha : E(\clk) \to E(\clk)$ givet
ved rationale funktioner.
\end{definition}

Med en rational funktion forstår vi en kvotient af polynomier. Det vil altså sige, at 
en endomorfi $\alpha$ skal opfylde, at $\alpha(P_1 + P_2) = \alpha(P_1) + \alpha(P_2)$ 
og der skal findes rationale 
funktioner $R_1(x, y)$ og $R_2(x, y)$, begge med koefficienter i $\clk$, så
\begin{align*}
	\alpha(x, y) = (R_1(x, y), R_2(x, y)),
\end{align*}
for alle $(x, y) \in E(\clk)$. da $\alpha$ er en homomorfi gælder der, at $\alpha(\infty)=\infty$. Den trivielle endomorfi angives med $0$ og er den endomorfi, som sender ethvert punkt til $\infty$. Vi vil fremover antage, at $\alpha$ ikke er den trivielle endomorfi, hvilket betyder at der findes $(x, y)$ sådan at $\alpha(x, y) \neq \infty$.

\begin{example}
Lad $E$ være en elliptisk kurve og lad $\alpha$ være givet ved, at $\alpha(P)=2P$. Da er $\alpha$ en homomorfi og
$\alpha(x, y) = (R_1(x, y), R_2(x, y))$, hvor
\begin{align*}
	R_1(x, y) &= \left( \frac{3x^2 + A}{2y} \right)^2 - 2x, \\
	R_2(x, y) &= \left( \frac{3x^2 + A}{2y} \right) \left(x - \left( \left( \frac{3x^2 + A}{2y} \right)^2 - 
	2x \right) \right) - y \\
	&= \left( \frac{3x^2 + A}{2y} \right) \left(3x - \left( \frac{3x^2 + A}{2y} \right)^2 \right) - y.
\end{align*}
Da både $R_1$ og $R_2$ er rationale funktioner er $\alpha$ en endomorfi for $E$.
\end{example}


Vi ønsker nu, at finde en standard repræsentation for de rationale funktioner, som en endomorfi er givet ved. Følgende sætning gør dette muligt for os:

\begin{theorem}
\label{end_rep_theorem}
Lad $E$ være en elliptisk kurve over et legeme $K$. En endomorfi $\alpha$ kan da skrives som
\begin{align*}
	\alpha(x, y) = (r_1(x), r_2(x)y) = \left( \frac{p(x)}{q(x)}, \frac{s(x)}{t(x)}y \right),
\end{align*}
hvor $p, q$ henholdsvis $s, t$ ikke har nogen fælles faktor.
\end{theorem}
\begin{proof}
For et punkt $(x, y) \in E(\clk)$ gælder der, at $y^2 = x^3 + Ax + B$ så vi har også, at 
\begin{align*}
	y^{2k} = (x^3 + Ax + B)^k \quad \text{og} \quad y^{2k+1}= y^{2k}y = (x^3 + Ax + B)^k y, \quad k \in \mathbb{N}.
\end{align*}
Vi kan altså erstatte en lige potens af $y$ med et polynomium der kun afhænger af $x$, og en ulige potens med $y$ ganget med et polynomium der kun afhænger af $x$. For en rational funktion $R(x, y)$ kan vi da beskrive en anden rational funktion, som stemmer overens med denne på punkter fra $E(\clk)$. Vi kan altså antage, at
\begin{align}
	\label{rational_1}
	R(x, y) = \frac{p_1(x) + p_2(x)y}{p_3(x)+p_4(x)y}.
\end{align}
Vi kan endda gøre det endnu simplere ved at gange udtrykket i \eqref{rational_1} med $p_3(x)-p_4(x)$, hvilket giver 
\begin{align*}
	(p_3(x) - p_4(x)y)(p_3(x)+p_4(x)y) = p_3(x)^2 - p_4(x)^2 y^2,
\end{align*}
hvorefter vi kan erstatte $y^2$ med $x^3 + Ax + B$. Dette giver os altså, at 
\begin{align}
	\label{rational_final}
	R(x, y) = \frac{q_1(x) + q_2(x)y}{q_3(x)}.
\end{align}
Da $\alpha$ er en endomorfi er den givet ved
\begin{align*}
	\alpha(x, y) = (R_1(x, y), R_2(x, y)),
\end{align*}
hvor $R_1$ og $R_2$ er rationale funktioner. Da $\alpha$ specielt er en homomorfi bevarer den strukturen for en elliptisk kurve så vi har, at
\begin{align*}
	\alpha(x, -y) = \alpha(-(x, y)) =  -\alpha(x, y).
\end{align*}
Dette medfører, at 
\begin{align*}
	R_1(x, -y) = R_1(x, y) \quad \text{og} \quad R_2(x, -y) = -R_2(x, y).
\end{align*}
Skriver vi $R_1$ på samme form som i \eqref{rational_final} må $q_2(x) = 0$, og ligeledes må vi for $R_2$ have at $q_1(x) = 0$. Vi kan altså antage, at
\begin{align*}
		\alpha(x, y) = (r_1(x), r_2(x)y),
\end{align*}
hvor $r_1(x)$ og $r_2(x)$ er rationale funktioner. Skriv da
\begin{align*}
	r_1(x) = \frac{p(x)}{q(x)} \quad \text{og} \quad r_2(x)=\frac{s(x)}{t(x)}y,
\end{align*}
hvor $p, q$ henholdsvis $s, t$ ikke har nogen fælles faktorer. Hvis $q(x)=0$ for et punkt $(x, y)$ lader vi 
$\alpha(x, y) = \infty$. Hvis $q(x) \neq 0$ giver (ii) i lemma \ref{rational_lemma}, at $r_2(x)$ da også vil være defineret.
\end{proof}

\begin{lemma}
\label{rational_lemma}
Lad $\alpha$ være en endomorfi givet ved
\begin{align*}
	\alpha(x, y) = \left( \frac{p(x)}{q(x)}, \frac{s(x)}{t(x)}y \right),
\end{align*}
for en elliptisk kurve $E$. Lad $p, q$ henholdsvis $s, t$ være sådan, at de ikke har nogen fælles rødder. Da har vi, at
\begin{enumerate}[(i)]
	\item For et polynomium $u(x)$, som ikke har en fælles rod med $q(x)$ har vi, at
	\begin{align*}
		\frac{(x^3+Ax+B)s(x)^2}{t(x)^2} = \frac{u(x)}{q(x)^3}.
	\end{align*}
	\item $t(x_0)=0$ hvis og kun hvis $q(x_0)=0$.
\end{enumerate}
\end{lemma}
\begin{proof}
(i) For et punkt $(x, y) \in E(K)$ har vi også, at $\alpha(x, y) \in E(K)$, da $\alpha$ er en endomorfi. Derfor har vi, at
\begin{align*}
	\frac{(x^3 + Ax + B)s(x)^2}{t(x)^2} &= \frac{y^2 s(x)^2}{t(x)^2} = \left( \frac{s(x)}{t(x)}y \right)^2 \\
	&= \left( \frac{p(x)}{q(x)} \right)^3 + A \left( \frac{p(x)}{q(x)} \right) + B \\
	&= \frac{p(x)^3 + A p(x) q(x)^2 + B q(x)^3}{q(x)^3} = \frac{u(x)}{q(x)^3},
\end{align*}
hvor $u(x) = p(x)^3 + A p(x) q(x)^2 + B q(x)^3$. Antag nu, at $q(x_0)=0$. Hvis nu også $u(x_0)=0$ følger det, at
\begin{align*}
	u(x_0)=p(x_0)^3 + A p(x_0) q(x_0)^2 + B q(x_0)^3 = 0 &\Rightarrow p(x_0)^3 = 0 \\ &\Rightarrow p(x_0)=0,
\end{align*}
men $p$ og $q$ havde pr. antagelse ingen fælles rødder. Så hvis $q(x_0) = 0$ må $u(x_0) \neq 0$ og de har dermed ingen fælles rødder.

(ii) Vi ved fra (i), at 
\begin{align}
	\label{lemma_rational_sup}
	(x^3 + Ax + B)s(x)^2 q(x)^3 = u(x) t(x)^2.
\end{align}
Hvis $q(x_0) = 0$ følger det direkte fra \eqref{lemma_rational_sup}, at 
\begin{align*}
	u(x_0)t(x_0)^2 = 0.
\end{align*}
Da $q$ og $u$ ikke har nogen fælles rødder følger det, at $t(x_0) = 0$. Antag nu, at $t(x_0)=0$, da har vi fra
\eqref{lemma_rational_sup}, at
\begin{align*}
	({x_0}^3 + Ax_0 + B)s(x_0)^2 q(x_0)^3 = 0.
\end{align*}
Da $s$ og $t$ pr. antagelse ikke har nogen fælles rødder giver det yderligere, at 
\begin{align*}
	({x_0}^3 + Ax_0 + B)q(x_0)^3 = 0.
\end{align*}
Hvis ${x_0}^3 + Ax_0 + B \neq 0$ er $q(x_0)^3 = 0$ og dermed må $q(x_0) = 0$. Hvis vi derimod har, at 
${x_0}^3 + Ax_0 + B = 0$ er det klart, at $(x-x_0) \mid (x^3 + Ax + B)$. Med andre ord findes et polynomium $Q(x)$ sådan, at
\begin{align*}
	(x^3 + Ax + B) = (x-x_0)Q(x),
\end{align*}
hvor $Q(x_0) \neq 0$, da $x^3 + Ax + B$ ikke har nogen dobbeltrødder. Da $t(x_0)=0$ findes der også et polynomium $T(x)$ sådan, at
\begin{align*}
	t(x) = (x-x_0)T(x).
\end{align*}
Udtrykket fra \eqref{lemma_rational_sup} kan da skrives, som
\begin{align*}
	(x-x_0)Q(x)s(x)^2q(x)^3 = u(x) ((x-x_0)T(x))^2,
\end{align*}
hvilket med division med $(x-x_0)$ giver os, at
\begin{align*}
	Q(x)s(x)^2 q(x)^3 =  u(x) (x-x_0) T(x)^2.
\end{align*}
I tilfældet, hvor $x=x_0$ har vi så, at
\begin{align*}
	Q(x_0)s(x_0)^2q(x_0)^3 = 0,
\end{align*}
men da $Q(x_0) \neq 0$ og $s(x_0) \neq 0$ må $q(x_0)^3 = 0$ så $q(x_0)=0$.
\end{proof}


Med den nu etablerede standard repræsentation for endomorfier, er vi i stand til at give en definition for graden af en endomorfi:

\begin{definition}
Graden af en endomorfi $\alpha$ er givet ved
\begin{align*}
	\deg(\alpha) = \max \{ \deg p(x), \deg q(x) \},
\end{align*}
når $\alpha$ ikke er den trivielle endomorfi, altså for $\alpha \neq 0$. For $\alpha = 0$ lader vi $\deg (\alpha) = 0$.
\end{definition}

En endomorfi siges at være separabel hvis den afledede $r_1'(x) \neq 0$.

\begin{example}
Eksempel på en separabel endomorfi. Bogen ser på $2P$ som også er oplagt, men måske skulle man vælge en mere interessant.
\end{example}


Den følgende proposition er essentiel idet, at det tilknytter graden af en endomorfi til antallet af elementer i kernen for selvsamme endomorfi. Dette faktum benyttes direkte i beviset for Hasses sætning.

\begin{proposition}
\label{deg_to_ker}
Lad $E$ være en elliptisk kurve. Lad $\alpha \neq 0$ være en separabel 
endomorfi for $E$. Da er 
\begin{align*}
	\deg \alpha = \# \ker (\alpha),
\end{align*}
hvor $\ker (\alpha) = $ angiver kernen for homomorfien 
$\alpha : E(\clk) \to E(\clk)$. I tilfældet hvor $\alpha \neq 0$ ikke
er separabel gælder der, at 
\begin{align*}
	\deg \alpha > \# \ker (\alpha).
\end{align*}
\end{proposition}
\begin{proof}
Vi skriver $\alpha$ på standardformen, som vi introducerede tidligere, altså
sættes
\begin{align*}
	\alpha(x, y) = (r_1(x), r_2(x) y)),
\end{align*}
hvor $r_1(x) = p(x)/q(x)$. Da $\alpha$ er antaget til at være separabel er 
$r_1' \neq 0$ og dermed er $pq'-p'q$ ikke nulpolynomiet. Lad nu
\begin{align*}
	S = \{ x \in \clk \mid (pq' - p'q)(x)q(x) = 0 \}.
\end{align*}
Lad da $(a, b) \in E(\clk)$ være valgt sådan, at følgende er opfyldt
\begin{enumerate}
	\item $a \neq 0$, $b \neq 0$ og $(a, b) \neq \infty$,
	\item $\deg (p(x) - aq(x)) = \max \{ \deg p(x), \deg q(x) \} = \deg \alpha$,
	\item $a \notin r_1(S)$,
	\item $(a, b) \in \alpha(E(\clk))$.
\end{enumerate}
Da $pq'-p'q$ ikke er 
nulpolynomiet er $S$ en endelig mængde, hvilket dermed også betyder, at 
$\alpha(S)$ er en endelig mængde. Funktionen $r_1(x)$ antager uendeligt 
mange forskellige værdier når $x$ gennemløber $\clk$, da en algebraisk aflukning indeholder uendeligt mange elementer.
%da en algebraisk aflukning indeholder uendeligt mange elementer.
Da der for hvert $x$ er et punkt $(x, y) \in E(\clk)$ følger det, at 
$\alpha(E(\clk))$ er en uendelig mængde. Det er altså muligt, at vælge et
punkt $(a, b) \in E(\clk)$ med egenskaberne ovenfor.

Vi ønsker at vise, at der netop er $\deg \alpha$ punkter $(x_1, y_1) \in E(\clk)$ sådan,
at $\alpha(x_1, y_1) = (a, b)$. For et sådan punkt gælder der, at 
\begin{align*}
	\frac{p(x_1)}{q(x_1)} = a, \quad r_2(x_1) y_1 = b.
\end{align*}
Da $(a, b) \neq \infty$ er $q(x_1) \neq 0$. % Øvelse giver, at r_2(x_1) er defineret
Da $b \neq 0$ har vi også, at $y_1 = b / r_2(x_1)$. Dette betyder, at $y_1$ er bestemt
ved $x_1$, så vi behøver kun at tælle værdier for $x_1$. Fra antagelse (2) har vi, at 
$p(x) - aq(x) = 0$ har $\deg \alpha$ rødder talt med multiplicitet. Vi skal altså vise,
at $p-aq$ ikke har nogen multiple rødder. Antag for modstrid, at $x_0$ er en multipel
rod. Da har vi, at
\begin{align*}
	p(x_0) - aq(x_0) = 0 \quad \text{og} \quad p'(x_0) - aq'(x_0) = 0.
\end{align*}
Dette kan omskrives til ligningerne $p(x_0)=aq(x_0)$ og $aq'(x_0)=p'(x_0)$, 
som vi ganger med hinanden og får, at 
\begin{align*}
	a p(x_0)q'(x_0) = ap'(x_0)q(x_0).
\end{align*} 
Da $a \neq 0$ pr. (1) giver det os, at $x_0$ er en rod i $pq'-p'q$ så $x_0 \in S$.
Altså er $a=r_1(x_0) \in r_1(S)$, hvilket er i modstrid med (3). Dermed har 
$p-aq$ netop $\deg \alpha$ forskellige rødder. Da der er præcist $\deg \alpha$ 
punkter $(x_1, y_1)$ så $\alpha(x_1, y_1) = (a, b)$ har kernen for $\alpha$ netop
$\deg \alpha$ elementer.
\end{proof}



















\chapter{Elliptiske kurver over endelige legemer}

Vi skal i dette kapitel betragte elliptiske kurver over endelige legemer. Lad $\mathbb{F}$ være et endeligt legeme og lad $E$ være en elliptisk kurve over $\mathbb{F}$. Da er gruppen $E(\mathbb{F})$ endelig, da der kun findes endeligt mange talpar $(x, y)$ hvor $x, y \in \mathbb{F}$. Et endeligt legeme har $p^n$ elementer for et primtal $p$, hvor $n \geq 1$ (se bilag \ref{appendiks_legemer}). Derfor lader vi $\mathbb{F}_{q}$ være det endelige legeme med $q = p^n$ elementer. 

Vi vil vise Hasses sætning, som giver os en vurdering på antallet af punkter i gruppen $E(\mathbb{F}_q)$. Denne vurdering viser sig, at have en anvendelse indenfor heltalsfaktorisering, som vi ser på i kapitel 4. Vi ser også på en måde, hvorpå vi kan bestemme den eksakte orden af en gruppe $E(\mathbb{F}_{q^n})$, hvis vi kender ordenen af $E(\mathbb{F}_q)$, som er let at bestemme for legemer med få elementer.


\section{Eksempler}

Lad $E$ være en elliptisk kurve på formen
\begin{align*}
	E : y^2 = x^3 - x,
\end{align*}
defineret over $\mathbb{F}_5$. Da er gruppen $E(\mathbb{F}_5)$ endelig, som nævnt ovenfor. For at bestemme den eksakte orden af $E(\mathbb{F}_5)$ laver vi en tabel over alle mulige værdier for $x$, $x^3-x \modu{5}$ og for kvadratrødderne $y$ af $x^3 - x \modu{5}$. Dette giver os samtlige punkter på kurven:
\begin{center}
\begin{tabular}{c c c c }
$x$ & $x^3 - x$ & $y$ & Punkter \\ 
\hline
$0$ & $0$ & $0$ & $(0, 0)$ \\ 
$1$ & $0$ & $0$ & $(1, 0)$ \\ 
$2$ & $1$ & $\pm 1$ & $(2, 1), (2, 4)$ \\ 
$3$ & $4$ & $\pm 2$ & $(3, 2), (3, 3)$ \\ 
$4$ & $2$ & $-$ & $-$ \\ 
$\infty$ & & $\infty$ & $\infty$ \\
\end{tabular} 
\end{center}
Vi kan da tælle punkterne og vi ser, at $E(\mathbb{F}_5)$ har orden $7$, hvilket vi skriver som 
$\#E(\mathbb{F}_5) = 7$. Bemærk at $\sqrt{2} \notin \mathbb{Z}_5$, hvilket er hvorfor der ikke er en tilhørende værdi for $y$ til $x=4$.

Additionen af punkterne på en sådan kurve over et endeligt legeme foretages på samme måde, som i formlerne i gruppeloven, men de foretages modulo $p$. Eksempelvis hvis vi ville bestemme $(1,0) + (3,3)$ får vi, at 
\begin{align*} 
	m = \frac{y_2 - y_1}{x_2 - x_1} = \frac{3 - 0}{3 - 1} = \frac{3}{2} \equiv 4 \modu{5}.
\end{align*}
Dermed har vi, at 
\begin{align*}
	x_3 &= m^2 - x_1 - x_2 = 4^2 - 1 - 3 = 12 \equiv 2 \modu{5}, \\
	y_3 &= m(x_1 - x_3) - y_1 = 4(1-2)-0 = -4 \equiv 1 \modu{5}.
\end{align*}
Vi får altså punktet
\begin{align*}
	(1, 0) + (3, 3) = (2, 1),
\end{align*}
som netop er ét af punkterne vi har opgivet i tabellen.

\section{Frobenius endomorfien}
I det forrige kapitel så vi på endomorfier for generelle legemer. Nu vil vi se på en endomorfi som er defineret over endelige legemer, som viser sig at have en kritisk rolle i vores bevis for Hasses sætning. Denne endomorfi er Frobenius endomorfien $\phi_q$. For en elliptisk kurve $E$ over et endeligt legeme $\mathbb{F}_q$ er denne givet ved
\begin{align}
	\phi_q (x, y) = (x^q, y^q), \quad \phi_q(\infty) = \infty.
\end{align}
Vi skal nu vise nogle af de egenskaber, som denne endomorfi besidder:
\begin{lemma}
\label{lemma_end_degree_not_sep}
Lad $E$ være en elliptisk kurve over $\mathbb{F}_q$. Da er $\phi_q$ en 
endomorfi for $E$ af grad $q$, som ikke er separabel.
\end{lemma}
Vi skal bruge, at $(a + b)^q = a^q + b^q$ når $q=p^n$ hvor $p$ er et primtal (se appendiks \ref{appendiks_legemer}).
\begin{proof}
Vi vil først vise, at $\phi_q : E(\clfq) \to E(\clfq)$ er en homomorfi. Lad da $(x_1, y_1)$, $(x_2, y_2) \in E(\clfq)$ hvor $x_1 \neq x_2$. Det følger da fra gruppeloven, at summen af de to punkter $(x_3, y_3) = (x_1, y_1) + (x_2, y_2)$ er givet ved
\begin{align*}
	x_3 &= m^2 - x_1 - x_2, \\
	y_3 &= m(x_1 - x_3) - y_1, 
\end{align*}
hvor $m=(y_2-y_1)/(x_2-x_1)$.
Opløfter vi til $q$'ende potens får vi videre, at  
\begin{align*}
	x_{3}^{q} &= {m'}^2 - x_{1}^{q} - x_{2}^{q}, \\
	y_{3}^{q} &= m'(x_{1}^{q} - x_{3}^{q}) - y_{1}^{q},
\end{align*}
hvor $m' = (y_{2}^{q} - y_{1}^{q})/(x_{2}^{q} - x_{1}^{q})$. Sammensætter vi de resultater har vi netop, at $\phi_q(x_3, y_3) = \phi_q(x_1, y_1) + \phi_q(x_2, y_2)$, hvilket er hvad $\phi_q$ skal opfylde for at være en homomorfi (for alle punkter). 

I tilfældet hvor $x_1=x_2$ har vi fra gruppeloven, at 
\begin{align*}
	(x_3, y_3) =(x_1, y_1) + (x_2, y_2) = \infty.
\end{align*}
Men hvis $x_1 = x_2$ må $x_{1}^{q} = x_{2}^{q}$ hvilket betyder, at $\phi_q(x_1, y_1)+\phi_q(x_2, y_2) = \infty$. Så da $\infty^q = \infty$ får vi, at 
\begin{align*}
	\phi_q(x_3, y_3) = \phi_q(x_1, y_1) + \phi_q(x_2,  y_2).
\end{align*}
Hvis ét af punkterne er $\infty$, eksempelvis $(x_1, y_1)=\infty$, har vi fra gruppeloven, at 
$(x_3, y_3) = (x_1, y_1)+(x_2, y_2) = (x_2, y_2)$. Bruger vi igen, at $\infty^q = \infty$ følger det direkte, at 
$\phi_q(x_3, y_3) = \phi_q(x_1, y_1) + \phi_q(x_2, y_2)$.

Når $(x_1, y_1)=(x_2, y_2)$ hvor $y_1 = 0$ er $(x_3, y_3) = (x_1, y_1)+(x_2, y_2) = \infty$. Når $y_1=0$ er ${y_1}^q=0$ så $\phi_q(x_1, y_1) + \phi_q(x_2, y_2) = \infty$ og vi har endnu engang, at $\phi_q(x_3, y_3) = \phi_q(x_1, y_1) + \phi_q(x_2, y_2)$. 

Det resterende tilfælde er når $(x_1, y_1)=(x_2, y_2)$ og $y_1 \neq 0$. Fra gruppeloven har vi, at 
$(x_3, y_3)= 2(x_1, y_1)$, hvor
\begin{align*}
	x_3 &= m^2 - 2x_1, \\
	y_3 &= m(x_1-x_3)-y_1,
\end{align*}
hvor $m=(3x_{1}^{2} + A)/2y_1$. På samme måde som før opløfter vi dette til den $q$'ende potens og får, at 
\begin{align*}
	x_{3}^{q} &= {m'}^2 - 2 x_{1}^{q}, \\
	y_{3}^{q} &= m'(x_{1}^{q}- x_{3}^{q}) - y_{1}^{q},
\end{align*}
hvor $m' = (3^q (x_{1}^{q})^2 + A^q)/2^q y_{1}^{q}$.
Idet, at $2, 3, A \in \mathbb{F}_q$ følger det, at $2^q = 2, 3^q = 3$ og $A^q = A$. Dette er altså netop formlen for fordoblingen af punktet $(x_{1}^{q}, y_{1}^{q})$ på den elliptiske kurve $E$. Hvis $A^q \neq A$ ville vi have været på en anden elliptisk kurve. Vi har dermed vist, at $\phi_q $ er en homomorfi for $E$.

Da $\phi_q(x, y) = (x^q, y^q)$ er givet ved polynomier, som specielt er rationale funktioner, er $\phi_q$ en endomorfi. Den har tydeligvis grad $q$. Da $q=0$ i $\mathbb{F}_q$ er den afledte af $x^q$ lig nul, hvilket betyder at $\phi_q$ ikke er separabel.
\end{proof}

\begin{remark}
Da $\phi_q$ er en endomorfi for $E$ er ${\phi_q}^2 = \phi_q \circ \phi_q$ det også og dermed også 
${\phi_q}^n = \phi_q \circ \phi_q \circ \ldots \circ \phi_q$ for $n \geq 1$. Da multiplikation med $-1$ også er en endomorfi er ${\phi_q}^n - 1$ også en endomorfi for $E$.
\end{remark}

\begin{lemma}
\label{lemma03}
Lad $E$ være en elliptisk kurve over $\mathbb{F}_q$, da gælder der
\begin{enumerate}
	\item $\phi_q(x, y) \in E(\overline{\mathbb{F}}_q)$,
	\item $(x, y) \in E(\mathbb{F}_q) \Leftrightarrow \phi_q(x, y)=(x, y)$,
\end{enumerate}
for alle $(x, y) \in E(\clfq)$.
\end{lemma}
\begin{proof}
Vi har, at $y^2 = x^3 + Ax + B$, hvor $A, B \in \mathbb{F}_q$. Vi opløfter 
denne ligning til den $q$'ende potens og får, at 
\begin{align*}
	(y^q)^2 &= (x^q)^3 + A^q (x^q) + B^q \\
	&= (x^q)^3 + A (x^q) + B.
\end{align*}
hvor vi har brugt, at $(a+b)^q = a^q + b^q$ når $q$ er en potens af legemets karakteristik 
og at $a^q = a$ for alle $a \in \mathbb{F}_q$ (se appendiks \ref{appendiks_legemer}).
Men dette betyder netop, at 
$(x^q, y^q) \in E(\overline{\mathbb{F}}_q)$, hvilket viser (1).
For at vise (2) husker vi, at $\phi_q(x) = x \Leftrightarrow x \in \mathbb{F}_q$.
Det følger da, at 
\begin{align*}
	(x, y) \in E(\mathbb{F}_q) &\Leftrightarrow x, y \in \mathbb{F}_q \\
	&\Leftrightarrow \phi_q(x) = x \ \text{og} \ \phi_q(y) = y \\
	&\Leftrightarrow \phi_q(x, y) = (x, y),
\end{align*}
hvilket fuldfører beviset for (2).
\end{proof}

Følgende proposition er vigtig, da den skaber en sammenhæng mellem kernen for $\phi_{q}^{n} - 1$ og antallet af punkter på en elliptisk kurve $E$ over et endeligt legeme $\mathbb{F}_{q}$.

\begin{proposition}
\label{prop_imp}
Lad $E$ være en elliptisk kurve over $\mathbb{F}_q$ og lad $n \geq 1$. Da gælder der,
at 
\begin{enumerate}
	\item $\ker (\phi_{q}^n - 1) = E(\mathbb{F}_{q^n})$. \label{test}
	\item $\phi_{q}^{n}-1$ er separabel, så $\#E(\mathbb{F}_{q^n})=\deg (\phi_{q}^{n}-1)$. 
\end{enumerate}
\end{proposition}
\begin{proof}
Betragter vi $(\phi_{q}^{n} - 1)$ som en endomorfi har vi, at
\begin{align*}
	(\phi_{q}^{n} - 1)(x, y) = 0 \Leftrightarrow (x^{q^n}, y^{q^n}) - (x, y) = 0 
	\Leftrightarrow (x^{q^n}, y^{q^n}) = (x, y).
\end{align*}
Da $\phi_{q}^{n}$ er Frobenius afbildningen for $\mathbb{F}_{q^n}$ følger det at 
\begin{align*}
	\ker (\phi_{q}^{n} - 1) = E(\mathbb{F}_{q^n})
\end{align*} 
fra lemma \ref{lemma03}. At $\phi_{q}^{n} -1$ er separabel vil vi ikke vise, men et bevis kan findes
i \cite[s.~58]{Washington}. Da $\phi_{q}^{n} - 1$ er separabel følger det fra proposition \ref{deg_to_ker}, at
\begin{align*}
	\#E(E_{q^n})=\deg(\phi_{q}^{n} -1).
\end{align*} 
Vi har dermed vist det ønskede.
\end{proof}

\section{Hasses sætning}
Vi skal i dette afsnit vise Hasses sætning nu, da vi har fået etableret de nødvendige resultater vedrørende endomorfier på elliptiske kurver over endelige legemer.
\begin{thm}[Hasse]
\label{hasse}
Lad $E$ være en elliptisk kurve over et endeligt legeme $\mathbb{F}_q$. Da gælder der, at 
\begin{align*}
	|q + 1 - \#E(\mathbb{F}_q)| \leq 2 \sqrt{q}.
\end{align*}
\end{thm}
Vi skal i kapitel \ref{chap:kapitel_faktorisering} hvordan elliptiske kurver over endelige legemer kan benyttes til heltalsfaktorisering, hvilket bl.a. hviler på Hasses sætning. Det skal nævnes, at der også findes et elementært bevis for Hasses sætning af Manin, CITE, men vi vil benytte teorien om endomorfier til at bevise sætningen.
Lad i det følgende 
\begin{align}
	\label{hasse_as}
	a = q + 1 - \#E(\mathbb{F}_q) = q + 1 - \deg(\phi_q - 1).
\end{align}
Da skal vi vise, at $|a| \leq 2 \sqrt{q}$ for at vise Hasses sætning. Først har vi dog følgende lemma

\begin{lemma}
\label{degree_lemma}
Lad $r, s \in \mathbb{Z}$ så $\gcd(s, q) = 1$. Da er 
\begin{align*}
	\deg(r \phi_q - s) = r^2 q + s^2 - rsa.
\end{align*}
\end{lemma}
\begin{proof}
Vi vil ikke give beviset her, da det bygger på en række af tekniske resultater. For et bevis se \cite[s.~100]{Washington}.
\end{proof}

Nu er vi da i stand til, at gives beviset for Hasses sætning:

\begin{proofof}[Bevis for Hasses sætning.]
Da graden af en endomorfi altid er $\geq 0$ følger det fra lemma \ref{degree_lemma}, at 
\begin{align*}
	r^2q+s^2 -rsa = q \left( \frac{r^2}{s^2} \right) - \frac{rsa}{s^2} + 1 
	\geq 0,
\end{align*}
for alle $r, s \in \mathbb{Z}$ med $\gcd(s, q)=1$. Da mængden
\begin{align*}
	\left\{ \frac{r}{s} \mid \gcd(s, q)=1 \right\} \subseteq \mathbb{Q},
\end{align*}
er tæt i $\mathbb{R}$ følger det, at $qx^2 - ax + 1 \geq 0$,
for alle $x \in \mathbb{R}$ (se proposition \ref{appendiks_dense} i \ref{appendiks_andre}). Dette medfører at diskrimanten må være negativ eller lig $0$.
Altså har vi, at 
\begin{align*}
	a^2 - 4q \leq 0 \Rightarrow |a| \leq 2 \sqrt{q},
\end{align*}
hvilket viser Hasses sætning.
\end{proofof}

\section{Torsionspunkter}
Det viser sig, at proposition \ref{prop_imp} også har andre interessante konsekvenser, som vi vil se på her. 

Ordenen af et element $P$ fra en gruppe over en elliptisk kurve er det mindste positive heltal $k$ sådan at 
$kP = P + P + \ldots P = \infty$. Hvis der ikke findes et sådan $k$ siges ordenen af $P$ at være uendelig. Torsionspunkterne er netop de punkter, som har endelig orden. For en elliptisk kurve $E$ og et legeme $K$ definerer vi
\begin{align*}
	E[n] = \{ P \in E(\clk) \mid nP= \infty \}.
\end{align*}
Vi bemærker at alle punkter over et endeligt legeme er et torsionspunkt. Følgende sætning karakteriserer disse grupper af torsionspunkter:

\begin{thm}
\label{torsion_theorem}
Lad $E$ være en elliptisk kurve over et legeme $K$ og lad $n$ være et positivt heltal. Hvis karakteristikken for $K$ ikke deler $n$, eller ikke er $0$, da er
\begin{align*}
	E[n] \simeq \mathbb{Z}_n \oplus \mathbb{Z}_n.
\end{align*}
Hvis karakteristikken for $K$ er $p > 0$ og $p \mid n$, lader vi $n=p^r n'$ sådan at $p \nmid n'$. Da er
\begin{align*}
	E[n] \simeq \mathbb{Z}_{n'} \oplus \mathbb{Z}_{n'} \quad \text{eller} \quad \mathbb{Z}_n 
	\oplus \mathbb{Z}_{n'}.
\end{align*}
\end{thm}
\begin{proof}
Et bevis for denne sætning kan findes i \cite[s.~79]{Washington}.
\end{proof}
En konsekvens af sætning \ref{torsion_theorem} er, at vi snakke om en basis for $E[n]$, da vi nu ved hvordan den ser ud.

Lad da $\{\beta_1, \beta_2\}$ være en basis for $E[n] \simeq \mathbb{Z}_n \oplus \mathbb{Z}_n$. Ethvert element fra $E[n]$ kan altså skrives som $\beta_1 m_1 + \beta_2 m_2$, hvor $m_1, m_2 \in \mathbb{Z}$ er entydige mod $n$. For en homomorfi $\alpha : E(\clk) \to E(\clk)$ afbilleder $\alpha$ torsionspunkterne $E[n]$ til $E[n]$, derfor findes $a, b, c, d \in \mathbb{Z}$ sådan, at 
\begin{align*}
	\alpha(\beta_1) = a \beta_1 + b \beta_2, \quad \alpha(\beta_2) = c\beta_1 + d \beta_2.
\end{align*}
Vi kan altså repræsentere en sådan homomorfi med matricen
\begin{align*}
	\alpha_n = \left( 
	\begin{matrix}
		a & b \\ 
		c & d 
	\end{matrix} \right).
\end{align*}
Med disse detaljer på plads er vi i stand til at vise følgende sætning:
\begin{thm}
\label{trace_theorem}
Lad $E$ være en elliptisk kurve over $\mathbb{F}_q$. Lad $a$ være som i \eqref{hasse_as}. Da er $a$ det
entydige heltal så
\begin{align*}
	{\phi_q}^2 - a \phi_q + q = 0,
\end{align*}
set som endomorfier. Med andre ord er $a$ det entydige heltal sådan, at 
\begin{align*}
	({x^q}^2, {y^q}^2) - a(x^q, y^q) + q(x, y) = \infty,
\end{align*}
for alle $(x, y) \in E(\clfq)$. Desuden er $a$ det entydige heltal der opfylder, at
\begin{align*}
	a \equiv \text{Trace}((\phi_q)_m) \modu{m},
\end{align*}
for alle $m$, hvor $\gcd(m, q)=1$.
\end{thm}

\begin{proof}
Det følger direkte fra lemma \ref{deg_to_ker} at hvis $\phi_{q}^{2} -a \phi_q + q \neq 0$ (hvis den ikke er nul-endomorfier) er dens kerne endelig. Så hvis vi kan vise, at kernen er uendelig, da må endomorfien være lig $0$.

Lad nu $m \geq 1$ være valgt sådan, at $\gcd(m, q) = 1$. Lad da $(\phi_q)_m$ være den matricen, som beskriver virkningen af $\phi_q$ på $E[m]$, som vi beskrev ovenfor. Lad da
\begin{align*}
	(\phi_q)_m = \left( 
	\begin{matrix}
		s & t \\
		u & v
	\end{matrix} \right).
\end{align*}
Da $\phi_q - 1$ er separabel (se proposition \ref{prop_imp}) følger det fra proposition \ref{deg_to_ker} og fra det faktum, at $\det(\alpha_n) \equiv \deg(\alpha) \modu{n}$ (se \cite[proposition~3.15]{Washington}), at
\begin{align*}
	\# \ker (\phi_q - 1) &= \deg(\phi_q - 1) \equiv \det((\phi_q)_m - I) \\ 
	&= 
	\left| \begin{matrix}
		s-1 & t \\
		u & v - 1 
	\end{matrix} \right| \\
	&= sv - tu - (s + v) + 1 \modu{m}.
\end{align*}
Videre har vi, at $sv-tu = \det((\phi_q)_m) \equiv q \modu{m}$. Fra \eqref{hasse_as} har vi, at 
$\# \ker(\phi_q - 1) = q + 1 - a$, så
\begin{align*}
	\text{Trace}((\phi_q)_m) = s + v \equiv a \modu{m}.
\end{align*}
Idet vi husker, at $X^2 - aX + q$ er det karakteristiske polynomium for $(\phi_q)_m$ følger det fra
Cayley-Hamiltons sætning fra lineær algebra, at
\begin{align*}
	(\phi_q)_{m}^{2} - a(\phi_q)_m + qI \equiv 0 \modu{m},
\end{align*}
hvor $I$ er $2 \times 2$ identitetsmatricen. Vi har da, at endomorfien $\phi_{q}^{2} -a\phi_q + q$ er nul på $E[m]$. Da der er uendeligt mange muligheder for valget af $m$ er kernen for ${\phi}^2 - a\phi_q + q$ uendelig. Dermed er endomorfien lig $0$.

For at vise entydigheden af $a$ lader vi nu $a_1 \neq a$ være sådan, at 
\begin{align*}
	\phi_{q}^{2} - a_1 \phi_q + q = 0,
\end{align*}
er opfyldt. Da har vi også, at (ved at lægge $0$ til)
\begin{align*}
	(a - a_1) \phi_q = (\phi_{q}^{2} - a_1 \phi_q + q) - (\phi_{q}^{2} - a\phi_q + q) = 0
\end{align*}
Da $\phi_q : E(\clfq) \to E(\clfq)$ er surjektiv betyder det, at $(a - a_1)$ annihilerer $E(\clfq)$ (dvs. for hvert element $x \in E(\clfq)$ er $(a-a_1)x = 0$). Specielt har vi, at $(a - a_1)$ annihilerer $E[m]$ for hvert $m \geq 1$. Men da der er punkter i $E[m]$ med orden $m$ når $\gcd(m, q) =1$ har vi, at $a - a_1 \equiv 0 \modu{m}$ for sådan et $m$. Dermed er $a - a_ 1 = 0$ og vi har vist, at $a$ er entydig.
\end{proof}

Endeligt vil vi vise en sætning, som gør det muligt at bestemme ordenen af en gruppe af punkter for en elliptisk kurve. Hvis vi kender ordenen af $E(\mathbb{F}_q)$ for et lille endeligt legeme gør følgende sætning det muligt, at bestemme ordenen af $E(\mathbb{F}_{q^n})$.

\begin{thm}
\label{count}
Lad $\#E(\mathbb{F}_q) = q + 1 - a$. Skriv $X^2 - aX + q= (X-\alpha)(X-\beta)$. Da er
\begin{align*}
	\#E(\mathbb{F}_{q^n}) = q^n + 1 - (\alpha^n + \beta^n),
\end{align*}
for alle $n \geq 1$.
\end{thm}
Vi har brug for, at $\alpha^n + \beta^n$ er et heltal, hvilket følgende lemma giver os:

\begin{lemma}
Lad $s_n = \alpha^n + \beta^n$. Da er $s_0=2$, $s_1=a$ og $s_{n+1}=as_n - qs_{n-1}$ for alle $n \geq 1$.
\end{lemma}
\begin{proof}
Bemærk først, at $s_0 = \alpha^0 + \beta^0 = 2$ og $s_1 = a$.
Vi ser, at 
\begin{align*}
	(\alpha^2 - a\alpha + q) \alpha^{n-1} = \alpha^{n+1} - a \alpha^n + q\alpha^{n-1} = 0,
\end{align*}
da $\alpha$ er en rod i $X^2 - aX + q$. Altså har vi, at $\alpha^{n+1}=a \alpha^n - q\alpha^{n-1}$. På samme måde har vi, at $\beta^{n+1}=a \beta^n - q \beta^{n-1}$, da $\beta$ også er en rod. Lægges disse udtryk sammen får vi, at
\begin{align*}
	s_{n+1} = \alpha^{n+1} + \beta^{n+1} &= a \alpha^n - q \alpha^{n-1} + a \beta^n - q \beta^{n-1} \\
	&= a (\beta^n + \beta^n) - q (\alpha^{n-1} + \beta^{n-1}) \\
	&= a s_n - q s_{n-1}.
\end{align*}
Dermed er $s_n$ et heltal for alle $n \geq 0$.
\end{proof}

\begin{proofof}[Bevis for sætning \ref{count}.]
Lad først
\begin{align*}
	f(X) = (X^n - \alpha^n)(X^n - \beta^n) = X^{2n} - (\alpha^n + \beta^n)X^n + q^n.
\end{align*}
Da deler $X^2 - aX + q = (X- \alpha)(X-\beta)$ polynomiet $f(X)$. Kvotienten er et polynomium $Q(X)$ med heltallige koefficienter, da $X^2 - aX + q$ er monisk og $f(X)$ har heltallige koefficienter (se sætning \ref{monic} i appendikset). Derfor er
\begin{align}
	\label{alpha_beta}
	f(\phi_q) = (\phi_{q}^{n})^2 - (\alpha^n + \beta^n) \phi_{q}^{n} + q^n
	= Q(\phi_q)(\phi_{q}^{2} - a \phi_q + q) = 0,
\end{align}
som endomorfier for $E$ pr. sætning \ref{trace_theorem}. Idet vi husker, at $\phi_{q}^{n} = \phi_{q^n}$ giver sætning \ref{trace_theorem} også, at der findes entydigt $k \in \mathbb{Z}$ sådan at $\phi_{q^n}^{2} - k \phi_{q}^{n} + q^n = 0$. Sådan et $k$ er givet ved $k=q^n + 1 - \#E(\mathbb{F}_{q^n})$, og dette sammen med \eqref{alpha_beta} giver os netop, at
\begin{align*}
	\alpha^n + \beta^n = q^n + 1 - \#E(\mathbb{F}_{q^n}),
\end{align*}
hvilket netop var hvad  vi ønskede at vise.
\end{proofof}

\begin{example}
Eksempel på 4.12 i aktion.
\end{example}





\chapter{Faktoriseringsalgoritmer}

I dette kapitel ønsker vi at se på faktoriseringsalgoritmer. Det viser sig nemlig, at en af de anvendelser som elliptiske kurver besidder, er indenfor faktoriseringen af heltal. Faktoriseringsproblemet, altså hvordan man bestemmer en faktor for et tal $n$ er yderst relevant, da alle heltal kan faktoriseres:

\begin{theorem}[Aritmetikkens fundamentalsætning]
Et heltal $n > 1$ kan faktoriseres entydigt som et produkt af primtal, så hvis
\begin{align*}
	p_1 \cdot p_2 \cdot \ldots \cdot p_k = q_1 \cdot q_2 \cdot \ldots \cdot q_,
\end{align*}
hvor $p_i$ og $q_j$ er primtal for $1 \leq i \leq k$ og $1 \leq j \leq l$ er $k = l$ og $p_i = q_i$ for alle $i=1, 2, \ldots, k$ (efter eventuelle ombytninger). Desuden er faktorerne $p_{1}^{n_1}, p_{2}^{n_2}, \ldots, p_{k}^{n_k}$ entydigt bestemte.
\end{theorem}

For et bevis af sætningen se f.eks. \cite{Hansen}. Det vigtige at bemærke er, at beviset ikke er konstruktivt og dermed ikke giver os en måde, hvorpå vi kan finde disse faktorer. Men hvordan kan vi så finde disse faktorer, som vi nu ved findes? Hvis vi har et sammensat tal $n$, som vi ønsker at faktorisere kunne vi angribe problemet med en naiv tilgang. Vi antager for nemhedens skyld at $n = pq$, hvilket gør det klart at $\min \{p, q \} \leq \sqrt{n}$. Vi kan altså finde en faktor ved at undersøge om først $2 \mid n$, dernæst om $3 \mid n$ osv. indtil at vi finder en faktor, hvilket vil ske senest når vi når til $\sqrt{n}$. Denne løsning er fin for tilstrækkeligt små tal, men det bliver hurtigt uoverkommeligt for store tal (eksempler på hvor lang tid det tager?). 

Sikkerheden i moderne kryptosystemer hviler på dette faktum, at det tager lang tid at faktorisere et heltal. Derfor er det interessant at undersøge om man gøre det hurtigere end den med den naive tilgang. Vi skal se på to af sådanne algoritmer, nemlig Pollards $p-1$ algoritme og Lenstras algoritme, som benytter elliptiske kurver til at finde en faktor.

\section{Pollards $p-1$ algoritme}

Vi starter med at se på Pollards $p-1$ algoritme, da Lenstras algoritme er stærkt inspireret af denne og delvist kan ses som en analog til den, hvilket gør det naturligt at betragte den først. Pollards $p-1$ algoritme blev først præsenteret i \cite{Pollard} i 1970'erne af J. M. Pollard. Algoritmen hviler på Fermats lille sætning:

\begin{theorem}[Fermats lille sætning]
\label{fermats_small_theorem}
Lad $p$ være et primtal som ikke går op i $a$. Da gælder der, at
\begin{align*}
	a^{p-1} = 1 \modu{p}.
\end{align*}
\end{theorem}

\noindent Et bevis findes i appendikset.

Vi kan da se på, hvordan Pollards $p-1$ algoritme virker. Lad $n$ være et sammensat tal og lad $p$ være en primfaktor for $n$. Vi ved fra Fermats lille sætning, at $a^{p-1} \equiv 1 \modu{p}$ når $\gcd(a, p) = 1$. Hvis vi da kendte $p-1$ kunne vi bestemme $p$ (udover den åbenlyse måde) ved
\begin{align*}
	\gcd(a^{p-1} - 1, n) = p. 
\end{align*}
(måske et multiplum af $p$?), da hvis $x \equiv 1 \modu{l}$, hvor $l$ er en faktor i $n$, er $\gcd(x-1, n)$ divisibel med denne faktor $l$.

Vi kender dog ikke $p-1$ og vi kan derfor ikke foretage denne udregning. Det viser sig dog, at vi kan nøjes med et multiplum af $p-1$, da
\begin{align*}
	a^{t(p-1)} - 1 = (a^{p-1})^k - 1 \equiv 1^k - 1 \equiv 0 \modu{p}.
\end{align*}
Idéen er da, at vi vælger et heltal
\begin{align*}
	k = 2^{e_2} \cdot 3^{e_3} \cdot 5^{e_5} \cdots r^{e_r},
\end{align*}
hvor $2, 3, \ldots, r$ er primtal og $e_1, e_2, \ldots, e_r$ er små positive heltal. Vi udregner da $\gcd(a^k - 1, n)$. Hvis vi er i det heldige tilfælde, hvor $n$ har en faktor sådan, at $p-1 \mid k$, da vil $p \mid a^k - 1$ og
vi har så, at
\begin{align*}
	\gcd(a^k - 1, n) \geq p > 1.
\end{align*}
Hvis $\gcd(a^k - 1, n) \neq n$ har vi altså fundet en ikke-triviel faktor for $n$ og vi kan dele $n$ i to faktorer og gentage de ovenstående trin. Hvis vi derimod har, at $\gcd(a^k -1, n) = n$ vælger vi et andet $a$ og forsøger igen, og hvis $\gcd(a^k-1,n)=1$ vælger vi et større $k$.

Dette er tankegangen i Pollards $p-1$ algoritme og vi opsummerer det i algoritmen:

\begin{algorithm}[Pollards $p-1$ algoritme]
Lad $n \geq 2$ være et sammensat tal, som er tallet vi ønsker at 
finde en faktor for.
\begin{enumerate}
	\item Vælg $k \in \mathbb{Z}^+$ sådan, at $k$ er et produkt
	af mange små primtal opløftet i små potenser. Eksempelvis kan $k$ vælges til at være
	\begin{align*}
		k = \lcm [1, 2, \ldots, K],
	\end{align*}
	for et $K \in \mathbb{Z}^+$ og hvor $\lcm$ er det mindste fælles multiplum.
	\item Vælg et heltal $a$ sådan, at $1 < a <n$.
	\item Udregn $\gcd(a, n)$. Hvis $\gcd(a, n) > 1$ har vi fundet en
	ikke-triviel faktor for $n$ og vi er færdige. Ellers fortsæt til næste
	trin.
	\item Udregn $D = \gcd(a^k - 1, n)$. Hvis $1 < D < n$ er $D$ en ikke-triviel
	faktor for $n$ og vi er færdige. Hvis $D = 1$ gå da tilbage til trin 1
	og vælg et større $k$. Hvis $D = n$ gå da til trin 2 og vælg et nyt $a$. 
\end{enumerate}
\end{algorithm}

Følgende er et eksempel på anvendelsen af Pollards algoritme,
hvor det går godt, altså hvor $p-1$ har små primfaktorer:

\begin{example}
Vi vil forsøge at faktorisere 
\begin{align*}
	n = 30042491.
\end{align*}
Vi ser at $2^{n-1} = 2^{30042490} \equiv 25171326 \ (\textrm{mod}\ 30042491)$,
så $N$ er ikke et primtal. Vi vælger som beskrevet i algoritmen
\begin{align*}
	a = 2 \quad \text{og} \quad k = \mathrm{LCM}[1,2, \ldots, 7] = 420.
\end{align*}
Da $420 = 2^2 + 2^5 + 2^7 + 2^8$ skal vi udregne $2^{2^i}$ for 
$0 \leq i \leq 8$. Dette resulterer i følgende tabel:

\begin{center}
\begin{tabular}{c c c c}
$i$ & $2^{2^i} \modu{n}$ & & \\ 
\hline 
1 & 4 & 5 & 28933574 \\ 
2 & 16 & 6 & 27713768 \\ 
3 & 256 & 7 & 10802810 \\ 
4 & 65536 & 8 & 16714289 \\ 
5 & 28933574 & & 
\end{tabular} 
\end{center}

Denne tabel gør det forholdsvist let for os, at bestemme
\begin{align*}
	2^{420} &= 2^{2^2 + 2^5 + 2^7 + 2^8} \\
	&\equiv 16 \cdot 28933574 \cdot 10802810 \cdot 16714289
	\modu{30042491} \\
	&\equiv 27976515 \modu{30042491}.
\end{align*}

Ved anvendelse af den euklidiske algoritme finder vi dernæst, at
\begin{align*}
	\mathrm{gcd}(2^{420} - 1 \modu{n}, n) = \mathrm{gcd}(27976514, 30042491) = 1.
\end{align*}
Her fejler testen altså og vi er nået frem til, at $N$ ikke har nogle 
primtalsfaktorer $p$ sådan, at $p-1$ deler $420$. Algoritmen foreskriver da, at vi skal vælge et nyt $k$. Vi lader
\begin{align*}
	k = \mathrm{LCM}[1,2, \ldots, 11] = 27720.
\end{align*}
Da $27720 = 2^{14} + 2^{13} + 2^{11} + 2^{10} + 2^{6} + 2^3$ skal vi udvide tabellen til at indeholde værdierne for $2^{2^i}$ for $0 \leq i \leq 14$:

\begin{center}
\begin{tabular}{c c c c}
$i$ & $2^{2^i} \modu{n}$ & &  \\ 
\hline 
9 & 19694714 & 12 & 26818902 \\ 
10 & 3779241 & 13 & 8658967 \\ 
11 & 11677316 & 14 & 3783587 \\ 
\end{tabular} 
\end{center}

Vi fortsætter på samme måde, som vi gjorde før og bestemmer
\begin{align*}
	2^{27720} &= 2^{2^3 + 2^{2^6} + 2^{2^{10}} + 2^{2^{11}} + 2^{2^{13}} + 2^{2^{14}}} \\
	&= 256 \cdot 27713768 \cdot 3779241 \cdot 11677316 \cdot 8658967 \cdot 3783587 \\
	&= 16458222 \modu{30042491}.
\end{align*}
Vi finder dernæst, at
\begin{align*}
	\gcd(2^{27720} - 1 \modu{n}, n) = \gcd(16458221, 30042491) = 9241,
\end{align*}
hvilket betyder at vi har fundet en ikke-triviel faktor for $n$. Mere præcist har vi fundet faktoriseringen
\begin{align*}
	30042491 = 3251 \cdot 9241.
\end{align*}

\end{example}
\section{Lenstras elliptiske kurve algoritme}
I \cite{Lenstra}
Lenstra præsenterede H. W. Lenstra en algoritme til faktorisering af heltal, som anvender elliptiske kurver. Vi skal benytte koncepterne vi udviklede til gruppeloven, men vi skal anvende dem på $E(\mathbb{Z}_n)$, hvor $n$ er et sammensat tal. Når $n$ er et sammensat tal er $\mathbb{Z}_n$ ikke er legeme og derfor er $E(\mathbb{Z}_n)$ heller ikke en gruppe. Derfor vælger vi at definere elliptiske pseudokurver for at kunne give mening til dette:

\begin{definition}
Lad $A, B \in \mathbb{Z}_n$, hvor $\gcd(n, 6) = 1$. En elliptisk pseudokurve er da mængden
\begin{align*}
	E(\mathbb{Z}_n) = \{ \infty \} \cup \{ (x, y) \in \mathbb{Z}_n \times \mathbb{Z}_n \mid y^2 =x^3 + Ax +B \},
\end{align*}
hvor $\gcd(4A^3 + 27B^2, n) = 1$.
\end{definition}
Med denne definition følger det, at en elliptisk kurve specielt er en elliptisk pseudokurve, da hvis $p$ er et primtal er $\mathbb{Z}_p = \mathbb{F}_p$ som er et endeligt legeme. Vi foretager addition af punkter på en elliptisk pseudokurve på samme måde, som for elliptiske kurver. På grund af denne definition vil vi for to punkter $P$ og $Q$ på en elliptisk pseudokurve kunne komme ud for at $P+Q$ ikke vil være defineret. Et tilfælde hvor $P+Q$ ikke er defineret vil blive fanget i udregningen af hældningen $m$ i definition \ref{gruppeloven:definitionen}, da $\mathbb{Z}_n$ ikke er et legeme når $n$ er et sammensat, så det der går galt er at $x_2 - x_1$ eller $y_1$ ikke har en invers i $\mathbb{Z}_n$. At en addition kan "gå galt" er motivationen for at kalde disse kurver for pseudokurver.

Vi ser nu på, hvordan vi kan finde en faktor, når additionen af to punkter ikke er defineret. Algoritmen vi her præsenterer er inspireret af algoritmerne i \cite{Silverman} og \cite{Washington}. Lad $n$ være et sammensat tal. Vi husker fra gruppeloven, at vi ved additionen af to punkter skal bruge værdien af inverserne til $x_2 - x_1$ og $y_1$ alt efter, hvilket tilfælde vi er i. Disse inverser eksisterer kun modulo $n$, hvis (se bevis i appendiks)
\begin{align*}
	\gcd(x_1 - x_2, n) = 1 \quad \text{og} \quad \gcd(y_1, n) = 1.
\end{align*}
Men hvis vi er i stand til at finde punkter $P=(x_1, y_1)$ og $Q=(x_2, y_2)$ sådan, at summen $P+Q$ ikke er defineret, da er $\gcd(x, n) > 1$ hvor $x=x_1-x_2$ eller $x=y_1$ og dermed har vi muligvis fundet en ikke-triviel faktor i $n$. Vi kan da se på, hvordan vi kan udnytte dette faktum til at lave en algoritme, som giver os en faktor i $n$. Vælg først tilfældige heltal $x, y, A$ mellem $1$ og $n$, og sæt $B = y^2 - x^3 - Ax \modu{n}$. Vi har da en elliptisk kurve (egentlig ikke en elliptisk kurve, da $n$ er et sammensat tal, men vi lader som om)
\begin{align*}
	E : y^2 = x^3 + Ax + B,
\end{align*}
hvor vi ved at punktet $P=(x, y)$ er placeret. Vi tjekker da, at
\begin{align*}
	d= \gcd(4A^3 + 27B^2, n) = 1.
\end{align*}
Hvis det ikke er tilfældet og $1 < d < n$ har vi fundet en faktor i $n$. Hvis $d = n$ vælger vi et nyt $A$. For et heltal $K$ lader vi $k=\lcm[1, 2, \ldots, K]$ og vi forsøger da, at bestemme
\begin{align*}
	kP = \underbrace{P + P + \ldots + P}_{k \ \text{led}}.
\end{align*}
Det er ikke praktisk at udregne $P+P+ \ldots + P$ så vi gør ligesom i Pollards $p-1$ algoritme og skriver $k$ som den binære udvidelse
\begin{align*}
	k = k_0 + k_1 \cdot 2 + k_2 \cdot 2^2 + k_ 3 \cdot 2^3 + \ldots + k_r \cdot 2^r,
\end{align*}
hvor alle $k_i$ er $0$ eller $1$. Vi kan da udregne
\begin{align*}
	P_0 &= P \\
	P_1 &= 2P_0 = 2P \\
	P_2 &= 2P_1 = 2^2 P \\
	&\vdots \\
	P_r &= 2 P_{r-1} = 2^r P.
\end{align*}
Da kan vi bestemme $kP= (\text{summen af $P_i$'erne hvor $k_i = 1$})$. I hver udregning regner vi modulo $n$, da tallene ellers bliver alt for store og meget langsommelige at arbejde med. Vi håber på, at der i løbet af de udregninger er en addition, som ikke kan lade sig gøre og at dette giver os vores faktor. 


Med algoritmen på plads kan vi nu se på et eksempler:




Vi opsummerer diskussionen i algoritmen nedenfor:

\begin{algorithm}[Lenstras algoritme]
Lad $n \geq 2$ være et sammensat tal.
\begin{enumerate}
	\item Vælg heltal $x, y$ og $A$ mellem $1$ og $n$. Lad da $B = y^2 - x^3 - Ax \modu{n}$, så vi har den 	 
	elliptiske kurve
	\begin{align*}
		E : y^2 = x^3 + Ax + B, 
	\end{align*}
	hvor punktet $P=(x, y)$ er placeret.
	\item Tjek at $d = \gcd(4A^3 + 27B^2, n) = 1$. Hvis $d=n$ går vi tilbage
	til (1) og vælger et nyt $B$. Hvis $1 < d < n$ har vi fundet en faktor af $n$ og vi er færdige.
	\item Vælg et positivt heltal $k$ som et produkt af mange små primtal, lad eksempelvis
	\begin{align*}
		k = \lcm[1, 2, 3, ..., K],
	\end{align*}
	hvor $K \in \mathbb{Z}^+$.
	\item Forsøg at bestemme $kP = P + P + \ldots + P$. Hvis udregningen kan lade sig gøre går vi tilbage til (1) og 
	vælger en ny kurve, eller går til (3) og vælger et større $k$.
\end{enumerate}
\end{algorithm}
For at se hvorfor der er en god chance for, at vi støder på et valg af $x, y, A$ sådan at vi finder en faktor, lader vi $p$ være en primfaktor i $n$. Til den elliptiske kurve $E$ har vi den abelske gruppe $E(\mathbb{F}_p)$ og pr. sætning \ref{hasse} ved vi, at
\begin{align*}
	p + 1 - 2 \sqrt{p} < \#E(\mathbb{F}_p) < p + 1 + 2 \sqrt{p}.
\end{align*} 
En sætning af Deuring \cite{Deuring} siger, at for ethvert heltal $m \in (p+1-2\sqrt{p}, p+1+2\sqrt{p})$ findes der talpar $(A, B)$ i mængden
\begin{align*}
	\{ (A, B) \mid A, B \in \mathbb{F}_p, 4A^3 +27B^2 \neq 0 \},
\end{align*}
sådan at $\#E(\mathbb{F}_p) = m$. For hvert tal i intervallet findes der altså en elliptisk kurve, som har denne orden. Der er altså en positiv sandsynlighed for, at vi kan finde sådan en kurve.

Lenstras sandsynlighedsteoretiske overvejelser.

\begin{example}
Lad nu 
\begin{align*}
	n = 753161713
\end{align*}
være det tal, som vi ønsker at faktorisere. Da $2^{n-1} = 437782651 \modu{n}$ er $n$ ikke et primtal. Vi vælger da
$x = 0$, $y = 1$ og $A=164$. Vi har dermed, at $B = 1^2 - 0^3 - 164 \cdot 0 = 1$ og den elliptiske kurve vi vil arbejde over bliver
\begin{align*}
	E : y^2 = x^3 + 164x + 1,
\end{align*}
hvorpå punktet $P=(0, 1)$ er placeret. Vi ser, at 
\begin{align*}
	D &= \gcd(4 \cdot 164^3 + 27 \modu{753161713}, 753161713) \\ &= \gcd(17643803, 753161713) = 1,
\end{align*}
så vi fortsætter derfor med algoritmen. Vi lader
\begin{align*}
	k = \lcm[1, 2, \ldots, 10] = 2520.
\end{align*}
Da $2520 = 2^{11} + 2^8 + 2^7 + 2^6 + 2^4 + 2^3$ skal vi beregne $2^i P \modu{753161713}$ for $0 \leq i \leq 11$. Dette gøres med additionsformlen og vi opsummerer vores resultater i tabellen nedenfor:

%	Større eksempel, men som ikke er praktisk udskrevet her:
%	n =  1688955439703788849,
\begin{center}
\begin{tabular}{c c c c}
$i$ & $2^i P \modu{753161713}$ & $i$ & $2^i P \modu{753161713}$ \\ 
\hline 
0 & $(0, 1)$ & 6 & $(743238772, 703386057)$  \\ 
1 & $(6724, 752610344)$ & 7 & $(309161840, 219780637)$  \\ 
2 & $(293427237, 450490340)$ & 8 & $(116974611, 722899047)$ \\ 
3 & $(468952095, 385687511)$ & 9 & $(329743899, 182819134)$ \\ 
4 & $(288125200, 446796094)$ & 10 & $(163952469, 456288424)$ \\ 
5 & $(106753239, 115973502)$ & 11 & $(15710788, 301760412)$
\end{tabular} 
\end{center}
Vi kan nu addere disse punkter igen vha. additionsformlerne, hvor vi stadigvæk regner modulo $n$:
\begin{align*}
	(2^3 + 2^4)P &= (606730980, 447512524). 
\end{align*}
Algoritmen giver os en faktor netop når additionen bryder sammen, hvilket kan ske da $\mathbb{Z}/n \mathbb{Z}$ ikke er et legeme. Dette problem viser sig i dette eksempel allerede ved den næste addition, hvor vi forsøger at udregne
\begin{align*}
	(2^3 + 2^4 + 2^6) P = (743238772, 703386057) \\ + (606730980, 447512524) \modu{n}.
\end{align*}
For at denne addition skal kunne lade sig gøre, skal differensen af deres $x$-koordinater have en invers modulo $n$. Dette er kun tilfældet, hvis $\gcd(x_2 - x_1, n) = 1$ (se appendiks, sætning k). Men vi ser, at
\begin{align*}
	\gcd(606730980 - 743238772, 753161713) = 19259,
\end{align*}
så der findes altså ikke en invers, men vi har i stedet fundet en faktor i $n$. Dermed har vi faktoriseringen
\begin{align*}
	753161713 = 19259 \cdot 39107.
\end{align*}
Nu kan det virke til, at det var spild da vi lavede hele tabellen, men i beregningerne af $2^i P \modu{753161713}$ ville vi også kunne have løbet ind i et element, som ikke havde en invers og som dermed kunne give os en faktor.
\end{example}




\appendix
\chapter{Legemer}
\label{appendiks_legemer}
Lad $p$ være et primtal. Heltallene modulo $p$ giver os et legeme $\mathbb{F}_p = \mathbb{Z}/p\mathbb{Z}$ der indeholder $p$ elementer. Antallet af elementer i ethvert endeligt legeme er på formen $p^n$. For at se dette lader vi $K$ være et endeligt legeme. Dets karakteristik må være $p$ for et eller andet primtal, da et legeme med karakteristik $0$ er uendeligt. Derfor må $K$ være endeligt frembragt som et vektorrum over $\mathbb{Z}/p\mathbb{Z}$. Så lad $x_1, \ldots, x_n$ være en basis for $K$. Elementerne i $K$ kan da skrives entydigt som
\begin{align*}
	x= \alpha_1 x_1 + \alpha_2 x_2 + \ldots + \alpha_n x_n, \quad \text{hvor} \quad \alpha_i \in \mathbb{Z}/p\mathbb{Z}.
\end{align*}
Da hvert $\alpha_i \in K$ har vi altså $p$ forskellige valg for hver af $\alpha_1, \alpha_2, \ldots, \alpha_n$. Dette betyder, at vi har $p^n$ valg for $x$, som også giver os hele legemet $K$. Antallet af elementer i $K$ er altså $p^n$. 

Vi benytter notationen $\mathbb{F}_q = \mathbb{F}_{p^n}$ for det endelige legeme med $q=p^n$ elementer. Bemærk, at $\mathbb{Z}/p^n \mathbb{Z}$ ikke er et legeme for $n \geq 2$, da $p$ ikke har nogen multiplikativ invers.



\chapter{Talteoretiske resultater}
Her samler vi nogle af de (hovedsagligt) mindre resultater, som benyttes igennem kapitlerne. De præsenteres her kort og henvises til  i opgaven, når de er blevet anvendt.

\begin{proposition}
\label{inverse_exists_not}
Et element $a \in \mathbb{Z}_n = \mathbb{Z}/n\mathbb{Z}$ har ikke en invers i $\mathbb{Z}_n$ hvis $\gcd(a, n) > 1$. 
\end{proposition}
\begin{proof}
Antag for modstrid, at $d = \gcd(a, n) > 1$, men at der samtidigt eksisterer en invers $c$ til $a$ modulo $n$. Da $d = \gcd(a, n)$ findes et heltal $e$, som ikke er nul, sådan at $de = n$. Da $d >1$ har vi også, at $|e| < |n|$ så $e$ er ikke nul modulo $n$. Da $d$ deler $a$ har vi, at $n = de$ deler $ae$ så $ae = 0 \modu{n}$. Vi har altså, at
\begin{align*}
	e = e \cdot 1 = eac = 0 \cdot c = 0 \modu{n},
\end{align*}
hvilket er i modstrid med at $e$ ikke kunne være $0$ modulo $n$. Altså har $a$ ikke en invers når $\gcd(a, n) > 1$.
\end{proof}

Vi giver nu beviset for sætning 
\begin{proof}[Bevis for Fermats lille sætning]
Vi ser først på de $p-1$ positive multipla af $a$
\begin{align}
	\label{multiples}
	a, 2a, \ldots, (p-1)a.
\end{align}
Hvis $ra = sa \modu{p}$ har vi, at $r = s \modu{p}$, så elementerne listet i \eqref{multiples} er forskellige og ikke-nul. De må altså være kongruente til $1, 2, \ldots, p-1$ men ikke nødvendigvis i den opskrevne rækkefølge. Ganger vi elementerne sammen må de to kongruenser være de samme, altså er
\begin{align*}
	a \cdot 2a \cdots (p-1)a = 1 \cdot 2 \cdots (p-1) \modu{p},
\end{align*}
hvilket giver os, at 
\begin{align*}
	a^{p-1} (p-1)! = (p-1)! \modu{p} \Rightarrow a^{p-1} = 1 \modu{p}.
\end{align*}


\end{proof}
%\printbibliography
\end{document}