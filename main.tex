% Vi benytter os af MEMOIR (pr. daleif's anbefaling)
\documentclass[a4paper, 12pt]{memoir}

% Dette er hovedfilen, hvor jeg definerer min pre-amble.
% Idéen er at sprede de enkelte kapitler ud i hvert deres
% dokument for at gøre det hele lidt mere overskueligt.
% Eventuelle makroer vil også fremgå heri.
% Inden 6 måneder skulle det gerne ligne et bachelorprojekt. 
% 24.01.2013

% Matematikpakker
\usepackage{amsmath}
\usepackage{amsfonts}
\usepackage{amssymb}
\usepackage{amsthm}

% Dansk babel og tillad øæå osv.
\usepackage[danish]{babel}
\usepackage[utf8]{inputenc}

% Styling for sætninger osv., er valgt efter normal anbefaling
\theoremstyle{plain}
\newtheorem{theorem}{Sætning}
\newtheorem{proposition}{Proposition}
\newtheorem{lemma}{Lemma}
\newtheorem{corollary}{Korollar}

\theoremstyle{definition}
\newtheorem{definition}{Definition}
\newtheorem{remark}{Bemærkning}
\newtheorem{example}{Eksempel}
\newtheorem{algorithm}{Algoritme}

% Sørg for at niveauet til og med subsection nummereres
\setcounter{secnumdepth}{2}
\setcounter{tocdepth}{1}

% TikZ bruges til tegninger
\usepackage{tikz}

% -- Makroer der gør mit liv lettere
% Angiver punktet i uendelighed
\newcommand{\infp}{\mathcal{O}}

% Skriver LCM med tekst
\newcommand{\lcm}{\mathrm{LCM}}		

% Hurtig måde at lave modulo
\newcommand{\modu}[1]{\ (\mathrm{mod} \ #1)}


\begin{document}

\tableofcontents


\chapter{Etablering af gruppestrukturen}
I dette kapitel vil vi introducere elliptiske kurver.
Det viser sig, at være muligt at påføre de elliptiske kurver en gruppestruktur ved en geometrisk addition af punkter fra en sådan kurve. Vi vil indføre denne additionslov og vise, at det resulterer i en abelsk gruppe. For at kunne gøre dette skal vi desuden anvende projektiv geometri, som også vil blive introduceret.

\section{Elliptiske kurver}
For at kunne snakke om elliptiske kurver skal vi først og fremmest have defineret, hvad en elliptisk kurve er. Vi vil i denne tekst benytte følgende definition:

\begin{definition}
En elliptisk kurve $E$ er grafen for en ligning
\begin{align}
	\label{elliptic_curve}
	y^2 = x^3 + Ax + B,
\end{align}
hvor $A, B \in K$ er konstanter og $4A^3 + 27B^2 \neq 0$. Denne type elliptisk kurve siges at være på Weierstrass normalform.
\end{definition}

Da $\Delta = -16(4A^3 + 27B^2)$ er diskriminanten for \eqref{elliptic_curve} betyder det, at vi ikke tillader multiple rødder for en elliptisk kurve. Altså har kurvens rødder alle multiplicitet $1$. Der findes mere generelle definitioner af elliptiske kurver, men når vi arbejder over legemer som ikke har karakteristik $2$ eller $3$, kan vi altid skrive en elliptisk kurve på Weierstrass normalform.






\section{Det projektive plan}
Som tidligere nævnt vil vi etablere en gruppestruktur på de elliptiske
kurver. For at kunne gøre dette får vi brug for det projektive plan 
$\mathbb{P}^2$. Rent intuitivt kan man se det projektive plan, som
værende den affine plan 
\begin{align*}
	\mathbb{A}^2(K) = \{ (x, y) \in K \times K \},
\end{align*}
hvor $K$ et et legeme, med en ekstra linje "i uendelig". 
Vi ønsker at formalisere dette begreb. 
For $x, y, z \in K$ ikke alle nul og $\lambda \in K$, $\lambda \neq 0$, 
definerer vi en ækvivalensrelation. To tripler $(x_1, y_1, z_1)$ og 
$(x_2, y_2, z_2)$ siges at være ækvivalente hvis 
\begin{align*}
	(x_1, y_1, z_1) = (\lambda x_2, \lambda y_2, \lambda z_2),
\end{align*}
og vi skriver $(x_1, y_1, z_1) \sim (x_2, y_2, z_2)$. Vi vil fremover skrive
$(x:y:z)$ for en sådan ækvivalensklasse. I de tilfælde hvor $z \neq 0$ har vi, at
\begin{align*}
	(x, y, z) = (x/z, y/z, 1),
\end{align*}
hvilket er de punkter vi kalder for de "endelige" punkter i $\mathbb{P}^2(K)$.
Vi er nemlig i stand til at associere et punkt fra $\mathbb{A}^2(K)$ med et sådan
punkt. Vi har en afbildning (en inklusion for at være mere præcis) 
$\mathbb{A}^2(K) \to \mathbb{P}^2(K)$ givet ved
\begin{align*}
	(x, y) \mapsto (x, y, 1).
\end{align*}
Dette kan vi selvfølgelig ikke gøre, når $z=0$ og vi ser det som at vi
har $\infty$ i enten $x$- eller $y$-koordinaten. Vi kalder dermed punkterne
$(x, y, 0)$ for punkterne i "uendelig".








\section{Gruppeloven}
Lad nu $E$ være en elliptisk kurve over $K$ som i \ref{elliptic_curve}. Mængden
af punkter på $E$ med koordinater i $K$ er givet ved
\begin{align*}
	E(K) = \{ \infp \} \cup \{ (x, y) \in K \times K \mid y^2 = x^3 + Ax + B \},
\end{align*}
hvor $\infp$ er punktet i uendelighed, som vi vil definere senere. Vi definerer
da en binær operator/funktion $+$ på $E(K)$ ved følgende algoritme:

\begin{definition}[Gruppeloven for elliptiske kurver]
Givet to punkter $P_1, P_2 \in E(K)$, $P_i = (x_i, y_i)$. Et tredje punkt
$R = P_1 + P_2 = (x_3, y_3)$ findes da som følger
\begin{enumerate}
	\item Hvis $x_1 \neq x_2$ da er 
	\begin{align*}
		x_3 = m^2 - x_1 - x_2, \quad y_3 = m(x_1 - x_3) - y_1,
	\end{align*}		
	hvor $m = (y_2 - y_1)/(x_2 - x_1)$.
	\item Hvis $x_1 = x_2$, men $y_1 \neq y_2$ da er $R = P_1 + P_2 = \infp$.
	\item Hvis $P_1 = P_2$ og $y_1 \neq 0$ da er 
	\begin{align*}
		x_3 = m^2 - 2x_1, \quad y_3=m(x_1 - x_3) - y_1,
	\end{align*}
	hvor $m=(3{x_1}^2 + A)/2y_1$.
	\item Hvis $P_1 = P_2$ og $y_1 = 0$ da er $R = P_1 + P_2 = \infp$.
\end{enumerate}
Vi definerer desuden, at 
\begin{align*}
	P + \infp = \infp,
\end{align*}
for alle $P \in E(K)$.
\end{definition}
Vælg to punkter 
\begin{align*}
	P = (x_p, y_p), \quad Q = (x_q, y_q)
\end{align*}
på en elliptisk kurve $E$. Vi kan da trække en ret linje, $L$, igennem
punkterne $P$ og $Q$, som vil skære kurven for $E$ i et tredje 
punkt $P*Q$. Reflektér dette punkt og vi definerer $P+Q$ til at være
dette punkt. Lad desuden $\infp$ betegne punktet i uendelighed. 

Vi skal nu udlede formlerne for additionen af disse punkter. Lad først
$P \neq Q$ og lad $P$ og $Q$ være forskellige fra $\infp$. Da har vi,
at hældningen for linjen igennem $P$ og $Q$ er 
\begin{align*}
	m = \frac{y_q - y_p}{x_q - x_p}.
\end{align*}
Hvis $x_p = x_q$ er linjen lodret, hvilket er et tilfælde vi behandler
senere. Så lad $x_p \neq x_q$, da får vi videre at 
\begin{align*}
	y_q = m(x_q - x_p) + y_p.
\end{align*}
Vi indsætter dette i ligningen for $E$ og får, at 
\begin{align*}
	(m(x - x_p) + y_p)^2 = x^3 + Ax + B.
\end{align*}
Skriver vi dette ud får vi, at
\begin{align*}
	0 &= x^3 + Ax + B - 2y_p m(x-x_p) - m^2 (x - x_p)^2 - y_{p}^{2} \\
	&= x^3 + Ax + B - 2y_p m x - 2y_ p m x_p - m^2 (x^2 -2xx_p + x_{p}^{2}) - y_{p}^{2} \\
	&= x^3 - m^2 x^2 + (A-2my_p +2m^2x_p)x -2m y_p x_p -m^2 x_{p}^{2} - y_{p}^{2} + B. 
\end{align*}
Denne har tre rødder, som netop er de tre punkter, hvor $L$ skærer $E$.
Pr. vores konstruktion kender vi allerede de to rødder $x_p$ og $x_q$,
og vi ønsker at finde den tredje. Generelt for et kubisk polynomium 
$x^3 + ax^2 + bx + c$, med rødder $r, s, t$, har vi at 
\begin{align*}
	x^3 + ax^2 + bx + c = (x - r)(x - s)(x - t) = x^3 - (r + s + t)x^2 + \ldots,
\end{align*}
hvilket giver os, at $-a = r + s + t$. Hvis de to rødder vi kender er $r$ og $s$
kan vi finde den sidste som
\begin{align*}
	t = -a - r - s.
\end{align*}
I vores tilfælde er $a=-m^2$ så vi har, at 
\begin{align*}
	x = m^2 - x_p - x_q.
\end{align*}
Vi mangler da blot at reflektere dette punkt for at have fundet punktet 
punktet $P+Q=(x, y)$. Vi reflekterer over $x$-aksen og finder, at 
\begin{align*}
	x = m^2 - x_p - x_q, \quad y = m(x_p - x) - y_p.
\end{align*}
Vi vender nu tilbage til tilfældet, hvor $x_p = x_q$. Da vil linjen igennem 
$P$ og $Q$ være lodret, så den skærer $E$ i $\infp$. Vi husker, at når $\infp$ 
reflekteres over $x$-aksen får vi igen $\infp$. Vi får altså, at $P+Q=\infp$.

Tilfældet hvor $P=Q=(x, y)$ kræver lidt flere overvejelser, da ikke ligeså
let kan udvælge en linje. For to punkter der ligger tæt på hinanden vil linjen
igennem punkterne nærme sig tangenten til et af punkterne. Derfor vælger vi i
dette tilfælde, at lade linjen der går igennem punkterne være deres tangentlinje.

Blah blah blah.

Hvis $P=\infp$ er linjen igennem $P$ og $Q$ en lodret linje der skærer $E$ i
refleksionen af $Q$. Derfor får vi, at 
\begin{align*}
	\infp + Q = Q.
\end{align*}
Der gælder derfor også, at $\infp + \infp = \infp$.

Vi har nu dækket de mulige tilfælde og kan opstille gruppeloven som følger.




\chapter{Faktoriseringsalgoritmer}

I dette kapitel ønsker vi at se på faktoriseringsalgoritmer. Det viser sig nemlig, at en af de anvendelser som elliptiske kurver besidder, er indenfor faktoriseringen af heltal. Faktoriseringsproblemet, altså hvordan man bestemmer en faktor for et tal $n$ er yderst relevant, da alle heltal kan faktoriseres:

\begin{theorem}[Aritmetikkens fundamentalsætning]
Et heltal $n > 1$ kan faktoriseres entydigt som et produkt af primtal, så hvis
\begin{align*}
	p_1 \cdot p_2 \cdot \ldots \cdot p_k = q_1 \cdot q_2 \cdot \ldots \cdot q_,
\end{align*}
hvor $p_i$ og $q_j$ er primtal for $1 \leq i \leq k$ og $1 \leq j \leq l$ er $k = l$ og $p_i = q_i$ for alle $i=1, 2, \ldots, k$ (efter eventuelle ombytninger). Desuden er faktorerne $p_{1}^{n_1}, p_{2}^{n_2}, \ldots, p_{k}^{n_k}$ entydigt bestemte.
\end{theorem}

For et bevis af sætningen se f.eks. \cite{Hansen}. Det vigtige at bemærke er, at beviset ikke er konstruktivt og dermed ikke giver os en måde, hvorpå vi kan finde disse faktorer. Men hvordan kan vi så finde disse faktorer, som vi nu ved findes? Hvis vi har et sammensat tal $n$, som vi ønsker at faktorisere kunne vi angribe problemet med en naiv tilgang. Vi antager for nemhedens skyld at $n = pq$, hvilket gør det klart at $\min \{p, q \} \leq \sqrt{n}$. Vi kan altså finde en faktor ved at undersøge om først $2 \mid n$, dernæst om $3 \mid n$ osv. indtil at vi finder en faktor, hvilket vil ske senest når vi når til $\sqrt{n}$. Denne løsning er fin for tilstrækkeligt små tal, men det bliver hurtigt uoverkommeligt for store tal (eksempler på hvor lang tid det tager?). 

Sikkerheden i moderne kryptosystemer hviler på dette faktum, at det tager lang tid at faktorisere et heltal. Derfor er det interessant at undersøge om man gøre det hurtigere end den med den naive tilgang. Vi skal se på to af sådanne algoritmer, nemlig Pollards $p-1$ algoritme og Lenstras algoritme, som benytter elliptiske kurver til at finde en faktor.

\section{Pollards $p-1$ algoritme}

Vi starter med at se på Pollards $p-1$ algoritme, da Lenstras algoritme er stærkt inspireret af denne og delvist kan ses som en analog til den, hvilket gør det naturligt at betragte den først. Pollards $p-1$ algoritme blev først præsenteret i \cite{Pollard} i 1970'erne af J. M. Pollard. Algoritmen hviler på Fermats lille sætning:

\begin{theorem}[Fermats lille sætning]
\label{fermats_small_theorem}
Lad $p$ være et primtal som ikke går op i $a$. Da gælder der, at
\begin{align*}
	a^{p-1} = 1 \modu{p}.
\end{align*}
\end{theorem}

\noindent Et bevis findes i appendikset.

Vi kan da se på, hvordan Pollards $p-1$ algoritme virker. Lad $n$ være et sammensat tal og lad $p$ være en primfaktor for $n$. Vi ved fra Fermats lille sætning, at $a^{p-1} \equiv 1 \modu{p}$ når $\gcd(a, p) = 1$. Hvis vi da kendte $p-1$ kunne vi bestemme $p$ (udover den åbenlyse måde) ved
\begin{align*}
	\gcd(a^{p-1} - 1, n) = p. 
\end{align*}
(måske et multiplum af $p$?), da hvis $x \equiv 1 \modu{l}$, hvor $l$ er en faktor i $n$, er $\gcd(x-1, n)$ divisibel med denne faktor $l$.

Vi kender dog ikke $p-1$ og vi kan derfor ikke foretage denne udregning. Det viser sig dog, at vi kan nøjes med et multiplum af $p-1$, da
\begin{align*}
	a^{t(p-1)} - 1 = (a^{p-1})^k - 1 \equiv 1^k - 1 \equiv 0 \modu{p}.
\end{align*}
Idéen er da, at vi vælger et heltal
\begin{align*}
	k = 2^{e_2} \cdot 3^{e_3} \cdot 5^{e_5} \cdots r^{e_r},
\end{align*}
hvor $2, 3, \ldots, r$ er primtal og $e_1, e_2, \ldots, e_r$ er små positive heltal. Vi udregner da $\gcd(a^k - 1, n)$. Hvis vi er i det heldige tilfælde, hvor $n$ har en faktor sådan, at $p-1 \mid k$, da vil $p \mid a^k - 1$ og
vi har så, at
\begin{align*}
	\gcd(a^k - 1, n) \geq p > 1.
\end{align*}
Hvis $\gcd(a^k - 1, n) \neq n$ har vi altså fundet en ikke-triviel faktor for $n$ og vi kan dele $n$ i to faktorer og gentage de ovenstående trin. Hvis vi derimod har, at $\gcd(a^k -1, n) = n$ vælger vi et andet $a$ og forsøger igen, og hvis $\gcd(a^k-1,n)=1$ vælger vi et større $k$.

Dette er tankegangen i Pollards $p-1$ algoritme og vi opsummerer det i algoritmen:

\begin{algorithm}[Pollards $p-1$ algoritme]
Lad $n \geq 2$ være et sammensat tal, som er tallet vi ønsker at 
finde en faktor for.
\begin{enumerate}
	\item Vælg $k \in \mathbb{Z}^+$ sådan, at $k$ er et produkt
	af mange små primtal opløftet i små potenser. Eksempelvis kan $k$ vælges til at være
	\begin{align*}
		k = \lcm [1, 2, \ldots, K],
	\end{align*}
	for et $K \in \mathbb{Z}^+$ og hvor $\lcm$ er det mindste fælles multiplum.
	\item Vælg et heltal $a$ sådan, at $1 < a <n$.
	\item Udregn $\gcd(a, n)$. Hvis $\gcd(a, n) > 1$ har vi fundet en
	ikke-triviel faktor for $n$ og vi er færdige. Ellers fortsæt til næste
	trin.
	\item Udregn $D = \gcd(a^k - 1, n)$. Hvis $1 < D < n$ er $D$ en ikke-triviel
	faktor for $n$ og vi er færdige. Hvis $D = 1$ gå da tilbage til trin 1
	og vælg et større $k$. Hvis $D = n$ gå da til trin 2 og vælg et nyt $a$. 
\end{enumerate}
\end{algorithm}

Følgende er et eksempel på anvendelsen af Pollards algoritme,
hvor det går godt, altså hvor $p-1$ har små primfaktorer:

\begin{example}
Vi vil forsøge at faktorisere 
\begin{align*}
	n = 30042491.
\end{align*}
Vi ser at $2^{n-1} = 2^{30042490} \equiv 25171326 \ (\textrm{mod}\ 30042491)$,
så $N$ er ikke et primtal. Vi vælger som beskrevet i algoritmen
\begin{align*}
	a = 2 \quad \text{og} \quad k = \mathrm{LCM}[1,2, \ldots, 7] = 420.
\end{align*}
Da $420 = 2^2 + 2^5 + 2^7 + 2^8$ skal vi udregne $2^{2^i}$ for 
$0 \leq i \leq 8$. Dette resulterer i følgende tabel:

\begin{center}
\begin{tabular}{c c c c}
$i$ & $2^{2^i} \modu{n}$ & & \\ 
\hline 
1 & 4 & 5 & 28933574 \\ 
2 & 16 & 6 & 27713768 \\ 
3 & 256 & 7 & 10802810 \\ 
4 & 65536 & 8 & 16714289 \\ 
5 & 28933574 & & 
\end{tabular} 
\end{center}

Denne tabel gør det forholdsvist let for os, at bestemme
\begin{align*}
	2^{420} &= 2^{2^2 + 2^5 + 2^7 + 2^8} \\
	&\equiv 16 \cdot 28933574 \cdot 10802810 \cdot 16714289
	\modu{30042491} \\
	&\equiv 27976515 \modu{30042491}.
\end{align*}

Ved anvendelse af den euklidiske algoritme finder vi dernæst, at
\begin{align*}
	\mathrm{gcd}(2^{420} - 1 \modu{n}, n) = \mathrm{gcd}(27976514, 30042491) = 1.
\end{align*}
Her fejler testen altså og vi er nået frem til, at $N$ ikke har nogle 
primtalsfaktorer $p$ sådan, at $p-1$ deler $420$. Algoritmen foreskriver da, at vi skal vælge et nyt $k$. Vi lader
\begin{align*}
	k = \mathrm{LCM}[1,2, \ldots, 11] = 27720.
\end{align*}
Da $27720 = 2^{14} + 2^{13} + 2^{11} + 2^{10} + 2^{6} + 2^3$ skal vi udvide tabellen til at indeholde værdierne for $2^{2^i}$ for $0 \leq i \leq 14$:

\begin{center}
\begin{tabular}{c c c c}
$i$ & $2^{2^i} \modu{n}$ & &  \\ 
\hline 
9 & 19694714 & 12 & 26818902 \\ 
10 & 3779241 & 13 & 8658967 \\ 
11 & 11677316 & 14 & 3783587 \\ 
\end{tabular} 
\end{center}

Vi fortsætter på samme måde, som vi gjorde før og bestemmer
\begin{align*}
	2^{27720} &= 2^{2^3 + 2^{2^6} + 2^{2^{10}} + 2^{2^{11}} + 2^{2^{13}} + 2^{2^{14}}} \\
	&= 256 \cdot 27713768 \cdot 3779241 \cdot 11677316 \cdot 8658967 \cdot 3783587 \\
	&= 16458222 \modu{30042491}.
\end{align*}
Vi finder dernæst, at
\begin{align*}
	\gcd(2^{27720} - 1 \modu{n}, n) = \gcd(16458221, 30042491) = 9241,
\end{align*}
hvilket betyder at vi har fundet en ikke-triviel faktor for $n$. Mere præcist har vi fundet faktoriseringen
\begin{align*}
	30042491 = 3251 \cdot 9241.
\end{align*}

\end{example}
\section{Lenstras elliptiske kurve algoritme}
I \cite{Lenstra}
Lenstra præsenterede H. W. Lenstra en algoritme til faktorisering af heltal, som anvender elliptiske kurver. Vi skal benytte koncepterne vi udviklede til gruppeloven, men vi skal anvende dem på $E(\mathbb{Z}_n)$, hvor $n$ er et sammensat tal. Når $n$ er et sammensat tal er $\mathbb{Z}_n$ ikke er legeme og derfor er $E(\mathbb{Z}_n)$ heller ikke en gruppe. Derfor vælger vi at definere elliptiske pseudokurver for at kunne give mening til dette:

\begin{definition}
Lad $A, B \in \mathbb{Z}_n$, hvor $\gcd(n, 6) = 1$. En elliptisk pseudokurve er da mængden
\begin{align*}
	E(\mathbb{Z}_n) = \{ \infty \} \cup \{ (x, y) \in \mathbb{Z}_n \times \mathbb{Z}_n \mid y^2 =x^3 + Ax +B \},
\end{align*}
hvor $\gcd(4A^3 + 27B^2, n) = 1$.
\end{definition}
Med denne definition følger det, at en elliptisk kurve specielt er en elliptisk pseudokurve, da hvis $p$ er et primtal er $\mathbb{Z}_p = \mathbb{F}_p$ som er et endeligt legeme. Vi foretager addition af punkter på en elliptisk pseudokurve på samme måde, som for elliptiske kurver. På grund af denne definition vil vi for to punkter $P$ og $Q$ på en elliptisk pseudokurve kunne komme ud for at $P+Q$ ikke vil være defineret. Et tilfælde hvor $P+Q$ ikke er defineret vil blive fanget i udregningen af hældningen $m$ i definition \ref{gruppeloven:definitionen}, da $\mathbb{Z}_n$ ikke er et legeme når $n$ er et sammensat, så det der går galt er at $x_2 - x_1$ eller $y_1$ ikke har en invers i $\mathbb{Z}_n$. At en addition kan "gå galt" er motivationen for at kalde disse kurver for pseudokurver.

Vi ser nu på, hvordan vi kan finde en faktor, når additionen af to punkter ikke er defineret. Algoritmen vi her præsenterer er inspireret af algoritmerne i \cite{Silverman} og \cite{Washington}. Lad $n$ være et sammensat tal. Vi husker fra gruppeloven, at vi ved additionen af to punkter skal bruge værdien af inverserne til $x_2 - x_1$ og $y_1$ alt efter, hvilket tilfælde vi er i. Disse inverser eksisterer kun modulo $n$, hvis (se bevis i appendiks)
\begin{align*}
	\gcd(x_1 - x_2, n) = 1 \quad \text{og} \quad \gcd(y_1, n) = 1.
\end{align*}
Men hvis vi er i stand til at finde punkter $P=(x_1, y_1)$ og $Q=(x_2, y_2)$ sådan, at summen $P+Q$ ikke er defineret, da er $\gcd(x, n) > 1$ hvor $x=x_1-x_2$ eller $x=y_1$ og dermed har vi muligvis fundet en ikke-triviel faktor i $n$. Vi kan da se på, hvordan vi kan udnytte dette faktum til at lave en algoritme, som giver os en faktor i $n$. Vælg først tilfældige heltal $x, y, A$ mellem $1$ og $n$, og sæt $B = y^2 - x^3 - Ax \modu{n}$. Vi har da en elliptisk kurve (egentlig ikke en elliptisk kurve, da $n$ er et sammensat tal, men vi lader som om)
\begin{align*}
	E : y^2 = x^3 + Ax + B,
\end{align*}
hvor vi ved at punktet $P=(x, y)$ er placeret. Vi tjekker da, at
\begin{align*}
	d= \gcd(4A^3 + 27B^2, n) = 1.
\end{align*}
Hvis det ikke er tilfældet og $1 < d < n$ har vi fundet en faktor i $n$. Hvis $d = n$ vælger vi et nyt $A$. For et heltal $K$ lader vi $k=\lcm[1, 2, \ldots, K]$ og vi forsøger da, at bestemme
\begin{align*}
	kP = \underbrace{P + P + \ldots + P}_{k \ \text{led}}.
\end{align*}
Det er ikke praktisk at udregne $P+P+ \ldots + P$ så vi gør ligesom i Pollards $p-1$ algoritme og skriver $k$ som den binære udvidelse
\begin{align*}
	k = k_0 + k_1 \cdot 2 + k_2 \cdot 2^2 + k_ 3 \cdot 2^3 + \ldots + k_r \cdot 2^r,
\end{align*}
hvor alle $k_i$ er $0$ eller $1$. Vi kan da udregne
\begin{align*}
	P_0 &= P \\
	P_1 &= 2P_0 = 2P \\
	P_2 &= 2P_1 = 2^2 P \\
	&\vdots \\
	P_r &= 2 P_{r-1} = 2^r P.
\end{align*}
Da kan vi bestemme $kP= (\text{summen af $P_i$'erne hvor $k_i = 1$})$. I hver udregning regner vi modulo $n$, da tallene ellers bliver alt for store og meget langsommelige at arbejde med. Vi håber på, at der i løbet af de udregninger er en addition, som ikke kan lade sig gøre og at dette giver os vores faktor. 


Med algoritmen på plads kan vi nu se på et eksempler:




Vi opsummerer diskussionen i algoritmen nedenfor:

\begin{algorithm}[Lenstras algoritme]
Lad $n \geq 2$ være et sammensat tal.
\begin{enumerate}
	\item Vælg heltal $x, y$ og $A$ mellem $1$ og $n$. Lad da $B = y^2 - x^3 - Ax \modu{n}$, så vi har den 	 
	elliptiske kurve
	\begin{align*}
		E : y^2 = x^3 + Ax + B, 
	\end{align*}
	hvor punktet $P=(x, y)$ er placeret.
	\item Tjek at $d = \gcd(4A^3 + 27B^2, n) = 1$. Hvis $d=n$ går vi tilbage
	til (1) og vælger et nyt $B$. Hvis $1 < d < n$ har vi fundet en faktor af $n$ og vi er færdige.
	\item Vælg et positivt heltal $k$ som et produkt af mange små primtal, lad eksempelvis
	\begin{align*}
		k = \lcm[1, 2, 3, ..., K],
	\end{align*}
	hvor $K \in \mathbb{Z}^+$.
	\item Forsøg at bestemme $kP = P + P + \ldots + P$. Hvis udregningen kan lade sig gøre går vi tilbage til (1) og 
	vælger en ny kurve, eller går til (3) og vælger et større $k$.
\end{enumerate}
\end{algorithm}
For at se hvorfor der er en god chance for, at vi støder på et valg af $x, y, A$ sådan at vi finder en faktor, lader vi $p$ være en primfaktor i $n$. Til den elliptiske kurve $E$ har vi den abelske gruppe $E(\mathbb{F}_p)$ og pr. sætning \ref{hasse} ved vi, at
\begin{align*}
	p + 1 - 2 \sqrt{p} < \#E(\mathbb{F}_p) < p + 1 + 2 \sqrt{p}.
\end{align*} 
En sætning af Deuring \cite{Deuring} siger, at for ethvert heltal $m \in (p+1-2\sqrt{p}, p+1+2\sqrt{p})$ findes der talpar $(A, B)$ i mængden
\begin{align*}
	\{ (A, B) \mid A, B \in \mathbb{F}_p, 4A^3 +27B^2 \neq 0 \},
\end{align*}
sådan at $\#E(\mathbb{F}_p) = m$. For hvert tal i intervallet findes der altså en elliptisk kurve, som har denne orden. Der er altså en positiv sandsynlighed for, at vi kan finde sådan en kurve.

Lenstras sandsynlighedsteoretiske overvejelser.

\begin{example}
Lad nu 
\begin{align*}
	n = 753161713
\end{align*}
være det tal, som vi ønsker at faktorisere. Da $2^{n-1} = 437782651 \modu{n}$ er $n$ ikke et primtal. Vi vælger da
$x = 0$, $y = 1$ og $A=164$. Vi har dermed, at $B = 1^2 - 0^3 - 164 \cdot 0 = 1$ og den elliptiske kurve vi vil arbejde over bliver
\begin{align*}
	E : y^2 = x^3 + 164x + 1,
\end{align*}
hvorpå punktet $P=(0, 1)$ er placeret. Vi ser, at 
\begin{align*}
	D &= \gcd(4 \cdot 164^3 + 27 \modu{753161713}, 753161713) \\ &= \gcd(17643803, 753161713) = 1,
\end{align*}
så vi fortsætter derfor med algoritmen. Vi lader
\begin{align*}
	k = \lcm[1, 2, \ldots, 10] = 2520.
\end{align*}
Da $2520 = 2^{11} + 2^8 + 2^7 + 2^6 + 2^4 + 2^3$ skal vi beregne $2^i P \modu{753161713}$ for $0 \leq i \leq 11$. Dette gøres med additionsformlen og vi opsummerer vores resultater i tabellen nedenfor:

%	Større eksempel, men som ikke er praktisk udskrevet her:
%	n =  1688955439703788849,
\begin{center}
\begin{tabular}{c c c c}
$i$ & $2^i P \modu{753161713}$ & $i$ & $2^i P \modu{753161713}$ \\ 
\hline 
0 & $(0, 1)$ & 6 & $(743238772, 703386057)$  \\ 
1 & $(6724, 752610344)$ & 7 & $(309161840, 219780637)$  \\ 
2 & $(293427237, 450490340)$ & 8 & $(116974611, 722899047)$ \\ 
3 & $(468952095, 385687511)$ & 9 & $(329743899, 182819134)$ \\ 
4 & $(288125200, 446796094)$ & 10 & $(163952469, 456288424)$ \\ 
5 & $(106753239, 115973502)$ & 11 & $(15710788, 301760412)$
\end{tabular} 
\end{center}
Vi kan nu addere disse punkter igen vha. additionsformlerne, hvor vi stadigvæk regner modulo $n$:
\begin{align*}
	(2^3 + 2^4)P &= (606730980, 447512524). 
\end{align*}
Algoritmen giver os en faktor netop når additionen bryder sammen, hvilket kan ske da $\mathbb{Z}/n \mathbb{Z}$ ikke er et legeme. Dette problem viser sig i dette eksempel allerede ved den næste addition, hvor vi forsøger at udregne
\begin{align*}
	(2^3 + 2^4 + 2^6) P = (743238772, 703386057) \\ + (606730980, 447512524) \modu{n}.
\end{align*}
For at denne addition skal kunne lade sig gøre, skal differensen af deres $x$-koordinater have en invers modulo $n$. Dette er kun tilfældet, hvis $\gcd(x_2 - x_1, n) = 1$ (se appendiks, sætning k). Men vi ser, at
\begin{align*}
	\gcd(606730980 - 743238772, 753161713) = 19259,
\end{align*}
så der findes altså ikke en invers, men vi har i stedet fundet en faktor i $n$. Dermed har vi faktoriseringen
\begin{align*}
	753161713 = 19259 \cdot 39107.
\end{align*}
Nu kan det virke til, at det var spild da vi lavede hele tabellen, men i beregningerne af $2^i P \modu{753161713}$ ville vi også kunne have løbet ind i et element, som ikke havde en invers og som dermed kunne give os en faktor.
\end{example}





Vi vil her give et bevis for Hasse's sætning. Til dette
formål skal vi først opbygge nogle resultater vedrørende
endomorfier på endelige legemer.

\begin{definition}[Algebraisk aflukning]
En algebraisk aflukning af et legeme $K$, er et legeme
$K \subseteq \overline{K}$, hvor $\overline{K}$ er en 
algebraisk udvidelse af $K$ samt, at ethvert ikke-konstant
polynomium fra $\overline{K}[X]$ har en rod i $\overline{K}$. 
\end{definition}

Det kan vises, at ethvert legeme har en algebraisk aflukning og
at to algebraiske aflukninger for det samme legeme vil være isomorfe.
Derfor giver det mening for os, at snakke om \emph{den} algebraiske
aflukning for et givent legeme. Lad nu $\mathbb{F}_q$ være et endeligt
legeme med algebraisk aflukning $\overline{\mathbb{F}}_q$. Vi ser på
Frobenius afbildningen 
\begin{align*}
	\phi_q : \overline{\mathbb{F}}_q \to \overline{\mathbb{F}}_q,
\end{align*}
som er givet ved, at $x \mapsto x^q$. For en elliptisk kurve $E$ over
$\mathbb{F}_q$ virker $\phi_q$ på koordinaterne $E(\overline{\mathbb{F}}_q)$ 
ved, at 
\begin{align*}
	\phi_q(x, y) = (x^q, y^q), \quad \phi_q (\infty) = \infty.
\end{align*}

I det følgende udnytter vi, at $q$ er et primtal (men bogen arbejder
måske blot med at $q = p^r$, hvor $p$ er et primtal. Da skal der tilføjes
lidt resultater.

\begin{lemma}
Lad $E$ være en elliptisk kurve over $\mathbb{F}_q$, og lad 
$(x, y) \in E(\overline{\mathbb{F}}_q)$. Da gælder der, at 
\begin{enumerate}
	\item $\phi_q(x, y) \in E(\overline{\mathbb{F}}_q)$,
	\item $(x, y) \in E(\mathbb{F}_q) \Leftrightarrow \phi_q(x, y)=(x, y)$.
\end{enumerate}
\end{lemma}
\begin{proof}
Vi har, at $y^2 = x^3 + ax + b$, hvor $a, b \in \mathbb{F}_q$. Vi opløfter 
denne ligning til den $q$'ende potens og får, at 
\begin{align*}
	(y^q)^2 = (x^q)^3 + (a^q x^q) + b^q,
\end{align*}
hvor vi har brugt Freshman's dream. Men dette betyder netop, at 
$(x^q, y^q) \in E(\overline{\mathbb{F}}_q)$, hvilket viser (1).
For at vise (2) husker vi, at $\phi_q(x) = x \Leftrightarrow x \in \mathbb{F}_q$.
Det følger da, at 
\begin{align*}
	(x, y) \in E(\mathbb{F}_q) &\Leftrightarrow x, y \in \mathbb{F}_q \\
	&\Leftrightarrow \phi_q(x) = x \ \text{og} \ \phi_q(y) = y \\
	&\Leftrightarrow \phi_q(x, y) = (x, y).
\end{align*}
\end{proof}


\begin{lemma}
Lad $E$ være en elliptisk kurve over $\mathbb{F}_q$. Da er $\phi_q$ en 
endomorfi for $E$ af grad $q$, desuden er $\phi_q$ ikke seperabel.
\end{lemma}






\end{document}