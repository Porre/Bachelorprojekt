Vi vil her give et bevis for Hasse's sætning, der giver os en 
(optimal) vurdering af antallet af punkter på en elliptisk kurve.
Til dette
formål skal vi først opbygge nogle resultater vedrørende
endomorfier på endelige legemer, som benyttes i beviset for
dette.

\begin{definition}
En endomorfi på $E$ er en homomorfi $\alpha : E(\clk) \to E(\clk)$ givet
ved rationale funktioner.
\end{definition}

Med en rational funktion forståes en kvotient af polynomier. Det vil altså sige, 
at $\alpha$ skal opfylde at $\alpha(P_1 + P_2)=\alpha(P_1)+\alpha(P_2)$ 
og der skal findes rationale 
funktioner $R_1(x, y)$ og $R_2(x, y)$, begge med koefficienter i $\clk$, så
\begin{align*}
	\alpha(x, y) = (R_1(x, y), R_2(x, y)),
\end{align*}
for alle $(x, y) \in E(\clk)$. Dette giver os også, da $\alpha$ er en homomorfi,
at $\alpha(\infty)=\infty$. Vi vil fremover antage, at $\alpha$ ikke er den 
trivielle endomorfi, altså at der findes $(x, y)$ sådan at $\alpha(x, y) \neq \infty$.

Vi ønsker da, at finde en standard repræsentation for de rationale funktioner, som
beskriver en endomorfi. For en elliptisk kurve $E$ på Weierstrass normalform gælder
der, at $y^2 = x^3 + Ax + B$ for alle $(x, y) \in E(\clk)$, hvilket betyder at
\begin{align*}
	y^{2k} = (x^3 + Ax + B)^k,
\end{align*}
hvor $k \in \mathbb{N}$. På lignende vis har vi også, at 
\begin{align*}
	y^{2k} y = (x^3 + Ax + B)^k y.
\end{align*}
For en rational funktion $R(x, y)$ kan vi nu beskrive en anden rational funktion,
som stemmer overens med denne på punkter fra $E(\clk)$. Vi kan med andre ord antage,
at 
\begin{align}
	\label{rational}
	R(x, y) = \frac{p_1(x) + p_2(x)y}{p_3(x)+p_4(x)y}.
\end{align}
Det er endda muligt, at gøre dette endnu simplere ved at gange udtrykket i \eqref{rational}
med $p_3(x)-p_4(x)y$, da 
\begin{align*}
	(p_3(x) - p_4(x)y)(p_3(x)+p_4(x)y) = p_3(x)^2 - p_4(x)^2 y^2,
\end{align*}
hvorefter vi kan erstatte $y^2$ med $x^3+Ax+B$. Vi får da, at 
\begin{align}
	\label{rational_final}
	R(x, y) = \frac{q_1(x) + q_2(x)y}{q_3(x)}.
\end{align}
Lader vi nu $\alpha$ være en endomorfi givet ved
\begin{align*}
	\alpha(x, y) = (R_1(x, y), R_2(x, y)),
\end{align*}
får vi, da $\alpha$ er en homomorfi, at 
\begin{align*}
	\alpha(x, -y) = \alpha(-(x, y)) - \alpha(x, y).
\end{align*}
Dette medfører, at 
\begin{align*}
	R_1(x, -y) = R_1(x, y) \quad \text{og} \quad R_2(x, -y) = -R_2(x, y).
\end{align*}
Skrives $R_1$ og $R_2$ på samme form som i \eqref{rational_final} følger det da, at 
$q_2(x)=0$ for $R_1$ og $q_1(x)=0$ for $R_2$. Vi kan altså antage, at 
\begin{align*}
	\alpha(x, y) = (r_1(x), r_2(x)y),
\end{align*}
hvor $r_1(x)$ og $r_2(x)$ er rationale funktioner. Skriv da $r_1(x)=p(x)/q(x)$ (opgave om
den rent faktisk er defineret).

\begin{definition}
Graden af en endomorfi $\alpha$ er givet ved
\begin{align*}
	\deg(\alpha) = \max \{ \deg p(x), \deg q(x) \},
\end{align*}
hvis $\alpha$ er ikke-triviel. Hvis $\alpha$ er den trivielle endomorfi lader vi 
$\deg (\alpha) = 0$.
\end{definition}

Vi siger, at $\alpha$ er en separabel endomorfi, hvis den afledede $r_1'(x)$ ikke
er lig nul.


\begin{definition}[Algebraisk aflukning]
En algebraisk aflukning af et legeme $K$, er et legeme
$K \subseteq \overline{K}$, hvor $\overline{K}$ er en 
algebraisk udvidelse af $K$ samt, at ethvert ikke-konstant
polynomium fra $\overline{K}[X]$ har en rod i $\overline{K}$. 
\end{definition}

Det kan vises, at ethvert legeme har en algebraisk aflukning og
at to algebraiske aflukninger for det samme legeme vil være isomorfe.
Derfor giver det mening for os, at snakke om \emph{den} algebraiske
aflukning for et givent legeme. Lad nu $\mathbb{F}_q$ være et endeligt
legeme med algebraisk aflukning $\overline{\mathbb{F}}_q$. Vi ser på
Frobenius afbildningen 
\begin{align*}
	\phi_q : \overline{\mathbb{F}}_q \to \overline{\mathbb{F}}_q,
\end{align*}
som er givet ved, at $x \mapsto x^q$. For en elliptisk kurve $E$ over
$\mathbb{F}_q$ virker $\phi_q$ på koordinaterne $E(\overline{\mathbb{F}}_q)$ 
ved, at 
\begin{align*}
	\phi_q(x, y) = (x^q, y^q), \quad \phi_q (\infty) = \infty.
\end{align*}

I det følgende udnytter vi, at $q$ er et primtal (men bogen arbejder
måske blot med at $q = p^r$, hvor $p$ er et primtal. Da skal der tilføjes
lidt resultater.

\begin{lemma}
Lad $E$ være en elliptisk kurve over $\mathbb{F}_q$, og lad 
$(x, y) \in E(\overline{\mathbb{F}}_q)$. Da gælder der, at 
\begin{enumerate}
	\item $\phi_q(x, y) \in E(\overline{\mathbb{F}}_q)$,
	\item $(x, y) \in E(\mathbb{F}_q) \Leftrightarrow \phi_q(x, y)=(x, y)$.
\end{enumerate}
\end{lemma}
\begin{proof}
Vi har, at $y^2 = x^3 + ax + b$, hvor $a, b \in \mathbb{F}_q$. Vi opløfter 
denne ligning til den $q$'ende potens og får, at 
\begin{align*}
	(y^q)^2 = (x^q)^3 + (a^q x^q) + b^q,
\end{align*}
hvor vi har brugt Freshman's dream. Men dette betyder netop, at 
$(x^q, y^q) \in E(\overline{\mathbb{F}}_q)$, hvilket viser (1).
For at vise (2) husker vi, at $\phi_q(x) = x \Leftrightarrow x \in \mathbb{F}_q$.
Det følger da, at 
\begin{align*}
	(x, y) \in E(\mathbb{F}_q) &\Leftrightarrow x, y \in \mathbb{F}_q \\
	&\Leftrightarrow \phi_q(x) = x \ \text{og} \ \phi_q(y) = y \\
	&\Leftrightarrow \phi_q(x, y) = (x, y),
\end{align*}
hvilket fuldfører beviset for (2).
\end{proof}

\begin{lemma}
Lad $E$ være en elliptisk kurve over $\mathbb{F}_q$. Da er $\phi_q$ en 
endomorfi for $E$ af grad $q$, desuden er $\phi_q$ ikke seperabel.
\end{lemma}
\begin{proof}
Beviset går på at vise, at $\phi_q : E(\clfq) \to E(\clfq)$ er en homomorfi. 
Dette gøres vha. additionsformlerne. Se f.eks. [1].
\end{proof}

Dette betyder, at ${\phi_q}^2 = \phi_q \circ \phi_q$ også er en endomorfi,
hvilket vi kan udvide til ${\phi_q}^n$ for $n \geq 1$ er en endomorfi. 
Da multiplikation samtidigt er en endomorfi følger det, at ${\phi_q}^n -1$ 
også er det. 

Det næste resultat bliver afgørende for beviset for Hasses sætning. 
Det fortæller os om graden af en endomorfi for en elliptisk kurve $E$,
som eksempelvis $\phi_q$ er det.

\begin{proposition}
Lad $E$ være en elliptisk kurve. Lad $\alpha \neq 0$ være en seperabel 
endomorfi for $E$. Da er 
\begin{align*}
	\deg \alpha = |\ker (\alpha)|,
\end{align*}
hvor $\ker (\alpha) = $ angiver kernen for homomorfien 
$\alpha : E(\clk) \to E(\clk)$.
\end{proposition}
\begin{proof}
Vi gemmer beviset til et senere tidspunkt.
\end{proof}
