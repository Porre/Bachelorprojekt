\section{Etablering af gruppestrukturen}
Det er muligt at påføre de elliptiske kurver en gruppestruktur, ved en
geometrisk addition af punkter fra en sådan kurve. Vi skal i dette kapitel indføre 
denne gruppelov og vise, at det resulterer i en abelsk gruppe.	


\subsection{Elliptiske kurver}
Først og fremmest skal vi have defineret, hvad en elliptisk kurve er. Her 
følger definitionen, som vi vil benytte igennem dette projekt.

\begin{definition}[Elliptisk kurve]
En elliptisk kurve er grafen for en ligning på formen
\begin{align}
	\label{elliptic}
	y^2 = x^3 + Ax + B
\end{align}
for $A, B \in K$. Denne form kaldes for Weierstrass normalform.
\end{definition}

\begin{remark}
I tilfældet hvor karakteristikken af $K$ er $2$ eller $3$ kan vi
ikke opnå at få funktionen på Weierstrass normalform.
\end{remark}

Vi kræver desuden for en elliptisk kurve, at 
\begin{align*}
	\Delta = -16(4A^3 + 27B^2) \neq 0,
\end{align*}
hvor $\Delta$ er diskriminanten. Dette er tilsvarende til, at kurvens
rødder har multiplicitet $1$.


\subsection{Det projektive plan}
Som tidligere nævnt vil vi etablere en gruppestruktur på de elliptiske
kurver. For at kunne gøre dette får vi brug for det projektive plan 
$\mathbb{P}^2$. Rent intuitivt kan man se det projektive plan, som
værende den affine plan 
\begin{align*}
	\mathbb{A}^2(K) = \{ (x, y) \in K \times K \},
\end{align*}
hvor $K$ et et legeme, med en ekstra linje "i uendelig". 
Vi ønsker at formalisere dette begreb. 
For $x, y, z \in K$ ikke alle nul og $\lambda \in K$, $\lambda \neq 0$, 
definerer vi en ækvivalensrelation. To tripler $(x_1, y_1, z_1)$ og 
$(x_2, y_2, z_2)$ siges at være ækvivalente hvis 
\begin{align*}
	(x_1, y_1, z_1) = (\lambda x_2, \lambda y_2, \lambda z_2),
\end{align*}
og vi skriver $(x_1, y_1, z_1) \sim (x_2, y_2, z_2)$. Vi vil fremover skrive
$(x:y:z)$ for en sådan ækvivalensklasse. I de tilfælde hvor $z \neq 0$ har vi, at
\begin{align*}
	(x, y, z) = (x/z, y/z, 1),
\end{align*}
hvilket er de punkter vi kalder for de "endelige" punkter i $\mathbb{P}^2(K)$.
Vi er nemlig i stand til at associere et punkt fra $\mathbb{A}^2(K)$ med et sådan
punkt. Vi har en afbildning (en inklusion for at være mere præcis) 
$\mathbb{A}^2(K) \to \mathbb{P}^2(K)$ givet ved
\begin{align*}
	(x, y) \mapsto (x, y, 1).
\end{align*}
Dette kan vi selvfølgelig ikke gøre, når $z=0$ og vi ser det som at vi
har $\infty$ i enten $x$- eller $y$-koordinaten. Vi kalder dermed punkterne
$(x, y, 0)$ for punkterne i "uendelig".








\subsection{Gruppeloven}
Lad nu $E$ være en elliptisk kurve over $K$ som i \ref{elliptic}. Mængden
af punkter på $E$ med koordinater i $K$ er givet ved
\begin{align*}
	E(K) = \{ \infp \} \cup \{ (x, y) \in K \times K \mid y^2 = x^3 + Ax + B \},
\end{align*}
hvor $\infp$ er punktet i uendelighed, som vi vil definere senere. Vi definerer
da en binær operator/funktion $+$ på $E(K)$ ved følgende algoritme:

\begin{definition}[Gruppeloven for elliptiske kurver]
Givet to punkter $P_1, P_2 \in E(K)$, $P_i = (x_i, y_i)$. Et tredje punkt
$R = P_1 + P_2 = (x_3, y_3)$ findes da som følger
\begin{enumerate}
	\item Hvis $x_1 \neq x_2$ da er 
	\begin{align*}
		x_3 = m^2 - x_1 - x_2, \quad y_3 = m(x_1 - x_3) - y_1,
	\end{align*}		
	hvor $m = (y_2 - y_1)/(x_2 - x_1)$.
	\item Hvis $x_1 = x_2$, men $y_1 \neq y_2$ da er $R = P_1 + P_2 = \infp$.
	\item Hvis $P_1 = P_2$ og $y_1 \neq 0$ da er 
	\begin{align*}
		x_3 = m^2 - 2x_1, \quad y_3=m(x_1 - x_3) - y_1,
	\end{align*}
	hvor $m=(3{x_1}^2 + A)/2y_1$.
	\item Hvis $P_1 = P_2$ og $y_1 = 0$ da er $R = P_1 + P_2 = \infp$.
\end{enumerate}
Vi definerer desuden, at 
\begin{align*}
	P + \infp = \infp,
\end{align*}
for alle $P \in E(K)$.
\end{definition}
Vælg to punkter 
\begin{align*}
	P = (x_p, y_p), \quad Q = (x_q, y_q)
\end{align*}
på en elliptisk kurve $E$. Vi kan da trække en ret linje, $L$, igennem
punkterne $P$ og $Q$, som vil skære kurven for $E$ i et tredje 
punkt $P*Q$. Reflektér dette punkt og vi definerer $P+Q$ til at være
dette punkt. Lad desuden $\infp$ betegne punktet i uendelighed. 

Vi skal nu udlede formlerne for additionen af disse punkter. Lad først
$P \neq Q$ og lad $P$ og $Q$ være forskellige fra $\infp$. Da har vi,
at hældningen for linjen igennem $P$ og $Q$ er 
\begin{align*}
	m = \frac{y_q - y_p}{x_q - x_p}.
\end{align*}
Hvis $x_p = x_q$ er linjen lodret, hvilket er et tilfælde vi behandler
senere. Så lad $x_p \neq x_q$, da får vi videre at 
\begin{align*}
	y_q = m(x_q - x_p) + y_p.
\end{align*}
Vi indsætter dette i ligningen for $E$ og får, at 
\begin{align*}
	(m(x - x_p) + y_p)^2 = x^3 + Ax + B.
\end{align*}
Skriver vi dette ud får vi, at
\begin{align*}
	0 &= x^3 + Ax + B - 2y_p m(x-x_p) - m^2 (x - x_p)^2 - y_{p}^{2} \\
	&= x^3 + Ax + B - 2y_p m x - 2y_ p m x_p - m^2 (x^2 -2xx_p + x_{p}^{2}) - y_{p}^{2} \\
	&= x^3 - m^2 x^2 + (A-2my_p +2m^2x_p)x -2m y_p x_p -m^2 x_{p}^{2} - y_{p}^{2} + B. 
\end{align*}
Denne har tre rødder, som netop er de tre punkter, hvor $L$ skærer $E$.
Pr. vores konstruktion kender vi allerede de to rødder $x_p$ og $x_q$,
og vi ønsker at finde den tredje. Generelt for et kubisk polynomium 
$x^3 + ax^2 + bx + c$, med rødder $r, s, t$, har vi at 
\begin{align*}
	x^3 + ax^2 + bx + c = (x - r)(x - s)(x - t) = x^3 - (r + s + t)x^2 + \ldots,
\end{align*}
hvilket giver os, at $-a = r + s + t$. Hvis de to rødder vi kender er $r$ og $s$
kan vi finde den sidste som
\begin{align*}
	t = -a - r - s.
\end{align*}
I vores tilfælde er $a=-m^2$ så vi har, at 
\begin{align*}
	x = m^2 - x_p - x_q.
\end{align*}
Vi mangler da blot at reflektere dette punkt for at have fundet punktet 
punktet $P+Q=(x, y)$. Vi reflekterer over $x$-aksen og finder, at 
\begin{align*}
	x = m^2 - x_p - x_q, \quad y = m(x_p - x) - y_p.
\end{align*}
Vi vender nu tilbage til tilfældet, hvor $x_p = x_q$. Da vil linjen igennem 
$P$ og $Q$ være lodret, så den skærer $E$ i $\infp$. Vi husker, at når $\infp$ 
reflekteres over $x$-aksen får vi igen $\infp$. Vi får altså, at $P+Q=\infp$.

Tilfældet hvor $P=Q=(x, y)$ kræver lidt flere overvejelser, da ikke ligeså
let kan udvælge en linje. For to punkter der ligger tæt på hinanden vil linjen
igennem punkterne nærme sig tangenten til et af punkterne. Derfor vælger vi i
dette tilfælde, at lade linjen der går igennem punkterne være deres tangentlinje.

Blah blah blah.

Hvis $P=\infp$ er linjen igennem $P$ og $Q$ en lodret linje der skærer $E$ i
refleksionen af $Q$. Derfor får vi, at 
\begin{align*}
	\infp + Q = Q.
\end{align*}
Der gælder derfor også, at $\infp + \infp = \infp$.

Vi har nu dækket de mulige tilfælde og kan opstille gruppeloven som følger.



