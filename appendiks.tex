\appendix
\chapter{Legemer}
\label{appendiks_legemer}
Lad $p$ være et primtal. Heltallene modulo $p$ giver os et legeme $\mathbb{F}_p = \mathbb{Z}/p\mathbb{Z}$ der indeholder $p$ elementer. Antallet af elementer i ethvert endeligt legeme er på formen $p^n$. For at se dette lader vi $K$ være et endeligt legeme. Dets karakteristik må være $p$ for et eller andet primtal, da et legeme med karakteristik $0$ er uendeligt. Derfor må $K$ være endeligt frembragt som et vektorrum over $\mathbb{Z}/p\mathbb{Z}$. Så lad $x_1, \ldots, x_n$ være en basis for $K$. Elementerne i $K$ kan da skrives entydigt som
\begin{align*}
	x= \alpha_1 x_1 + \alpha_2 x_2 + \ldots + \alpha_n x_n, \quad \text{hvor} \quad \alpha_i \in \mathbb{Z}/p\mathbb{Z}.
\end{align*}
Da hvert $\alpha_i \in K$ har vi altså $p$ forskellige valg for hver af $\alpha_1, \alpha_2, \ldots, \alpha_n$. Dette betyder, at vi har $p^n$ valg for $x$, som også giver os hele legemet $K$. Antallet af elementer i $K$ er altså $p^n$. 

Vi benytter notationen $\mathbb{F}_q = \mathbb{F}_{p^n}$ for det endelige legeme med $q=p^n$ elementer. Bemærk, at $\mathbb{Z}/p^n \mathbb{Z}$ ikke er et legeme for $n \geq 2$, da $p$ ikke har nogen multiplikativ invers.



\chapter{Talteoretiske resultater}
Her samler vi nogle af de (hovedsagligt) mindre resultater, som benyttes igennem kapitlerne. De præsenteres her kort og henvises til  i opgaven, når de er blevet anvendt.

\begin{proposition}
\label{inverse_exists_not}
Et element $a \in \mathbb{Z}_n = \mathbb{Z}/n\mathbb{Z}$ har ikke en invers i $\mathbb{Z}_n$ hvis $\gcd(a, n) > 1$. 
\end{proposition}
\begin{proof}
Antag for modstrid, at $d = \gcd(a, n) > 1$, men at der samtidigt eksisterer en invers $c$ til $a$ modulo $n$. Da $d = \gcd(a, n)$ findes et heltal $e$, som ikke er nul, sådan at $de = n$. Da $d >1$ har vi også, at $|e| < |n|$ så $e$ er ikke nul modulo $n$. Da $d$ deler $a$ har vi, at $n = de$ deler $ae$ så $ae = 0 \modu{n}$. Vi har altså, at
\begin{align*}
	e = e \cdot 1 = eac = 0 \cdot c = 0 \modu{n},
\end{align*}
hvilket er i modstrid med at $e$ ikke kunne være $0$ modulo $n$. Altså har $a$ ikke en invers når $\gcd(a, n) > 1$.
\end{proof}

Vi giver nu beviset for sætning 
\begin{proof}[Bevis for Fermats lille sætning]
Vi ser først på de $p-1$ positive multipla af $a$
\begin{align}
	\label{multiples}
	a, 2a, \ldots, (p-1)a.
\end{align}
Hvis $ra = sa \modu{p}$ har vi, at $r = s \modu{p}$, så elementerne listet i \eqref{multiples} er forskellige og ikke-nul. De må altså være kongruente til $1, 2, \ldots, p-1$ men ikke nødvendigvis i den opskrevne rækkefølge. Ganger vi elementerne sammen må de to kongruenser være de samme, altså er
\begin{align*}
	a \cdot 2a \cdots (p-1)a = 1 \cdot 2 \cdots (p-1) \modu{p},
\end{align*}
hvilket giver os, at 
\begin{align*}
	a^{p-1} (p-1)! = (p-1)! \modu{p} \Rightarrow a^{p-1} = 1 \modu{p}.
\end{align*}


\end{proof}