\appendix

\chapter{Udeladte resultater}
Her samler vi nogle af de (hovedsagligt) mindre resultater, som benyttes igennem kapitlerne. De præsenteres her kort og henvises til de steder de anvendes i hovedopgaven.

\section{Legemer}
\label{appendiks_legemer}
Lad $p$ være et primtal. Heltallene modulo $p$ giver os et legeme $\mathbb{F}_p = \mathbb{Z}/p\mathbb{Z}$ der indeholder $p$ elementer. Antallet af elementer i ethvert endeligt legeme er på formen $p^n$. For at se dette lader vi $K$ være et endeligt legeme. Dets karakteristik må være $p$ for et eller andet primtal, da et legeme med karakteristik $0$ er uendeligt. Derfor må $K$ være endeligt frembragt som et vektorrum over $\mathbb{Z}/p\mathbb{Z}$. Så lad $x_1, \ldots, x_n$ være en basis for $K$. Elementerne i $K$ kan da skrives entydigt som
\begin{align*}
	x= \alpha_1 x_1 + \alpha_2 x_2 + \ldots + \alpha_n x_n, \quad \text{hvor} \quad \alpha_i \in \mathbb{Z}/p\mathbb{Z}.
\end{align*}
Da hvert $\alpha_i \in K$ har vi altså $p$ forskellige valg for hver af $\alpha_1, \alpha_2, \ldots, \alpha_n$. Dette betyder, at vi har $p^n$ valg for $x$, som også giver os hele legemet $K$. Antallet af elementer i $K$ er altså $p^n$. 

Vi benytter notationen $\mathbb{F}_q = \mathbb{F}_{p^n}$ for det endelige legeme med $q=p^n$ elementer. Bemærk, at $\mathbb{Z}/p^n \mathbb{Z}$ ikke er et legeme for $n \geq 2$, da $p$ ikke har nogen multiplikativ invers.

\section{Andre resultater}

\begin{thm}
\label{monic}
Lad $K$ være et legeme. Hvis $f, g \in K[X]$, da findes der polynomier $q, r \in K[X]$ sådan at $\deg r < \deg g$ og
\begin{align*}
	f = qg + r.
\end{align*}
Hvis $f, g \in \mathbb{Z}[X]$ og $g$ er monisk, da findes der $q, r \in \mathbb{Z}$ sådan at $\deg r < \deg g$ og
\begin{align*}
	f = qg + r.
\end{align*}
\end{thm}
\begin{proof}
Det følger af divisionsalgoritmen for polynomier. Hvis $f$ har det ledende term $ax^n$ og $g$ har det ledende term $bx^m$ hvor $n \geq m$, da har $f - \frac{a}{b} x^{n-m}$ grad mindre end $f$. Vi kan dermed blive ved med at trække multipla af $g$ fra $f$ indtil resultatet har grad mindre end $\deg g$.

Hvis $g$ er monisk er $b=1$ so hver gang trækker vi et polynomium med heltalskoefficienter fra så både kvotienten $q$ og resten $r$ vil dermed have heltalskoefficienter.
\end{proof}
\label{appendiks_andre}
Følgende propositon benyttes i Hasses sætning: 
\begin{proposition}
\label{appendiks_dense}
Mængden 
\begin{align*}
	\left\{ \frac{r}{s} \mid \gcd(s, q) = 1 \right\} \subseteq \mathbb{Q}
\end{align*}
er tæt i $\mathbb{R}$ og 
\begin{align*}
	qx^2 - ax +1 \geq 0
\end{align*}
for alle $x \in \mathbb{R}$.
\end{proposition}
\begin{proof}
En delmængde $X \subseteq \mathbb{R}$ siges at være tæt, hvis der for alle heltal $a \in \mathbb{R}$ findes et interval med centrum i $a$, som også indeholder punkter fra $X$. 

Lad $X$ være mængden i propositionen og lad $s$ være en potens af $2$ eller $3$. Ét af de to må nødvendigvis være primisk med $q$, da $q$ er en potens af ét enkelt primtal $p$. Vi ser, at rationalerne på formen $r/2^n$ er tæt i $\mathbb{R}$ på følgende måde. Lad $a < b \in \mathbb{R}$. Der findes da $n \in \mathbb{N}$ sådan, at 
\begin{align*}
	0 < \frac{1}{n} < b - a \Rightarrow 0 < \frac{1}{2^n} < \frac{1}{n} < b - a.
\end{align*}
Vi har altså at $1 < 2^n \cdot b - 2^n \cdot a$. Når afstanden mellem $2^n \cdot b$ og $2^n \cdot a$ er større end $1$ findes der $r \in \mathbb{Z}$ så $2^n \cdot a < r < 2^n \cdot b \Rightarrow a < \frac{r}{2^n} < b$ 
($2^n \neq 0$). Så $r/2^n$ er tæt i $\mathbb{R}$. Det samme kan gøres for $r/3^n$.

Vi har da, at $X$ indeholder en delmængde som er tæt i $\mathbb{R}$, hvilket giver os at $X$ selv er tæt i $\mathbb{R}$.

En konsekvens af tætheden af $X$ er, at vi kan konkludere at $qx^2 - ax +1 \geq 0$ for alle 
$x \in \mathbb{R}$. Antag for modstrid, at der findes $r \in \mathbb{R}$ sådan at $ar^2 -ar + 1 < 0$. Vi ser på en følge af intervaller omkring $r$:
\begin{align*}
	(r - \varepsilon, r + \varepsilon), \quad \text{hvor} \ \varepsilon = \frac{1}{n}, \quad n = 1, 2, \ldots
\end{align*}
I hvert af disse intervaller findes $x_n \in X$, hvor $X$ er som i proposition \ref{appendiks_dense}, da vi netop har vist at $X$ er en tæt mængde i $\mathbb{R}$. Vi får en følge $x_1, x_2, \ldots$ af tal som nærmer sig $r$. For et stort nok $i$ har vi, at $qx_{i}^{2} -ax_i + 1$ er arbitrært tæt på $qr^2 - ar + 1$. Men da $x_i \in X$ må den første af disse to være $\geq 0$ mens den anden er $< 0$. Dette er en modstrid og vi har, at 
\begin{align*}
	qx^2 - ax +1 \geq 0,
\end{align*}
for alle $x \in \mathbb{R}$.
\end{proof}


\begin{proposition}
\label{inverse_exists_not}
Et element $a \in \mathbb{Z}_n = \mathbb{Z}/n\mathbb{Z}$ har ikke en invers i $\mathbb{Z}_n$ hvis $\gcd(a, n) > 1$. 
\end{proposition}
\begin{proof}
Antag for modstrid, at $d = \gcd(a, n) > 1$, men at der samtidigt eksisterer en invers $c$ til $a$ modulo $n$. Da $d = \gcd(a, n)$ findes et heltal $e$, som ikke er nul, sådan at $de = n$. Da $d >1$ har vi også, at $|e| < |n|$ så $e$ er ikke nul modulo $n$. Da $d$ deler $a$ har vi, at $n = de$ deler $ae$ så $ae = 0 \modu{n}$. Vi har altså, at
\begin{align*}
	e = e \cdot 1 = eac = 0 \cdot c = 0 \modu{n},
\end{align*}
hvilket er i modstrid med at $e$ ikke kunne være $0$ modulo $n$. Altså har $a$ ikke en invers når $\gcd(a, n) > 1$.
\end{proof}

Vi giver nu beviset for sætning 
\begin{proof}[Bevis for Fermats lille sætning]
Vi ser først på de $p-1$ positive multipla af $a$
\begin{align}
	\label{multiples}
	a, 2a, \ldots, (p-1)a.
\end{align}
Hvis $ra = sa \modu{p}$ har vi, at $r = s \modu{p}$, så elementerne listet i \eqref{multiples} er forskellige og ikke-nul. De må altså være kongruente til $1, 2, \ldots, p-1$ men ikke nødvendigvis i den opskrevne rækkefølge. Ganger vi elementerne sammen må de to kongruenser være de samme, altså er
\begin{align*}
	a \cdot 2a \cdots (p-1)a = 1 \cdot 2 \cdots (p-1) \modu{p},
\end{align*}
hvilket giver os, at 
\begin{align*}
	a^{p-1} (p-1)! = (p-1)! \modu{p} \Rightarrow a^{p-1} = 1 \modu{p}.
\end{align*}


\end{proof}