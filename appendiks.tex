\appendix

\chapter{Udeladte resultater}
Her samler vi nogle af de resterende resultater, som benyttes igennem kapitlerne. I opgaven henvises der til resultaterne her, når de anvendes i beviserne. 

\section{Legemer}
\label{appendiks_legemer}
Lad $p$ være et primtal. Heltallene modulo $p$ giver os et legeme $\mathbb{F}_p = \mathbb{Z}/p\mathbb{Z}$ der indeholder $p$ elementer. Antallet af elementer i ethvert endeligt legeme er på formen $p^n$. For at se dette lader vi $K$ være et endeligt legeme. Dets karakteristik må være $p$ for et eller andet primtal, da et legeme med karakteristik $0$ er uendeligt. Derfor må $K$ være endeligt frembragt som et vektorrum over $\mathbb{Z}/p\mathbb{Z}$. Så lad $x_1, \ldots, x_n$ være en basis for $K$. Elementerne i $K$ kan da skrives entydigt som
\begin{align*}
	x= \alpha_1 x_1 + \alpha_2 x_2 + \ldots + \alpha_n x_n, \quad \text{hvor} \quad \alpha_i \in \mathbb{Z}/p\mathbb{Z}.
\end{align*}
Da hvert $\alpha_i \in K$ har vi altså $p$ forskellige valg for hver af $\alpha_1, \alpha_2, \ldots, \alpha_n$. Dette betyder, at vi har $p^n$ valg for $x$, som også giver os hele legemet $K$. Antallet af elementer i $K$ er altså $p^n$. 

Vi benytter notationen $\mathbb{F}_q = \mathbb{F}_{p^n}$ for det endelige legeme med $q=p^n$ elementer. Bemærk, at $\mathbb{Z}/p^n \mathbb{Z}$ ikke er et legeme for $n \geq 2$, da $p$ ikke har nogen multiplikativ invers.

\begin{thm}
\label{appendiks:f_qtheorem}
Lad $\overline{\mathbb{F}}_p$ være den algebraiske aflukning af $\mathbb{F}_p$ og lad $q = p^n$, da er
\begin{align}
	\label{appendiks:f_q}
	\mathbb{F}_q = \{ \alpha \in \overline{\mathbb{F}}_p \mid \alpha^q = \alpha 
	\}.
\end{align}
\end{thm}
\begin{proof}
Vi vil først vise, at $\mathbb{F}_q \subseteq \{ \alpha \in \overline{\mathbb{F}}_p \mid \alpha^q = \alpha \}$. Ikke-nul elementerne fra $\mathbb{F}_q$ danner en gruppe af orden $q-1$ så $\alpha^{q-1} = 1$ når 
$0 \neq \alpha \in \mathbb{F}_q$. Da $0^q = 0$ har vi, at $\alpha^q = \alpha$ for alle $\alpha \in \mathbb{F}_q$ og vi har den ønskede inklusion.

Idet vi husker, at et polynomium $g(X)$ kun har multiple rødder, hvis den har en fælles rod med dens afledede $g'(X)$ ser vi, at 
\begin{align*}
	\frac{d}{dX} (X^q - X) = qX^{q-1} - 1 = -1,
\end{align*}
da $q = p^n = 0$ i $\mathbb{F}_p$. Dermed har $g(X)$ altså ikke nogen multipel rod, så der findes $q$ forskellige elementer $\alpha \in \overline{\mathbb{F}}_p$ sådan at $\alpha^q = \alpha$.

Da begge mængder i \eqref{appendiks:f_q} har samme antal elementer og den ene er indeholdt i den anden, må de nødvendigvis være ens.
\end{proof}
Lad $\phi_q(x) = x^q$ for alle $x \in \overline{\mathbb{F}}_q$. Vi kalder da $\phi_q$ for Frobenius automorfien (analogt til Frobenius endomorfien). Følgende proposition fastsætter nogle af dens egenskaber:

\begin{proposition}
Lad $q=p^n$ hvor $p$ er et primtal.
\begin{enumerate}
	\item Lad $\alpha \in \overline{\mathbb{F}}_q$. Da er $\alpha \in \mathbb{F}_{q^n}$ 
	hvis og kun hvis $\phi_{q}^{n}(\alpha) = \alpha$.
	\item $\phi_q$ er en automorfi for $\overline{\mathbb{F}}_q$. Specielt gælder der,
	at 
	\begin{align*}
		\phi_q (x+y) = \phi_q(x) + \phi_q(y) \quad \text{og} \quad \phi_q(xy) =
		 \phi_q(x)\phi_q(y).
	\end{align*}
\end{enumerate}
\end{proposition}
En automorfi er en bijektiv homomorfi (isomorfi), som afbilleder til sig selv.
\begin{proof}
(1) følger fra sætning \ref{appendiks:f_qtheorem} med $q^n$ i stedet for $q$. Vi kan derfor fokusere på (2). Lad $1 \leq j \leq p-1$, da vi vil binomial koefficienten 
$\binom{p}{j}$ have en faktor $p$ i dens tæller som ikke går ud med nævneren. Derfor har vi, at 
\begin{align*}
	\binom{p}{j} \equiv 0 \modu{p}.
\end{align*}
Det følger heraf, at 
\begin{align*}
	(x + y)^p = x^p + \binom{p}{1}x^{p-1}y + \binom{p}{2}x^{p-2}y^2 + 
	\ldots + y^p = x^p + y^p,
\end{align*}
da vi er i karakteristik $p$. Vi vil lave et induktionsbevis og dette viser basistilfældet. Antag nu, at der gælder at $(x+y)^{p^n} = x^{p^n} + y^{p^n}$. Da har vi, at
\begin{align*}
	(x+y)^{p^{n+1}} &= (x+y)^{p^n p} = ((x+y)^{p^n})^p
	= (x^{p^n} + y^{p^n})^p \\
	&= x^{p^{n+1}} + y^{p^{n+1}}.
\end{align*}
Dette giver os, at $\phi_q(x+y) = \phi_q(x) + \phi_q(y)$. Det er klart, at 
$\phi_q(xy)=\phi_q(x) \phi_q(y)$, da $(xy)^q = x^q y^q$. Vi har da, at $\phi_q$ er en homomorfi for legemer. Vi mangler at vise, at $\phi_q$ er en bijektion. En homomorfi for legemer er injektiv (kernen af en homomorfi er et ideal, så over et legeme er kernen tom eller hele legemet). Hvis $\alpha \in \overline{\mathbb{F}}_p$ da er 
$\alpha \in \mathbb{F}_{q^n}$ for et eller andet $n$, så $\phi_{q}^{n}(\alpha) = \alpha$. Dermed er $\alpha$ i billedet af $\phi_q$ så $\phi_q$ er surjektiv. Dermed er $\phi_q$ en automorfi.
\end{proof}

\section{Tæthedsargumentet i Hasses sætning}
Følgende propositon benyttes i Hasses sætning: 
\begin{proposition}
\label{appendiks_dense}
Mængden 
\begin{align*}
	\left\{ \frac{r}{s} \mid \gcd(s, q) = 1 \right\} \subseteq \mathbb{Q}
\end{align*}
er tæt i $\mathbb{R}$ og $qx^2 - ax + 1 \geq 0$ for alle $x \in \mathbb{R}$.
\end{proposition}
\begin{proof}
En delmængde $X \subseteq \mathbb{R}$ siges at være tæt, hvis der for alle heltal $a \in \mathbb{R}$ findes et interval med centrum i $a$, som også indeholder punkter fra $X$. 

Lad $X$ være mængden i propositionen og lad $s$ være en potens af $2$ eller $3$. Ét af de to må nødvendigvis være primisk med $q$, da $q$ er en potens af ét enkelt primtal $p$. Vi ser, at rationalerne på formen $r/2^n$ er tæt i $\mathbb{R}$ på følgende måde. Lad $a, b \in \mathbb{R}$ være sådan at $a < b$. Der findes da $n \in \mathbb{N}$ sådan, at 
\begin{align*}
	0 < \frac{1}{n} < b - a \Rightarrow 0 < \frac{1}{2^n} < \frac{1}{n} < b - a.
\end{align*}
Vi har altså at $1 < 2^n b - 2^n a$. Når afstanden mellem $2^n b$ og $2^n a$ er større end $1$ findes der $r \in \mathbb{Z}$ sådan, at
\begin{align*}
	2^n a < r < 2^n b \Rightarrow a < \frac{r}{2^n} < b.
\end{align*}
Dette viser netop, at $r/2^n$ er tæt i $\mathbb{R}$. Det samme kan gøres for $r/3^n$. Vi har da, at $X$ indeholder en delmængde som er tæt i $\mathbb{R}$, hvilket giver os at $X$ selv er tæt i $\mathbb{R}$.

Vi vi vil nu vise, at $qx^2 - ax + 1 \geq 0$ for alle $x \in \mathbb{R}$.
Antag for modstrid, at der findes $r \in \mathbb{R}$ sådan at $ar^2 -ar + 1 < 0$. Vi ser på en følge af intervaller omkring $r$:
\begin{align*}
	(r - \varepsilon, r + \varepsilon), \quad \text{hvor} \ \varepsilon = \frac{1}{n}, \quad n = 1, 2, \ldots
\end{align*}
I hvert af disse intervaller findes $x_n \in X$, da vi netop har vist at $X$ er en tæt mængde i $\mathbb{R}$. Vi får en følge $x_1, x_2, \ldots$ af tal som nærmer sig $r$. For et stort nok $i$ har vi, at $qx_{i}^{2} -ax_i + 1$ er arbitrært tæt på $qr^2 - ar + 1$. Men da $x_i \in X$ må den første af disse to være $\geq 0$ mens den anden er $< 0$. Dette er en modstrid og vi har, at 
$qx^2 - ax + 1 \geq 0$ for alle $x \in \mathbb{R}$.
\end{proof}

\section{Andre resultater}
\begin{thm}
\label{monic}
Lad $K$ være et legeme. Hvis $f, g \in K[X]$, da findes der polynomier $q, r \in K[X]$ sådan at $\deg r < \deg g$ og
\begin{align*}
	f = qg + r.
\end{align*}
Hvis $f, g \in \mathbb{Z}[X]$ og $g$ er monisk, da findes der $q, r \in \mathbb{Z}$ sådan at $\deg r < \deg g$ og
\begin{align*}
	f = qg + r.
\end{align*}
\end{thm}
\begin{proof}
Det følger direkte af divisionsalgoritmen for polynomier. Hvis $f$ har ledende term $ax^n$ og $g$ har ledende term $bx^m$ hvor $n \geq m$, da har $f - \frac{a}{b} x^{n-m}$ grad mindre end $f$. Vi kan dermed blive ved med at trække multipla af $g$ fra $f$ indtil resultatet har grad mindre end $\deg g$.
Hvis $g$ er monisk er $b=1$ så hver gang trækker vi et polynomium med heltalskoefficienter fra, så både kvotienten $q$ og resten $r$ vil dermed have heltalskoefficienter.
\end{proof}
\label{appendiks_andre}
\begin{proposition}
\label{inverse_exists_not}
Et element $a \in \mathbb{Z}_n = \mathbb{Z}/n\mathbb{Z}$ har ikke en invers i $\mathbb{Z}_n$ hvis 
\begin{align*}
	\gcd(a, n) > 1.
\end{align*}
\end{proposition}
\begin{proof}
Antag for modstrid, at $d = \gcd(a, n) > 1$, men at der samtidigt eksisterer en invers $c$ til $a$ modulo $n$. Da $d = \gcd(a, n)$ findes et heltal $e$, som ikke er nul, sådan at $de = n$. Da $d >1$ har vi også, at $|e| < |n|$ så $e$ er ikke nul modulo $n$. Da $d$ deler $a$ har vi, at $n = de$ deler $ae$ så $ae \equiv 0 \modu{n}$. Vi har altså, at
\begin{align*}
	e = e \cdot 1 = eac \equiv 0 \cdot c = 0 \modu{n},
\end{align*}
hvilket er i modstrid med at $e$ ikke kunne være $0$ modulo $n$. Altså har $a$ ikke en invers når $\gcd(a, n) > 1$.
\end{proof}