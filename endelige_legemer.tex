\chapter{Elliptiske kurver over endelige legemer}

Vi skal i dette kapitel undersøge elliptiske kurver over endelige legemer. 
Lad $\mathbb{F}$ være et endeligt legeme og lad $E$ være en elliptisk kurve på
formen 
\begin{align*}
	y^2 = x^3 + Ax + B,
\end{align*}
som er defineret over $\mathbb{F}$. Da er gruppen $E(\mathbb{F})$ endelig, da 
der kun findes endeligt mange talpar $(x, y)$ så $x, y \in \mathbb{F}$. Lad 
$E$ være den elliptiske kurve $y^2 = x^3 - x$ over $\mathbb{F}_5$. For at bestemme
ordenen af $E(\mathbb{F})$ laver vi en tabel over mulige værdier for $x$, $x^3 - x \modu{5}$
og for $y$ som er kvadratrødderne af $x^3 - x$. Dette giver os samtlige punkter på kurven:


\begin{center}
\begin{tabular}{c c c c }
$x$ & $x^3 - x$ & $y$ & Punkter \\ 
\hline
$0$ & $0$ & $0$ & $(0, 0)$ \\ 
$1$ & $0$ & $0$ & $(1, 0)$ \\ 
$2$ & $1$ & $\pm 1$ & $(2, 1), (2, 4)$ \\ 
$3$ & $4$ & $\pm 2$ & $(3, 2), (3, 3)$ \\ 
$4$ & $2$ & $-$ & $-$ \\ 
$\infty$ & & $\infty$ & $\infty$ \\
\end{tabular} 
\end{center}

Bemærk, at $\sqrt{2} \notin \mathbb{Z}$ og derfor har $2$ ikke en kvadratrod i $\mathbb{F}_5$.
Dette giver os, at $E(\mathbb{F}_5)$ har orden $7$ og vi skriver $\#E(\mathbb{F}_5)=7$. Vi skal
i dette kapitel vise Hasses sætning, som giver en vurdering for antallet af punkter på en elliptisk
kurve over et endeligt legeme:

\begin{theorem}[Hasse]
\label{hasse}
Lad $E$ være en elliptisk kurve over et endeligt legeme $\mathbb{F}_q$. Da gælder der, at 
\begin{align*}
	|q + 1 - \#E(\mathbb{F}_q)| \leq 2 \sqrt{q}.
\end{align*}
\end{theorem}
Vi vil i kapitel 3 se på en af anvendelserne, som disse elliptiske kurver over endelige legemer har, 
nemlig indenfor faktorisering af heltal.

\section{Endomorfier}
Vi skal først have etableret nogle resultater vedrørende endomorfier på endelige legemer, som
er nødvendige for beviset af Hasses sætning. Lad $K$ være et legeme og $\clk$ dens tilhørende algebraiske aflukning. Når vi skriver om en elliptisk kurve $E$ menes den at være på formen $y^2 = x^3 + Ax + B$. 

Vi begynder da med følgende definition:

\begin{definition}
En endomorfi på $E$ er en homomorfi $\alpha : E(\clk) \to E(\clk)$ givet
ved rationale funktioner.
\end{definition}

Med en rational funktion forstår vi en kvotient af polynomier. Det vil altså sige, at 
en endomorfi $\alpha$ skal opfylde, at $\alpha(P_1 + P_2) = \alpha(P_1) + \alpha(P_2)$ 
og der skal findes rationale 
funktioner $R_1(x, y)$ og $R_2(x, y)$, begge med koefficienter i $\clk$, så
\begin{align*}
	\alpha(x, y) = (R_1(x, y), R_2(x, y)),
\end{align*}
for alle $(x, y) \in E(\clk)$. da $\alpha$ er en homomorfi gælder der, at $\alpha(\infty)=\infty$. Den trivielle endomorfi angives med $0$ og er den endomorfi, som sender ethvert punkt til $\infty$. Vi vil fremover antage, at $\alpha$ ikke er den trivielle endomorfi, hvilket betyder at der findes $(x, y)$ sådan at $\alpha(x, y) \neq \infty$.

\begin{example}
Skal vi have en endomorfi her?
\end{example}

Vi ønsker nu, at finde en standard repræsentation for de rationale funktioner, som en endomorfi er givet ved. Følgende sætning gør dette muligt for os:

\begin{theorem}
\label{end_rep_theorem}
Lad $E$ være en elliptisk kurve over et legeme $K$. En endomorfi $\alpha$ kan da skrives som
\begin{align*}
	\alpha(x, y) = (r_1(x), r_2(x)y) = \left( \frac{p(x)}{q(x)}, \frac{s(x)}{t(x)}y \right),
\end{align*}
hvor $p, q$ henholdsvis $s, t$ ikke har nogen fælles faktor.
\end{theorem}
\begin{proof}
For et punkt $(x, y) \in E(\clk)$ gælder der, at $y^2 = x^3 + Ax + B$ så vi har også, at 
\begin{align*}
	y^{2k} = (x^3 + Ax + B)^k \quad \text{og} \quad y^{2k+1}= y^{2k}y = (x^3 + Ax + B)^k y, \quad k \in \mathbb{N}.
\end{align*}
Vi kan altså erstatte en lige potens af $y$ med et polynomium der kun afhænger af $x$, og en ulige potens med $y$ ganget med et polynomium der kun afhænger af $x$. For en rational funktion $R(x, y)$ kan vi da beskrive en anden rational funktion, som stemmer overens med denne på punkter fra $E(\clk)$. Vi kan altså antage, at
\begin{align}
	\label{rational_1}
	R(x, y) = \frac{p_1(x) + p_2(x)y}{p_3(x)+p_4(x)y}.
\end{align}
Vi kan endda gøre det endnu simplere ved at gange udtrykket i \eqref{rational_1} med $p_3(x)-p_4(x)$, hvilket giver 
\begin{align*}
	(p_3(x) - p_4(x)y)(p_3(x)+p_4(x)y) = p_3(x)^2 - p_4(x)^2 y^2,
\end{align*}
hvorefter vi kan erstatte $y^2$ med $x^3 + Ax + B$. Dette giver os altså, at 
\begin{align}
	\label{rational_final}
	R(x, y) = \frac{q_1(x) + q_2(x)y}{q_3(x)}.
\end{align}
Da $\alpha$ er en endomorfi er den givet ved
\begin{align*}
	\alpha(x, y) = (R_1(x, y), R_2(x, y)),
\end{align*}
hvor $R_1$ og $R_2$ er rationale funktioner. Da $\alpha$ specielt er en homomorfi bevarer den strukturen for en elliptisk kurve så vi har, at
\begin{align*}
	\alpha(x, -y) = \alpha(-(x, y)) =  -\alpha(x, y).
\end{align*}
Dette medfører, at 
\begin{align*}
	R_1(x, -y) = R_1(x, y) \quad \text{og} \quad R_2(x, -y) = -R_2(x, y).
\end{align*}
Skriver vi $R_1$ på samme form som i \eqref{rational_final} må $q_2(x) = 0$, og ligeledes må vi for $R_2$ have at $q_1(x) = 0$. Vi kan altså antage, at
\begin{align*}
		\alpha(x, y) = (r_1(x), r_2(x)y),
\end{align*}
hvor $r_1(x)$ og $r_2(x)$ er rationale funktioner. Skriv da
\begin{align*}
	r_1(x) = \frac{p(x)}{q(x)} \quad \text{og} \quad \frac{s(x)}{t(x)}y,
\end{align*}
hvor $p, q$ henholdsvis $s, t$ ikke har nogen fælles faktorer. Hvis $q(x)=0$ for et punkt $(x, y)$ lader vi 
$\alpha(x, y) = \infty$. Hvis $q(x) \neq 0$ giver (ii) i lemma \ref{rational_lemma}, at $r_2(x)$ da også vil være defineret.
\end{proof}

\begin{lemma}
\label{rational_lemma}
Lad $\alpha$ være en endomorfi givet ved
\begin{align*}
	\alpha(x, y) = \left( \frac{p(x)}{q(x)}, \frac{s(x)}{t(x)}y \right),
\end{align*}
for en elliptisk kurve $E$. Lad $p, q$ henholdsvis $s, t$ være sådan, at de ikke har nogen fælles rødder. Da har vi, at
\begin{enumerate}[(i)]
	\item For et polynomium $u(x)$, som ikke har en fælles rod med $q(x)$ har vi, at 
	\begin{align*}
		\frac{(x^3+Ax+B)s(x)^2}{t(x)^2} = \frac{u(x)}{q(x)^3}.
	\end{align*}
	\item Hvis $t(x_0)=0$ er $q(x_0)=0$, hvilket betyder at hvis $q(x_0) \neq 0$ er $\alpha(x_0, y_0)$ defineret.
\end{enumerate}
\end{lemma}
\begin{proof}
(i) For et punkt $(x, y) \in E(K)$ har vi også, at $\alpha(x, y) \in E(K)$, da $\alpha$ er en endomorfi. Derfor har vi, at
\begin{align*}
	\frac{(x^3 + Ax + B)s(x)^2}{t(x)^2} &= \frac{y^2 s(x)^2}{t(x)^2} = \left( \frac{s(x)}{t(x)}y \right)^2 \\
	&= \left( \frac{p(x)}{q(x)} \right)^3 + A \left( \frac{p(x)}{q(x)} \right) + B \\
	&= \frac{p(x)^3 + A p(x) q(x)^2 + B q(x)^3}{q(x)^3} = \frac{u(x)}{q(x)^3},
\end{align*}
hvor $u(x) = p(x)^3 + A p(x) q(x)^2 + B q(x)^3$. Vi skal dog stadigvæk vise, at $u(x)$ og $q(x)$ ikke har en fælles rod. Antag, at $q(x_0) = 0$. Hvis $u(x_0)=0$ har vi, at
\begin{align*}
	u(x_0)=p(x_0)^3 + A p(x_0) q(x_0)^2 + B q(x_0)^3 = 0 &\Rightarrow p(x_0)^3 = 0 \\ &\Rightarrow p(x_0)=0,
\end{align*}
men $p$ og $q$ havde pr. antagelse ingen fælles rødder. Så hvis $q(x_0) = 0$ må $u(x_0) \neq 0$ og de har dermed ingen fælles rødder.

(ii) 

\end{proof}

Med denne standard repræsentation for endomorfier er vi nu i stand til at give en definition for graden af en endomorfi:

\begin{definition}
Graden af en endomorfi $\alpha$ er givet ved
\begin{align*}
	\deg(\alpha) = \max \{ \deg p(x), \deg q(x) \},
\end{align*}
når $\alpha$ ikke er den trivielle endomorfi, altså for $\alpha \neq 0$. For $\alpha = 0$ lader vi $\deg (\alpha) = 0$.
\end{definition}

En endomorfi siges at være separabel hvis den afledede $r_1'(x) \neq 0$.

\begin{example}
Eksempel på en separabel endomorfi. Bogen ser på $2P$ som også er oplagt, men måske skulle man vælge en mere interessant.
\end{example}

Den følgende proposition er essentiel idet, at det tilknytter graden af en endomorfi til antallet af elementer i kernen for selvsamme endomorfi, hvilket vi skal benytte direkte i beviset for Hasses sætning.

\begin{proposition}
\label{deg_to_ker}
Lad $E$ være en elliptisk kurve. Lad $\alpha \neq 0$ være en separabel 
endomorfi for $E$. Da er 
\begin{align*}
	\deg \alpha = \# \ker (\alpha),
\end{align*}
hvor $\ker (\alpha) = $ angiver kernen for homomorfien 
$\alpha : E(\clk) \to E(\clk)$. I tilfældet hvor $\alpha \neq 0$ ikke
er separabel gælder der, at 
\begin{align*}
	\deg \alpha > \# \ker (\alpha).
\end{align*}
\end{proposition}
\begin{proof}
Vi skriver $\alpha$ på standardformen, som vi introducerede tidligere, altså
sættes
\begin{align*}
	\alpha(x, y) = (r_1(x), r_2(x) y)),
\end{align*}
hvor $r_1(x) = p(x)/q(x)$. Da $\alpha$ er antaget til at være separabel er 
$r_1' \neq 0$ og dermed er $pq'-p'q$ ikke nulpolynomiet. Lad nu
\begin{align*}
	S = \{ x \in \clk \mid (pq' - p'q)(x)q(x) = 0 \}.
\end{align*}
Lad da $(a, b) \in E(\clk)$ være valgt sådan, at følgende er opfyldt
\begin{enumerate}
	\item $a \neq 0$, $b \neq 0$ og $(a, b) \neq \infty$,
	\item $\deg (p(x) - aq(x)) = \max \{ \deg p(x), \deg q(x) \} = \deg \alpha$,
	\item $a \notin r_1(S)$,
	\item $(a, b) \in \alpha(E(\clk))$.
\end{enumerate}
Da $pq'-p'q$ ikke er 
nulpolynomiet er $S$ en endelig mængde, hvilket dermed også betyder, at 
$\alpha(S)$ er en endelig mængde. Funktionen $r_1(x)$ antager uendeligt 
mange forskellige værdier når $x$ gennemløber $\clk$, da en algebraisk aflukning indeholder uendeligt mange elementer.
%da en algebraisk aflukning indeholder uendeligt mange elementer.
Da der for hvert $x$ er et punkt $(x, y) \in E(\clk)$ følger det, at 
$\alpha(E(\clk))$ er en uendelig mængde. Det er altså muligt, at vælge et
punkt $(a, b) \in E(\clk)$ med egenskaberne ovenfor.

Vi ønsker at vise, at der netop er $\deg \alpha$ punkter $(x_1, y_1) \in E(\clk)$ sådan,
at $\alpha(x_1, y_1) = (a, b)$. For et sådan punkt gælder der, at 
\begin{align*}
	\frac{p(x_1)}{q(x_1)} = a, \quad r_2(x_1) y_1 = b.
\end{align*}
Da $(a, b) \neq \infty$ er $q(x_1) \neq 0$. % Øvelse giver, at r_2(x_1) er defineret
Da $b \neq 0$ har vi også, at $y_1 = b / r_2(x_1)$. Dette betyder, at $y_1$ er bestemt
ved $x_1$, så vi behøver kun at tælle værdier for $x_1$. Fra antagelse (2) har vi, at 
$p(x) - aq(x) = 0$ har $\deg \alpha$ rødder talt med multiplicitet. Vi skal altså vise,
at $p-aq$ ikke har nogen multiple rødder. Antag for modstrid, at $x_0$ er en multipel
rod. Da har vi, at
\begin{align*}
	p(x_0) - aq(x_0) = 0 \quad \text{og} \quad p'(x_0) - aq'(x_0) = 0.
\end{align*}
Dette kan omskrives til ligningerne $p(x_0)=aq(x_0)$ og $aq'(x_0)=p'(x_0)$, 
som vi ganger med hinanden og får, at 
\begin{align*}
	a p(x_0)q'(x_0) = ap'(x_0)q(x_0).
\end{align*} 
Da $a \neq 0$ pr. (1) giver det os, at $x_0$ er en rod i $pq'-p'q$ så $x_0 \in S$.
Altså er $a=r_1(x_0) \in r_1(S)$, hvilket er i modstrid med (3). Dermed har 
$p-aq$ netop $\deg \alpha$ forskellige rødder. Da der er præcist $\deg \alpha$ 
punkter $(x_1, y_1)$ så $\alpha(x_1, y_1) = (a, b)$ har kernen for $\alpha$ netop
$\deg \alpha$ elementer.
\end{proof}


\section{Frobenius endomorfien}

En endomorfi med en absolut kritisk rolle for teorien om elliptiske kurver over endelige legemer er Frobenius endomorfien $\phi_q$. For en elliptisk kurve $E$ over et endeligt legeme $\mathbb{F}_q$ er denne givet ved
\begin{align}
	\phi_q (x, y) = (x^q, y^q),
\end{align}
og $\phi_q(\infty) = \infty$.
Denne endomorfi spiller en vigtig rolle i beviset for Hasses sætning, men vi skal først vise nogle af dens egenskaber.

\begin{lemma}
\label{lemma_end_degree_not_sep}
Lad $E$ være en elliptisk kurve over $\mathbb{F}_q$. Da er $\phi_q$ en 
endomorfi for $E$ af grad $q$, desuden er $\phi_q$ ikke seperabel.
\end{lemma}
\begin{proof}
Vi skal vise, at $\phi_q : E(\clfq) \to E(\clfq)$ er en homomorfi. Lad da $(x_1, y_1), (x_2, y_2) \in E(\clfq)$, hvor $x_1 \neq x_2$. Da følger det fra gruppeloven, at summen af de to punkter $(x_3, y_3) = (x_1, y_1) + (x_2, y_2)$ er givet ved
\begin{align*}
	x_3 = m^2 - x_1 - x_2, \quad y_3 = m(x_1 - x_3) - y_1, \ \ \text{hvor} \ m=\frac{y_2-y_1}{x_2-x_1}.
\end{align*}
Opløftes dette i $q$'ende potens får vi, at 
\begin{align*}
	{x_3}^q = {m'}^2 - {x_1}^q - {x_2}^q, \quad {y_3}^q = m'({x_1}^q - {x_3}^q) - {y_1}^q, \ \
	\text{hvor} \ m' = \frac{{y_2}^q - {y_1}^q}{{x_2}^q - {x_1}^q}.
\end{align*}
Dette giver os, at $\phi_q(x_3, y_3) = \phi_q(x_1, y_1) + \phi_q(x_2, y_2)$, hvilket netop er hvad $\phi_q$ skal opfylde for at være en homomorfi. I tilfældet hvor $x_1=x_2$ har vi fra gruppeloven, at 
$(x_3, y_3) =(x_1, y_1) + (x_2, y_2) = \infty$. Men hvis $x_1 = x_2$ må ${x_1}^q = {x_2}^q$ hvilket betyder, at $\phi_q(x_1, y_1)+\phi_q(x_2, y_2) = \infty$. Så da $\infty^q = \infty$ (lægges $\infty$ sammen $q$ gange er det stadigvæk $\infty$) får vi, at 
\begin{align*}
	\phi_q(x_3, y_3) = \phi_q(x_1, y_1) + \phi_q(x_2,  y_2).
\end{align*}
Hvis ét af punkterne er $\infty$, eksempelvis $(x_1, y_1)=\infty$, har vi fra gruppeloven, at 
$(x_3, y_3) = (x_1, y_1)+(x_2, y_2) = (x_2, y_2)$. Bruger vi igen, at $\infty^q = \infty$ følger det direkte, at 
$\phi_q(x_3, y_3) = \phi_q(x_1, y_1) + \phi_q(x_2, y_2)$.

Når $(x_1, y_1)=(x_2, y_2)$ hvor $y_1 = 0$ er $(x_3, y_3) = (x_1, y_1)+(x_2, y_2) = \infty$. Når $y_1=0$ er ${y_1}^q=0$ så $\phi_q(x_1, y_1) + \phi_q(x_2, y_2) = \infty$ og dermed er $\phi_q(x_3, y_3) = \phi_q(x_1, y_1) + \phi_q(x_2, y_2)$. 

Det resterende tilfælde er når $(x_1, y_1)=(x_2, y_2)$ og $y_1 \neq 0$. Fra gruppeloven har vi, at 
$(x_3, y_3)= 2(x_1, y_1)$, hvor
\begin{align*}
	x_3 = m^2 - 2x_1, \quad y_3 = m(x_1-x_3)-y_1, \ \ \text{hvor} \ m=\frac{3{x_1}^2+A}{2y_1}.
\end{align*}
Som tidligere opløftes dette til den $q$'ende potens
\begin{align*}
	{x_3}^q = {m'}^2 - 2{x_1}^q, \quad {y_3}^q=m'({x_1}^q-{x_3}^q) - {y_1}^q, \ \ \text{hvor}
	\ m' = \frac{3^q ({x_1}^q)^2 + A^q}{2^q {y_1}^q}.
\end{align*}
Idet, at $2, 3, A \in \mathbb{F}_q$ følger det, at $2^q = 2, 3^q = 3$ og $A^q = A$. Vi står altså tilbage
med formlen for fordoblingen af punktet $({x_1}^q, {y_1}^q)$ på den elliptiske kurve $E$. 

Dermed har vi vist, at $\phi_q$ er en homomorfi for $E$. Da $\phi_q(x, y) = (x^q, y^q)$ er givet ved polynomier, som specielt er rationale funktioner, er $\phi_q$ en endomorfi. Den har tydeligvis grad $q$. Da $q=0$ i $\mathbb{F}_q$ er den afledte af $x^q$ lig nul, hvilket betyder at $\phi_q$ ikke er separabel.
\end{proof}

Bemærk, at da $\phi_q$ er en endomorfi for $E$ er ${\phi_q}^2 = \phi_q \circ \phi_q$ det også og dermed også 
${\phi_q}^n = \phi_q \circ \phi_q \circ \ldots \circ \phi_q$ for $n \geq 1$. Da multiplikation med $-1$ også er en endomorfi er ${\phi_q}^n - 1$ også en endomorfi for $E$.

\begin{lemma}
\label{lemma03}
Lad $E$ være en elliptisk kurve over $\mathbb{F}_q$, og lad 
$(x, y) \in E(\overline{\mathbb{F}}_q)$. Da gælder der, at 
\begin{enumerate}
	\item $\phi_q(x, y) \in E(\overline{\mathbb{F}}_q)$,
	\item $(x, y) \in E(\mathbb{F}_q) \Leftrightarrow \phi_q(x, y)=(x, y)$.
\end{enumerate}
\end{lemma}
\begin{proof}
Vi har, at $y^2 = x^3 + ax + b$, hvor $a, b \in \mathbb{F}_q$. Vi opløfter 
denne ligning til den $q$'ende potens og får, at 
\begin{align*}
	(y^q)^2 = (x^q)^3 + (a^q x^q) + b^q,
\end{align*}
hvor vi har brugt, at $(a+b)^q = a^q + b^q$ når $q$ er en potens af legemets karakteristik 
(detaljer placeres i appendiks?).
Men dette betyder netop, at 
$(x^q, y^q) \in E(\overline{\mathbb{F}}_q)$, hvilket viser (1).
For at vise (2) husker vi, at $\phi_q(x) = x \Leftrightarrow x \in \mathbb{F}_q$.
Det følger da, at 
\begin{align*}
	(x, y) \in E(\mathbb{F}_q) &\Leftrightarrow x, y \in \mathbb{F}_q \\
	&\Leftrightarrow \phi_q(x) = x \ \text{og} \ \phi_q(y) = y \\
	&\Leftrightarrow \phi_q(x, y) = (x, y),
\end{align*}
hvilket fuldfører beviset for (2).
\end{proof}

\begin{proposition}
\label{prop_imp}
Lad $E$ være en elliptisk kurve over $\mathbb{F}_q$ og lad $n \geq 1$. Da gælder der,
at 
\begin{enumerate}
	\item $\ker (\phi_{q}^n - 1) = E(\mathbb{F}_{q^n})$. \label{test}
	\item $\phi_{q}^{n}-1$ er separabel, så $\#E(\mathbb{F}_{q^n})=\deg (\phi_{q}^{n}-1)$. 
\end{enumerate}
\end{proposition}
\begin{proof}
Da $(\phi_{q}^{n} - 1)(x,y) = 0 \Leftrightarrow (x^q, y^q) = (x, y)$ følger det fra
lemma \ref{lemma03}, at $\ker(\phi_{q}^{n}-1)=E(\mathbb{F}_q)$.
Da $\phi_{q}^{n}$ er Frobenius afbildningen for $\mathbb{F}_{q^n}$ følger \eqref{test} fra lemma \ref{lemma03}. At $\phi_{q}^{n} -1$ er separabel vil vi ikke vise, men et bevis kan findes
i [LW]. Da følger det fra proposition \ref{deg_to_ker}, at $\#E(E_{q^n})=\deg(\phi_{q}^{n} -1)$.
\end{proof}

\section{Hasses sætning}
Med de foregående resultater er vi nu næsten klar til at vise Hasses sætning (sætning \ref{hasse}). Lad i det følgende afsnit
\begin{align}
	\label{hasse_as}
	a = q + 1 - \#E(\mathbb{F}_q) = q + 1 - \deg(\phi_q - 1).
\end{align}
Da skal vi vise, at $|a| \leq 2 \sqrt{q}$ for at vise Hasses sætning. Først har vi dog følgende lemma

\begin{lemma}
\label{degree_lemma}
Lad $r, s \in \mathbb{Z}$ så $\gcd(s, q) = 1$. Da er 
\begin{align*}
	\deg(r \phi_q - s) = r^2 q + s^2 - rsa.
\end{align*}
\end{lemma}
\begin{proof}
Vi vil ikke give beviset her, da det bygger på en række af tekniske resultater. Et bevis kan findes i [LW].
\end{proof}

Nu er vi da i stand til, at gives beviset for Hasses sætning:

\begin{proof}[Bevis for Hasses sætning]
Da graden af en endomorfi altid er $\geq 0$ følger det fra lemma \ref{degree_lemma}, at 
\begin{align*}
	r^2q+s^2 -rsa = q \left( \frac{r^2}{s^2} \right) - \frac{rsa}{s^2} + 1 
	\geq 0,
\end{align*}
for alle $r, s \in \mathbb{Z}$ med $\gcd(s, q)=1$. Da mængden
\begin{align*}
	\left\{ \frac{r}{s} \mid \gcd(s, q)=1 \right\} \subseteq \mathbb{Q},
\end{align*}
er tæt i $\mathbb{R}$ (se appendiks?) følger det, at $qx^2 - ax + 1 \geq 0$,
for alle $x \in \mathbb{R}$. Dette medfører at diskrimanten må være negativ eller lig $0$.
Altså har vi, at 
\begin{align*}
	a^2 - 4q \leq 0 \Rightarrow |a| \leq 2 \sqrt{q},
\end{align*}
hvilket viser Hasses sætning.
\end{proof}

Eventuelt afsnit for torsionspunkter?

Følgende sætning følger også fra proposition \ref{prop_imp}, som vil vise sig at være nyttigt til at udvide resultatet fra Hasses sætning.

\begin{theorem}
\label{trace_theorem}
Lad $E$ være en elliptisk kurve over $\mathbb{F}_q$. Lad $a$ være som i \eqref{hasse_as}. Da er $a$ det
entydige heltal så
\begin{align*}
	{\phi_q}^2 - a \phi_q + q = 0,
\end{align*}
set som endomorfier. Med andre ord er $a$ det entydige heltal sådan, at 
\begin{align*}
	({x^q}^2, {y^q}^2) - a(x^q, y^q) + q(x, y) = \infty,
\end{align*}
for alle $(x, y) \in E(\clfq)$. Desuden er $a$ det entydige heltal der opfylder, at
\begin{align*}
	a \equiv \text{Trace}((\phi_q)_m) \modu{m},
\end{align*}
for alle $m$, hvor $\gcd(m, q)=1$.
\end{theorem}

Før vi starter på beviset for sætning \ref{trace_theorem} skal vi først se på torsions punkterne for en elliptisk kurve. For en elliptisk kurve $E$ givet over et legeme $K$ lader vi
\begin{align*}
	E[n] = \{ P \in E(\clk) \mid nP= \infty \}.
\end{align*}
Det er altså de punkter, hvis orden er endelig (alle punkter over et endeligt legeme er torsions punkter). 

Opskriv eventuelt sætning 3.2?

Lad da $\{\beta_1, \beta_2\}$ være en basis for $E[n] \simeq \mathbb{Z}_n \oplus \mathbb{Z}_n$. Ethvert element fra $E[n]$ kan altså skrives som $\beta_1 m_1 + \beta_2 m_2$, hvor $m_1, m_2 \in \mathbb{Z}$ er entydige mod $n$. For en homomorfi $\alpha : E(\clk) \to E(\clk)$ afbilleder $\alpha$ torsionspunkterne $E[n]$ til $E[n]$, derfor findes $a, b, c, d \in \mathbb{Z}$ sådan, at 
\begin{align*}
	\alpha(\beta_1) = a \beta_1 + b \beta_2, \quad \alpha(\beta_2) = c\beta_1 + d \beta_2.
\end{align*}
Vi kan altså repræsentere en sådan homomorfi med matricen
\begin{align*}
	\alpha_n = \left( 
	\begin{matrix}
		a & b \\ 
		c & d 
	\end{matrix} \right).
\end{align*}

\begin{proof}[Bevis for sætning \ref{trace_theorem}]
Det følger direkte fra lemma \ref{deg_to_ker}, at hvis ${\phi_q}^2 -a \phi_q + q \neq 0$, altså hvis den ikke er nul-endomorfien, da er dens kerne endelig. Så hvis vi kan vise, at kernen er uendelig, da må endomorfien være lig $0$.

Lad nu $m \geq 1$ være valgt sådan, at $\gcd(m, q) = 1$. Lad da $(\phi_q)_m$ være den matricen, som beskriver virkningen af $\phi_q$ på $E[m]$, som vi beskrev ovenfor. Lad da
\begin{align*}
	(\phi_q)_m = \left( 
	\begin{matrix}
		s & t \\
		u & v
	\end{matrix} \right).
\end{align*}
Da $\phi_q - 1$ er separabel følger det fra proposition \ref{deg_to_ker} og 3.15 (nævn resultat og henvis?, at
\begin{align*}
	\# \ker (\phi_q - 1) &= \deg(\phi_q - 1) \equiv \det((\phi_q)_m - I) \\ 
	&= 
	\left| \begin{matrix}
		s-1 & t \\
		u & v - 1 
	\end{matrix} \right| \\
	&= sv - tu - (s + v) + 1 \modu{m}.
\end{align*}
Fra 3.15 (henvis, opskriv?) har vi, at $sv-tu = \det((\phi_q)_m) \equiv q \modu{m}$. Fra \eqref{hasse_as} har vi, at 
$\# \ker(\phi_q - 1) = q + 1 - a$ så det følger, at 
\begin{align*}
	\text{Trace}((\phi_q)_m) = s + v \equiv a \modu{m}.
\end{align*}
Idet vi husker, at $X^2 - aX + q$ er det karakteristiske polynomium for $(\phi_q)_m$ følger det fra
Cayley-Hamiltons sætning fra lineær algebra, at
\begin{align*}
	{(\phi_q)_m}^2 - a(\phi_q)_m + qI \equiv I \modu{m},
\end{align*}
hvor $I$ er $2 \times 2$ identitetsmatricen. Vi har da, at endomorfien ${\phi_q}^2 -a\phi_q + q$ er nul på $E[m]$. Da der er uendeligt mange muligheder for valget af $m$ er kernen for ${\phi}^2 - a\phi_q + q$ uendelig. Dermed er endomorfien lig $0$.

Mangler beviset for entydigheden af $a$.
\end{proof}

Endeligt vil vi vise en sætning, som gør det muligt at bestemme ordenen af en gruppe af punkter for en elliptisk kurve. Hvis vi kender ordenen af $E(\mathbb{F}_q)$ for et lille endeligt legeme gør følgende sætning det muligt, at bestemme ordenen af $E(\mathbb{F}_{q^n})$.

Sætning 4.12 og bevis, som afslutning på kapitlet.





