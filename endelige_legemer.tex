\chapter{Elliptiske kurver over endelige legemer}

Vi skal i dette kapitel undersøge elliptiske kurver over endelige legemer. 
Lad $\mathbb{F}$ være et endeligt legeme og lad $E$ være en elliptisk kurve på
formen 
\begin{align*}
	y^2 = x^3 + Ax + B,
\end{align*}
som er defineret over $\mathbb{F}$. Da er gruppen $E(\mathbb{F})$ endelig, da 
der kun findes endeligt mange talpar $(x, y)$ så $x, y \in \mathbb{F}$. Lad 
$E$ være den elliptiske kurve $y^2 = x^3 - x$ over $\mathbb{F}_5$. For at bestemme
ordenen af $E(\mathbb{F})$ laver vi en tabel over mulige værdier for $x$, $x^3 - x \modu{5}$
og for $y$ som er kvadratrødderne af $x^3 - x$. Dette giver os samtlige punkter på kurven:


\begin{center}
\begin{tabular}{c c c c }
$x$ & $x^3 - x$ & $y$ & Punkter \\ 
\hline
$0$ & $0$ & $0$ & $(0, 0)$ \\ 
$1$ & $0$ & $0$ & $(1, 0)$ \\ 
$2$ & $1$ & $\pm 1$ & $(2, 1), (2, 4)$ \\ 
$3$ & $4$ & $\pm 2$ & $(3, 2), (3, 3)$ \\ 
$4$ & $2$ & $-$ & $-$ \\ 
$\infty$ & & $\infty$ & $\infty$ \\
\end{tabular} 
\end{center}

Bemærk, at $\sqrt{2} \notin \mathbb{Z}$ og derfor har $2$ ikke en kvadratrod i $\mathbb{F}_5$.
Dette giver os, at $E(\mathbb{F}_5)$ har orden $7$ og vi skriver $\#E(\mathbb{F}_5)=7$. Vi skal
i dette kapitel vise Hasses sætning, som giver en vurdering for antallet af punkter på en elliptisk
kurve over et endeligt legeme:

\begin{theorem}[Hasse]
\label{hasse}
Lad $E$ være en elliptisk kurve over et endeligt legeme $\mathbb{F}_q$. Da gælder der, at 
\begin{align*}
	|q + 1 - \#E(\mathbb{F}_q)| \leq 2 \sqrt{q}.
\end{align*}
\end{theorem}
Vi vil i kapitel 3 se på en af anvendelserne, som disse elliptiske kurver over endelige legemer har, 
nemlig indenfor faktorisering af heltal.


\section{Frobenius endomorfien}

En endomorfi med en absolut kritisk rolle for teorien om elliptiske kurver over endelige legemer er Frobenius endomorfien $\phi_q$. For en elliptisk kurve $E$ over et endeligt legeme $\mathbb{F}_q$ er denne givet ved
\begin{align}
	\phi_q (x, y) = (x^q, y^q),
\end{align}
og $\phi_q(\infty) = \infty$.
Denne endomorfi spiller en vigtig rolle i beviset for Hasses sætning, men vi skal først vise nogle af dens egenskaber.

\begin{lemma}
\label{lemma_end_degree_not_sep}
Lad $E$ være en elliptisk kurve over $\mathbb{F}_q$. Da er $\phi_q$ en 
endomorfi for $E$ af grad $q$, desuden er $\phi_q$ ikke seperabel.
\end{lemma}
\begin{proof}
Vi skal vise, at $\phi_q : E(\clfq) \to E(\clfq)$ er en homomorfi. Lad da $(x_1, y_1), (x_2, y_2) \in E(\clfq)$, hvor $x_1 \neq x_2$. Da følger det fra gruppeloven, at summen af de to punkter $(x_3, y_3) = (x_1, y_1) + (x_2, y_2)$ er givet ved
\begin{align*}
	x_3 = m^2 - x_1 - x_2, \quad y_3 = m(x_1 - x_3) - y_1, \ \ \text{hvor} \ m=\frac{y_2-y_1}{x_2-x_1}.
\end{align*}
Opløftes dette i $q$'ende potens får vi, at 
\begin{align*}
	{x_3}^q = {m'}^2 - {x_1}^q - {x_2}^q, \quad {y_3}^q = m'({x_1}^q - {x_3}^q) - {y_1}^q, \ \
	\text{hvor} \ m' = \frac{{y_2}^q - {y_1}^q}{{x_2}^q - {x_1}^q}.
\end{align*}
Dette giver os, at $\phi_q(x_3, y_3) = \phi_q(x_1, y_1) + \phi_q(x_2, y_2)$, hvilket netop er hvad $\phi_q$ skal opfylde for at være en homomorfi. I tilfældet hvor $x_1=x_2$ har vi fra gruppeloven, at 
$(x_3, y_3) =(x_1, y_1) + (x_2, y_2) = \infty$. Men hvis $x_1 = x_2$ må ${x_1}^q = {x_2}^q$ hvilket betyder, at $\phi_q(x_1, y_1)+\phi_q(x_2, y_2) = \infty$. Så da $\infty^q = \infty$ (lægges $\infty$ sammen $q$ gange er det stadigvæk $\infty$) får vi, at 
\begin{align*}
	\phi_q(x_3, y_3) = \phi_q(x_1, y_1) + \phi_q(x_2,  y_2).
\end{align*}
Hvis ét af punkterne er $\infty$, eksempelvis $(x_1, y_1)=\infty$, har vi fra gruppeloven, at 
$(x_3, y_3) = (x_1, y_1)+(x_2, y_2) = (x_2, y_2)$. Bruger vi igen, at $\infty^q = \infty$ følger det direkte, at 
$\phi_q(x_3, y_3) = \phi_q(x_1, y_1) + \phi_q(x_2, y_2)$.

Når $(x_1, y_1)=(x_2, y_2)$ hvor $y_1 = 0$ er $(x_3, y_3) = (x_1, y_1)+(x_2, y_2) = \infty$. Når $y_1=0$ er ${y_1}^q=0$ så $\phi_q(x_1, y_1) + \phi_q(x_2, y_2) = \infty$ og dermed er $\phi_q(x_3, y_3) = \phi_q(x_1, y_1) + \phi_q(x_2, y_2)$. 

Det resterende tilfælde er når $(x_1, y_1)=(x_2, y_2)$ og $y_1 \neq 0$. Fra gruppeloven har vi, at 
$(x_3, y_3)= 2(x_1, y_1)$, hvor
\begin{align*}
	x_3 = m^2 - 2x_1, \quad y_3 = m(x_1-x_3)-y_1, \ \ \text{hvor} \ m=\frac{3{x_1}^2+A}{2y_1}.
\end{align*}
Som tidligere opløftes dette til den $q$'ende potens
\begin{align*}
	{x_3}^q = {m'}^2 - 2{x_1}^q, \quad {y_3}^q=m'({x_1}^q-{x_3}^q) - {y_1}^q, \ \ \text{hvor}
	\ m' = \frac{3^q ({x_1}^q)^2 + A^q}{2^q {y_1}^q}.
\end{align*}
Idet, at $2, 3, A \in \mathbb{F}_q$ følger det, at $2^q = 2, 3^q = 3$ og $A^q = A$. Vi står altså tilbage
med formlen for fordoblingen af punktet $({x_1}^q, {y_1}^q)$ på den elliptiske kurve $E$. 

Dermed har vi vist, at $\phi_q$ er en homomorfi for $E$. Da $\phi_q(x, y) = (x^q, y^q)$ er givet ved polynomier, som specielt er rationale funktioner, er $\phi_q$ en endomorfi. Den har tydeligvis grad $q$. Da $q=0$ i $\mathbb{F}_q$ er den afledte af $x^q$ lig nul, hvilket betyder at $\phi_q$ ikke er separabel.
\end{proof}

Bemærk, at da $\phi_q$ er en endomorfi for $E$ er ${\phi_q}^2 = \phi_q \circ \phi_q$ det også og dermed også 
${\phi_q}^n = \phi_q \circ \phi_q \circ \ldots \circ \phi_q$ for $n \geq 1$. Da multiplikation med $-1$ også er en endomorfi er ${\phi_q}^n - 1$ også en endomorfi for $E$.

\begin{lemma}
\label{lemma03}
Lad $E$ være en elliptisk kurve over $\mathbb{F}_q$, og lad 
$(x, y) \in E(\overline{\mathbb{F}}_q)$. Da gælder der, at 
\begin{enumerate}
	\item $\phi_q(x, y) \in E(\overline{\mathbb{F}}_q)$,
	\item $(x, y) \in E(\mathbb{F}_q) \Leftrightarrow \phi_q(x, y)=(x, y)$.
\end{enumerate}
\end{lemma}
\begin{proof}
Vi har, at $y^2 = x^3 + ax + b$, hvor $a, b \in \mathbb{F}_q$. Vi opløfter 
denne ligning til den $q$'ende potens og får, at 
\begin{align*}
	(y^q)^2 = (x^q)^3 + (a^q x^q) + b^q,
\end{align*}
hvor vi har brugt, at $(a+b)^q = a^q + b^q$ når $q$ er en potens af legemets karakteristik 
(detaljer placeres i appendiks?).
Men dette betyder netop, at 
$(x^q, y^q) \in E(\overline{\mathbb{F}}_q)$, hvilket viser (1).
For at vise (2) husker vi, at $\phi_q(x) = x \Leftrightarrow x \in \mathbb{F}_q$.
Det følger da, at 
\begin{align*}
	(x, y) \in E(\mathbb{F}_q) &\Leftrightarrow x, y \in \mathbb{F}_q \\
	&\Leftrightarrow \phi_q(x) = x \ \text{og} \ \phi_q(y) = y \\
	&\Leftrightarrow \phi_q(x, y) = (x, y),
\end{align*}
hvilket fuldfører beviset for (2).
\end{proof}

\begin{proposition}
\label{prop_imp}
Lad $E$ være en elliptisk kurve over $\mathbb{F}_q$ og lad $n \geq 1$. Da gælder der,
at 
\begin{enumerate}
	\item $\ker (\phi_{q}^n - 1) = E(\mathbb{F}_{q^n})$. \label{test}
	\item $\phi_{q}^{n}-1$ er separabel, så $\#E(\mathbb{F}_{q^n})=\deg (\phi_{q}^{n}-1)$. 
\end{enumerate}
\end{proposition}
\begin{proof}
Da $(\phi_{q}^{n} - 1)(x,y) = 0 \Leftrightarrow (x^q, y^q) = (x, y)$ følger det fra
lemma \ref{lemma03}, at $\ker(\phi_{q}^{n}-1)=E(\mathbb{F}_q)$.
Da $\phi_{q}^{n}$ er Frobenius afbildningen for $\mathbb{F}_{q^n}$ følger \eqref{test} fra lemma \ref{lemma03}. At $\phi_{q}^{n} -1$ er separabel vil vi ikke vise, men et bevis kan findes
i [LW]. Da følger det fra proposition \ref{deg_to_ker}, at $\#E(E_{q^n})=\deg(\phi_{q}^{n} -1)$.
\end{proof}

\section{Hasses sætning}
Med de foregående resultater er vi nu næsten klar til at vise Hasses sætning (sætning \ref{hasse}). Lad i det følgende afsnit
\begin{align}
	\label{hasse_as}
	a = q + 1 - \#E(\mathbb{F}_q) = q + 1 - \deg(\phi_q - 1).
\end{align}
Da skal vi vise, at $|a| \leq 2 \sqrt{q}$ for at vise Hasses sætning. Først har vi dog følgende lemma

\begin{lemma}
\label{degree_lemma}
Lad $r, s \in \mathbb{Z}$ så $\gcd(s, q) = 1$. Da er 
\begin{align*}
	\deg(r \phi_q - s) = r^2 q + s^2 - rsa.
\end{align*}
\end{lemma}
\begin{proof}
Vi vil ikke give beviset her, da det bygger på en række af tekniske resultater. Et bevis kan findes i [LW].
\end{proof}

Nu er vi da i stand til, at gives beviset for Hasses sætning:

\begin{proof}[Bevis for Hasses sætning]
Da graden af en endomorfi altid er $\geq 0$ følger det fra lemma \ref{degree_lemma}, at 
\begin{align*}
	r^2q+s^2 -rsa = q \left( \frac{r^2}{s^2} \right) - \frac{rsa}{s^2} + 1 
	\geq 0,
\end{align*}
for alle $r, s \in \mathbb{Z}$ med $\gcd(s, q)=1$. Da mængden
\begin{align*}
	\left\{ \frac{r}{s} \mid \gcd(s, q)=1 \right\} \subseteq \mathbb{Q},
\end{align*}
er tæt i $\mathbb{R}$ (se appendiks?) følger det, at $qx^2 - ax + 1 \geq 0$,
for alle $x \in \mathbb{R}$. Dette medfører at diskrimanten må være negativ eller lig $0$.
Altså har vi, at 
\begin{align*}
	a^2 - 4q \leq 0 \Rightarrow |a| \leq 2 \sqrt{q},
\end{align*}
hvilket viser Hasses sætning.
\end{proof}

Eventuelt afsnit for torsionspunkter?

Følgende sætning følger også fra proposition \ref{prop_imp}, som vil vise sig at være nyttigt til at udvide resultatet fra Hasses sætning.

\begin{theorem}
\label{trace_theorem}
Lad $E$ være en elliptisk kurve over $\mathbb{F}_q$. Lad $a$ være som i \eqref{hasse_as}. Da er $a$ det
entydige heltal så
\begin{align*}
	{\phi_q}^2 - a \phi_q + q = 0,
\end{align*}
set som endomorfier. Med andre ord er $a$ det entydige heltal sådan, at 
\begin{align*}
	({x^q}^2, {y^q}^2) - a(x^q, y^q) + q(x, y) = \infty,
\end{align*}
for alle $(x, y) \in E(\clfq)$. Desuden er $a$ det entydige heltal der opfylder, at
\begin{align*}
	a \equiv \text{Trace}((\phi_q)_m) \modu{m},
\end{align*}
for alle $m$, hvor $\gcd(m, q)=1$.
\end{theorem}

Før vi starter på beviset for sætning \ref{trace_theorem} skal vi først se på torsions punkterne for en elliptisk kurve. For en elliptisk kurve $E$ givet over et legeme $K$ lader vi
\begin{align*}
	E[n] = \{ P \in E(\clk) \mid nP= \infty \}.
\end{align*}
Det er altså de punkter, hvis orden er endelig (alle punkter over et endeligt legeme er torsions punkter). 

Opskriv eventuelt sætning 3.2?

Lad da $\{\beta_1, \beta_2\}$ være en basis for $E[n] \simeq \mathbb{Z}_n \oplus \mathbb{Z}_n$. Ethvert element fra $E[n]$ kan altså skrives som $\beta_1 m_1 + \beta_2 m_2$, hvor $m_1, m_2 \in \mathbb{Z}$ er entydige mod $n$. For en homomorfi $\alpha : E(\clk) \to E(\clk)$ afbilleder $\alpha$ torsionspunkterne $E[n]$ til $E[n]$, derfor findes $a, b, c, d \in \mathbb{Z}$ sådan, at 
\begin{align*}
	\alpha(\beta_1) = a \beta_1 + b \beta_2, \quad \alpha(\beta_2) = c\beta_1 + d \beta_2.
\end{align*}
Vi kan altså repræsentere en sådan homomorfi med matricen
\begin{align*}
	\alpha_n = \left( 
	\begin{matrix}
		a & b \\ 
		c & d 
	\end{matrix} \right).
\end{align*}

\begin{proof}[Bevis for sætning \ref{trace_theorem}]
Det følger direkte fra lemma \ref{deg_to_ker}, at hvis ${\phi_q}^2 -a \phi_q + q \neq 0$, altså hvis den ikke er nul-endomorfien, da er dens kerne endelig. Så hvis vi kan vise, at kernen er uendelig, da må endomorfien være lig $0$.

Lad nu $m \geq 1$ være valgt sådan, at $\gcd(m, q) = 1$. Lad da $(\phi_q)_m$ være den matricen, som beskriver virkningen af $\phi_q$ på $E[m]$, som vi beskrev ovenfor. Lad da
\begin{align*}
	(\phi_q)_m = \left( 
	\begin{matrix}
		s & t \\
		u & v
	\end{matrix} \right).
\end{align*}
Da $\phi_q - 1$ er separabel følger det fra proposition \ref{deg_to_ker} og 3.15 (nævn resultat og henvis?, at
\begin{align*}
	\# \ker (\phi_q - 1) &= \deg(\phi_q - 1) \equiv \det((\phi_q)_m - I) \\ 
	&= 
	\left| \begin{matrix}
		s-1 & t \\
		u & v - 1 
	\end{matrix} \right| \\
	&= sv - tu - (s + v) + 1 \modu{m}.
\end{align*}
Fra 3.15 (henvis, opskriv?) har vi, at $sv-tu = \det((\phi_q)_m) \equiv q \modu{m}$. Fra \eqref{hasse_as} har vi, at 
$\# \ker(\phi_q - 1) = q + 1 - a$ så det følger, at 
\begin{align*}
	\text{Trace}((\phi_q)_m) = s + v \equiv a \modu{m}.
\end{align*}
Idet vi husker, at $X^2 - aX + q$ er det karakteristiske polynomium for $(\phi_q)_m$ følger det fra
Cayley-Hamiltons sætning fra lineær algebra, at
\begin{align*}
	{(\phi_q)_m}^2 - a(\phi_q)_m + qI \equiv I \modu{m},
\end{align*}
hvor $I$ er $2 \times 2$ identitetsmatricen. Vi har da, at endomorfien ${\phi_q}^2 -a\phi_q + q$ er nul på $E[m]$. Da der er uendeligt mange muligheder for valget af $m$ er kernen for ${\phi}^2 - a\phi_q + q$ uendelig. Dermed er endomorfien lig $0$.

Mangler beviset for entydigheden af $a$.
\end{proof}

Endeligt vil vi vise en sætning, som gør det muligt at bestemme ordenen af en gruppe af punkter for en elliptisk kurve. Hvis vi kender ordenen af $E(\mathbb{F}_q)$ for et lille endeligt legeme gør følgende sætning det muligt, at bestemme ordenen af $E(\mathbb{F}_{q^n})$.

\begin{theorem}
\label{count}
Lad $\#E(\mathbb{F}_q) = q + 1 - a$. Skriv $X^2 - aX + q= (X-\alpha)(X-\beta)$. Da er
\begin{align*}
	\#E(\mathbb{F}_{q^n}) = q^n + 1 - (\alpha^n + \beta^n),
\end{align*}
for alle $n \geq 1$.
\end{theorem}

Vi har brug for, at $\alpha^n + \beta^n$ er et heltal, hvilket følgende lemma giver os:

\begin{lemma}
Lad $s_n = \alpha^n + \beta^n$. Da er $s_0=2$, $s_1=a$ og $s_{n+1}=as_n - qs_{n-1}$ for alle $n \geq 1$.
\end{lemma}
\begin{proof}
Bemærk først, at $s_0 = \alpha^0 + \beta^0 = 2$ og $s_1 = a$.
Vi ser, at 
\begin{align*}
	(\alpha^2 - a\alpha + q) \alpha^{n-1} = \alpha^{n+1} - a \alpha^n + q\alpha^{n-1} = 0,
\end{align*}
da $\alpha$ er en rod i $X^2 - aX + q$. Altså har vi, at $\alpha^{n+1}=a \alpha^n - q\alpha^{n-1}$. På samme måde har vi, at $\beta^{n+1}=a \beta^n - q \beta^{n-1}$, da $\beta$ også er en rod. Lægges disse udtryk sammen får vi, at
\begin{align*}
	s_{n+1} = \alpha^{n+1} + \beta^{n+1} &= a \alpha^n - q \alpha^{n-1} + a \beta^n - q \beta^{n-1} \\
	&= a (\beta^n + \beta^n) - q (\alpha^{n-1} + \beta^{n-1}) \\
	&= a s_n - q s_{n-1}.
\end{align*}
Dermed er $s_n$ et heltal for alle $n \geq 0$.
\end{proof}

\begin{proof}[Bevis for sætning \ref{count}]
Lad nu 
\begin{align*}
	f(X) = (X^n - \alpha^n)(X^n - \beta^n) = X^{2n} - (\alpha^n + \beta^n)X^n + q^n.
\end{align*}
Da deler $X^2 - aX + q = (X- \alpha)(X-\beta)$ polynomiet $f(X)$. Kvotienten er et polynomium $Q(X)$ med heltallige koefficienter, da $X^2 - aX + q$ er monisk og $f(X)$ har heltallige koefficienter (se appendiks). Derfor er
\begin{align}
	\label{alpha_beta}
	f(\phi_q) = (\phi_{q}^{n})^2 - (\alpha^n + \beta^n) \phi_{q}^{n} + q^n
	= Q(\phi_q)(\phi_{q}^{2} - a \phi_q + q) = 0,
\end{align}
som endomorfier for $E$ pr. sætning \ref{trace_theorem}. Idet vi husker, at $\phi_{q}^{n} = \phi_{q^n}$ giver sætning \ref{trace_theorem} også, at der findes entydigt $k \in \mathbb{Z}$ sådan at $\phi_{q^n}^{2} - k \phi_{q}^{n} + q^n = 0$. Sådan et $k$ er givet ved $k=q^n + 1 - \#E(\mathbb{F}_{q^n})$, og dette sammen med \eqref{alpha_beta} giver os netop, at
\begin{align*}
	\alpha^n + \beta^n = q^n + 1 - \#E(\mathbb{F}_{q^n}),
\end{align*}
hvilket netop var hvad  vi ønskede at vise.
\end{proof}

\begin{example}
Eksempel på 4.12 i aktion.
\end{example}




