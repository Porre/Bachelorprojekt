\section{Lenstras elliptiske kurve algoritme}
I \cite{Lenstra} præsenterede H. W. Lenstra en algoritme til faktorisering af heltal, som anvender elliptiske kurver. Vi skal benytte koncepterne vi udviklede til gruppeloven, men vi skal anvende dem på $E(\mathbb{Z}_n)$, hvor $n$ er et sammensat tal. Når $n$ er et sammensat tal er $\mathbb{Z}_n$ ikke er legeme og derfor kan vi ikke være sikre på at $E(\mathbb{Z}_n)$ er en gruppe. Derfor er det teknisk set ikke elliptiske kurver vi arbejder med, men derimod hvad man kunne kalde for elliptiske pseudokurver. Der vil altså være tidspunkter, hvor en addition fra gruppeloven ikke vil give mening, men det er netop det der kan give os en faktor.

Antag, at $n$ er et sammensat tal og $\gcd(6, n) = 1$ (det udelukker faktorer af $2$ og $3$) og lad $p$ være en primfaktor i $n$. 
Vi ser nu på, hvordan vi kan finde en faktor, når additionen af to punkter ikke er defineret. Vi husker fra gruppeloven, at vi ved additionen af to punkter $P=(x_1, y_1)$ og $P=(x_2, y_2)$ (forskellige fra $\infty$) skal bruge værdien af inverserne til $x_2 - x_1$ og $y_1$ alt efter, hvilket tilfælde vi er i. Disse inverser eksisterer kun modulo $n$, hvis (se proposition \ref{inverse_exists_not} i appendikset)
\begin{align*}
	\gcd(x_1 - x_2, n) = 1 \quad \text{og} \quad \gcd(y_1, n) = 1.
\end{align*}
Når dette ikke er tilfældet giver de i stedet en faktor i $n$, som vi håber på er ikke-triviel.
Hvornår kan vi støde på en situation, hvor denne addition ikke er defineret? Antag, at $x_1 \neq x_2$ og $y_1 \neq y_2$. Hvis vi støder på en udregning der ikke kan lade sig gøre er det fordi, at $(x_1-x_2)^{-1}$ ikke eksisterer modulo $n$. Antag derudover, at $x_1 - x_2$ er et multiplum af $p$. Vi har da, at $x_1 - x_2 \equiv 0 \modu{p}$. Gruppeloven giver os, at når $x_1 \equiv x_2 \modu{p}$ er $P_1 + P_2 = \infty$ på kurven modulo $p$.
Fordoblingen af punktet $P_1$ fejler, når $y_1$ er et multiplum af $p$. Dette sker når $y_1 \equiv 0 \modu{p}$, men da er $2P_1 = \infty$ på kurven modulo $p$.

Generelt har vi altså, at en addition slår fejl på en kurve modulo $n$, når additionen giver $P_1 + P_2 = \infty$ på kurven modulo $p$. Måden hvorpå vi kan finde en faktor er altså ved at foretage disse additioner og fordoblinger på en kurve modulo $n$, hvor vi håber på at en addition giver os $\infty$ på kurven modulo en primfaktor $p$. Vi præsenterer da en algoritme, inspireret af \cite{Silverman} og \cite{Washington}, som forsøger at generere elementet $\infty$ fra $E(\mathbb{F}_p)$.

Vælg først tilfældige heltal $x, y, A$ mellem $1$ og $n$, og sæt
\begin{align*}
	B = y^2 - x^3 - Ax \modu{n}.
\end{align*} 
Vi har da en elliptisk pseudokurve $y^2 = x^3 + Ax + B$, 
hvor vi ved at punktet $P=(x, y)$ er placeret. Vi tjekker da, at
\begin{align*}
	d= \gcd(4A^3 + 27B^2, n) = 1.
\end{align*}
Hvis det ikke er tilfældet og $1 < d < n$ har vi fundet en faktor i $n$. Hvis $d = n$ vælger vi et nyt $A$. For et heltal $B$ lader vi $k=\lcm[1, 2, \ldots, B]$ og vi forsøger da, at bestemme
\begin{align*}
	kP = \underbrace{P + P + \ldots + P}_{k \ \text{led}}.
\end{align*}
Det er ikke praktisk at udregne $P+P+ \ldots + P$ så vi gør ligesom i Pollards $p-1$ algoritme og skriver $k$ som den binære udvidelse
\begin{align*}
	k = k_0 + k_1 \cdot 2 + k_2 \cdot 2^2 + k_ 3 \cdot 2^3 + \ldots + k_r \cdot 2^r,
\end{align*}
hvor alle $k_i$ er $0$ eller $1$. Vi kan da udregne
\begin{align*}
	P_0 &= P \\
	P_1 &= 2P_0 = 2P \\
	P_2 &= 2P_1 = 2^2 P \\
	&\vdots \\
	P_r &= 2 P_{r-1} = 2^r P.
\end{align*}
Da kan vi bestemme $kP= (\text{summen af $P_i$'erne hvor $k_i = 1$})$. I hver udregning regner vi modulo $n$, da tallene ellers bliver alt for store og meget langsommelige at arbejde med. Vi håber på, at der i løbet af de udregninger er en addition, som ikke kan lade sig gøre og at dette giver os vores faktor. 

Vi ved fra Hasses sætning, at $\#E(\mathbb{F}_p)$ er endelig. Lagranges sætning giver os da, at ordenen af punktet $P$ på kurven modulo $p$ deler $\#E(\mathbb{F}_p)$. Hvis $k$ er et multiplum af $\#E(\mathbb{F}_p)$ er $kP = \infty$ på kurven modulo $p$. For at algoritmen kan finde en faktor er det nødvendigt, at for en eller anden primfaktor $p$ i $n$, at $\#E(\mathbb{F}_p)$ er $K$-glat. Fordelen ved Lenstras algoritme over Pollards $p-1$ algoritme viser sig her, da vi i Pollards algoritme kun kan vælge én gruppe, som har orden $p-1$, kan vi her skifte vores elliptiske kurve hvis $\#E(\mathbb{F}_p)$ ikke er $K$-glat.

Vi kan også være uheldige, at løbe ind i den trivielle faktor $n$. Antag, at $n = pq$ for primtal $p$ og $q$. Hvis både $\#E(\mathbb{F}_p)$ og $\#E(\mathbb{F}_q)$ er $K$-glatte da vil $k$ være et multiplum af både $\#E(\mathbb{F}_p)$ og $\#E(\mathbb{F}_q)$. Så $kP = \infty$ på kurven både modulo $p$ og $q$. Dermed er $x_1 - x_2$ og $y_1$ multipla af både $p$ og $q$, så vi får
\begin{align*}
	\gcd(x_1 - x_2, n) = n \quad \text{og} \quad \gcd(y_1, n) = n.
\end{align*}
Vi opsummerer diskussionen af algoritmen nedenfor:

\begin{algorithm}[Lenstras algoritme]
Lad $n \geq 2$ være et sammensat tal.
\begin{enumerate}
	\item Vælg heltal $x, y$ og $A$ mellem $1$ og $n$. Lad da $B = y^2 - x^3 - Ax \modu{n}$, så vi har den 	 
	elliptiske kurve
	\begin{align*}
		E : y^2 = x^3 + Ax + B, 
	\end{align*}
	hvor punktet $P=(x, y)$ er placeret.
	\item Tjek at $d = \gcd(4A^3 + 27B^2, n) = 1$. Hvis $d=n$ går vi tilbage
	til (1) og vælger et nyt $B$. Hvis $1 < d < n$ har vi fundet en faktor af $n$ og vi er færdige.
	\item Vælg et positivt heltal $k$ som et produkt af mange små primtal, lad eksempelvis
	\begin{align*}
		k = \lcm[1, 2, 3, ..., K],
	\end{align*}
	hvor $K \in \mathbb{Z}^+$.
	\item Forsøg at bestemme $kP = P + P + \ldots + P$. Hvis udregningen kan lade sig gøre går vi tilbage til (1) og 
	vælger en ny kurve, eller går til (3) og vælger et større $k$. Ellers har vi fundet en faktor som enten
	$d=\gcd(x_1 - x_2, n)$ eller $d=\gcd(y_1, n)$. Hvis $d=1$ gå da til (1) og vælg en ny kurve eller gå til (3) og 
	vælg større $k$. Hvis $d=n$ gå til (3) og vælg mindre $k$. Ellers er $d$ en ikke-triviel faktor.
\end{enumerate}
\end{algorithm}
For at se hvorfor der er en chance for, at vi støder på et valg af $x, y, A$ sådan at vi finder en faktor, lader vi $p$ være en primfaktor i $n$. Til den elliptiske kurve $E$ har vi den abelske gruppe $E(\mathbb{F}_p)$ og pr. sætning \ref{hasse} ved vi, at
\begin{align*}
	p + 1 - 2 \sqrt{p} < \#E(\mathbb{F}_p) < p + 1 + 2 \sqrt{p}.
\end{align*} 
En sætning af Deuring \cite{Deuring} siger, at for ethvert heltal $m \in (p+1-2\sqrt{p}, p+1+2\sqrt{p})$ findes der talpar $(A, B)$ i mængden
\begin{align*}
	\{ (A, B) \mid A, B \in \mathbb{F}_p, 4A^3 +27B^2 \neq 0 \},
\end{align*}
sådan at $\#E(\mathbb{F}_p) = m$ (se også \cite[afsnit~7.3]{Primes}). For hvert tal i intervallet findes der altså en elliptisk kurve, som har denne orden. Der er altså en positiv sandsynlighed for, at vi kan finde sådan en kurve. Hvor stor denne sandsynlighed er vil vi ikke komme ind på her, men nogle af disse overvejelser kan findes i både \cite{Lenstra} og \cite{Primes}.
\begin{example}
Lad nu 
\begin{align*}
	n = 753161713
\end{align*}
være det tal, som vi ønsker at faktorisere. Da $2^{n-1} = 437782651 \modu{n}$ er $n$ ikke et primtal. Vi vælger da
$x = 0$, $y = 1$ og $A=164$. Vi har dermed, at $B = 1^2 - 0^3 - 164 \cdot 0 = 1$ og den elliptiske kurve vi vil arbejde over bliver
\begin{align*}
	E : y^2 = x^3 + 164x + 1,
\end{align*}
hvorpå punktet $P=(0, 1)$ er placeret. Vi ser, at 
\begin{align*}
	D &= \gcd(4 \cdot 164^3 + 27 \modu{753161713}, 753161713) \\ &= \gcd(17643803, 753161713) = 1,
\end{align*}
så vi fortsætter derfor med algoritmen. Vi lader
\begin{align*}
	k = \lcm[1, 2, \ldots, 10] = 2520.
\end{align*}
Da $2520 = 2^{11} + 2^8 + 2^7 + 2^6 + 2^4 + 2^3$ skal vi beregne $2^i P \modu{753161713}$ for $0 \leq i \leq 11$. Dette gøres med additionsformlen og vi opsummerer vores resultater i tabellen nedenfor:

%	Større eksempel, men som ikke er praktisk udskrevet her:
%	n =  1688955439703788849,
\begin{center}
\begin{tabular}{c c c c}
$i$ & $2^i P \modu{753161713}$ & $i$ & $2^i P \modu{753161713}$ \\ 
\hline 
0 & $(0, 1)$ & 6 & $(743238772, 703386057)$  \\ 
1 & $(6724, 752610344)$ & 7 & $(309161840, 219780637)$  \\ 
2 & $(293427237, 450490340)$ & 8 & $(116974611, 722899047)$ \\ 
3 & $(468952095, 385687511)$ & 9 & $(329743899, 182819134)$ \\ 
4 & $(288125200, 446796094)$ & 10 & $(163952469, 456288424)$ \\ 
5 & $(106753239, 115973502)$ & 11 & $(15710788, 301760412)$
\end{tabular} 
\end{center}
Vi kan nu addere disse punkter igen vha. additionsformlerne, hvor vi stadigvæk regner modulo $n$:
\begin{align*}
	(2^3 + 2^4)P &= (606730980, 447512524). 
\end{align*}
Algoritmen giver os en faktor netop når additionen bryder sammen, hvilket kan ske da $\mathbb{Z}/n \mathbb{Z}$ ikke er et legeme. Dette problem viser sig i dette eksempel allerede ved den næste addition, hvor vi forsøger at udregne
\begin{align*}
	(2^3 + 2^4 + 2^6) P = (743238772, 703386057) \\ + (606730980, 447512524) \modu{n}.
\end{align*}
For at denne addition skal kunne lade sig gøre, skal differensen af deres $x$-koordinater have en invers modulo $n$. Dette er kun tilfældet, hvis $\gcd(x_2 - x_1, n) = 1$ (se appendiks, sætning k). Men vi ser, at
\begin{align*}
	\gcd(606730980 - 743238772, 753161713) = 19259,
\end{align*}
så der findes altså ikke en invers, men vi har i stedet fundet en faktor i $n$. Dermed har vi faktoriseringen
\begin{align*}
	753161713 = 19259 \cdot 39107.
\end{align*}
Det kan umiddelbart se ud til, at det var spild da vi lavede hele tabellen, men i beregningerne af $2^i P \modu{753161713}$ ville vi også kunne have stødt på et element, som ikke havde en invers og som dermed kunne give os en faktor.
\end{example}

\begin{remark}
Da $19258 = 2 \cdot 9629$ og $39106 = 2 \cdot 19553$ er de hhv. $9629$-glat og $19553$-glat så Pollards $p-1$ algoritme skulle vente på, at $k = \lcm[1, 2, \ldots, 9629]$ for at finde en faktor, hvilket ville være meget ineffektivt i forhold til Lenstras algoritme.
\end{remark}