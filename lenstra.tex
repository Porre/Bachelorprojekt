\section{Lenstras elliptiske kurve algoritme}

I \cite{Lenstra}
Lenstra præsenterede H. W. Lenstra en algoritme til faktorisering af heltal, som anvender elliptiske kurver. Vi skal arbejde med elliptiske kurver over $\mathbb{Z}/n \mathbb{Z}$, hvor $n$ er et sammensat tal. Dette er dog ikke et legeme og derfor er $E(\mathbb{Z}/n\mathbb{Z})$ ikke en gruppe, da additionen ikke er defineret for alle punkter.

Lad $n$ være et sammensat tal. Vi har, at inverserne til $x_1 - x_2$ og $y_1$ (se gruppeloven) kun findes modulo $n$, hvis (se bevis i appendiks)
\begin{align*}
	\gcd(x_1 - x_2, n) = 1 \quad \text{og} \quad \gcd(y_1, n) = 1.
\end{align*}
Men hvis vi er i stand til at finde punkter $P=(x_1, y_1)$ og $Q=(x_2, y_2)$ sådan, at summen $P+Q$ ikke er defineret, da er $\gcd(x, n) > 1$ hvor $x=x_1-x_2$ eller $x=y_1$ og dermed har vi muligvis fundet en ikke-triviel faktor i $n$. Vores behandling af algoritmen her er inspireret af \cite{Silverman} og \cite{Washington}.

Vælg først tilfældige heltal $x, y, A \in [1, n]$ og lad da $B=y^{2} - x^{3} - Ax \modu{n}$. Vi har da den elliptiske kurve (pseudo-kurve, da $n$ er et sammensat tal)
\begin{align*}
	E : y^2 = x^3 + Ax + B,
\end{align*} 
hvorpå vi har punktet $P=(x, y)$. Vi tjekker da, at $d= \gcd(4A^3 + 27B^2, n) = 1$. Hvis dette ikke er tilfældet og $1 < d < n$ har vi fundet en faktor i $n$. Hvis $d = n$ vælger vi et nyt $A$. For et heltal $K$ lader vi $k=\lcm[1, 2, \ldots, K]$ og vi forsøger da, at udregene 
\begin{align*}
	kP = \underbrace{P + P + \ldots + P}_{k \ \text{led}}.
\end{align*}
Det er ineffektivt at udregne $P+P + \ldots + P$ så vi gør ligesom i Pollards $p-1$ algoritme og skriver $k$ som den binære udvidelse
\begin{align*}
	k = k_0 + k_1 \cdot 2 + k_2 \cdot 2^2 + k_ 3 \cdot 2^3 + \ldots + k_r \cdot 2^r,
\end{align*}
hvor alle $k_i$ er $0$ eller $1$. Vi kan da udregne
\begin{align*}
	P_0 &= P \\
	P_1 &= 2P_0 = 2P \\
	P_2 &= 2P_1 = 2^2 P \\
	&\vdots \\
	P_r &= 2 P_{r-1} = 2^r P.
\end{align*}
Da kan vi bestemme $kP= (\text{summen af $P_i$'erne hvor $k_i = 1$})$. I hver udregning regner vi modulo $n$, da tallene ellers bliver alt for store og umulige at arbejde med. Vi håber på, at der i løbet af de udregninger er en addition, som ikke kan lade sig gøre og at dette giver os vores faktor. Vi opsummerer diskussionen i algoritmen nedenfor:

\begin{algorithm}[Lenstras algoritme]
Lad $n \geq 2$ være et sammensat tal, som vi ønsker at finde en faktor for.
\begin{enumerate}
	\item Vælg $x, y, A \in [1, n]$. Lad da $B = y^2 - x^3 - Ax \modu{n}$ for da har vi den elliptiske kurve
	\begin{align*}
		E : y^2 = x^3 + Ax + B, 
	\end{align*}
	hvorpå punktet $P=(x, y)$ er placeret.
	\item Tjek at $D = \gcd(4A^3 + 27B^2, n) = 1$. Hvis $D=n$ går vi tilbage
	til (1) og vælger et nyt $b$. Hvis $1 < D < n$ har vi fundet en faktor af $n$ og vi er færdige.
	\item Vælg et positivt heltal $k$ som et produkt af mange små primtal, lad eksempelvis
	\begin{align*}
		k = \lcm[1, 2, 3, ..., K],
	\end{align*}
	hvor $K \in \mathbb{Z}^+$.
	\item Forsøg at bestemme $kP = P + P + \ldots + P$. Hvis udregningen kan lade sig gøre går vi tilbage til (1) og 
	vælger en ny kurve, eller går til (3) og vælger et større $k$.
\end{enumerate}
\end{algorithm}
\noindent For at se hvorfor der er en god chance for, at vi støder på et valg af $x, y, A$ sådan, at additionsloven bryder sammen lader vi $p$ være en primfaktor i $n$. Beregningerne i diskussion af algoritmen blev alle lavet modulo $n$, men så gælder de også modulo $p$. Til den elliptiske kurve $E$ har vi den abelske gruppe $E(\mathbb{F}_p)$ og pr. sætning \ref{hasse} ved vi, at
\begin{align*}
	p + 1 - 2 \sqrt{p} < \#E(\mathbb{F}_p) < p + 1 + 2 \sqrt{p}.
\end{align*} 
En sætning af Deuring [DEURING] giver os endda, at for hvert heltal i intervallet findes $A, B \in \mathbb{F}_p$ sådan at $4A^3 + 27B^2 \neq 0$ og sådan at $E$ har denne orden. Lad igen $k = \lcm[1, 2, \ldots, K]$ og antag, at $m \mid k$ for $m \in (p+1-2\sqrt{p}, p + 1 + 2 \sqrt{p})$. Vi vil da vælge $A, x$ og $y$ indtil vi støder på en kurve $E$ så $\#E(\mathbb{F}_p) = m$. Pr. Deurings sætning er der en positiv sandsynlighed for, at vi finder en kurve sådan at dette er tilfældet. Pr. Waterhouse og andre er der specielt god chance hvis $p - \sqrt{p} < m < p + \sqrt{p}$. Vi kan her bemærke, at i Pollards $p-1$ algoritme var det nødvendigt, at $m \mid k$, men der kunne vi kun vælge $m$ sådan at 
$m = p-1$.

Antag da, at $\#E(\mathbb{F}_p) \mid k$ og at vi var i stand til at udregne $kP$, samt at $kP \neq \mathcal{O}$ på $E(\mathbb{Z}/n\mathbb{Z})$. Lad $P+Q = kP$ være den sidste addition, hvor $P=(x_1, y_1)$ og $Q=(x_2, y_2)$ begge er forskellige fra $\mathcal{O}$. Da er $P+Q=kP=\mathcal{O}$ på $E(\mathbb{F}_p)$ og vi har, at
\begin{align*}
	x_1 \equiv x_2 \modu{p}, \quad y_1 \equiv -y_2 \modu{p},
\end{align*}
hvilket videre giver os, at 
\begin{align*}
	p \mid \gcd(x_1 - x_2, n), \quad p \mid \gcd(y_1 + y_2, n).
\end{align*}
Men så er $P+Q$ ikke defineret på $E(\mathbb{Z}/n \mathbb{Z})$, men dette er i modstrid med vores antagelse. Vi har altså enten, at $kP = \mathcal{O}$ på $E(\mathbb{Z}/n\mathbb{Z})$ eller at vi i forsøget på at udregne $kP$ i $E(\mathbb{Z}/n\mathbb{Z})$ vil støde på to punkter, hvis sum ikke er defineret, hvilket med lidt held så resulterer i en ikke-triviel faktor i $n$.
\\[10pt]
Med algoritmen på plads kan vi nu se på et eksempler:

\begin{example}
Lad nu 
\begin{align*}
	n = 753161713
\end{align*}
være det tal, som vi ønsker at faktorisere. Da $2^{n-1} = 437782651 \modu{n}$ er $n$ ikke et primtal. Vi vælger da
$x = 0$, $y = 1$ og $A=164$. Vi har dermed, at $B = 1^2 - 0^3 - 164 \cdot 0 = 1$ og den elliptiske kurve vi vil arbejde over bliver
\begin{align*}
	E : y^2 = x^3 + 164x + 1,
\end{align*}
hvorpå punktet $P=(0, 1)$ er placeret. Vi ser, at 
\begin{align*}
	D &= \gcd(4 \cdot 164^3 + 27 \modu{753161713}, 753161713) \\ &= \gcd(17643803, 753161713) = 1,
\end{align*}
så vi fortsætter derfor med algoritmen. Vi lader
\begin{align*}
	k = \lcm[1, 2, \ldots, 10] = 2520.
\end{align*}
Da $2520 = 2^{11} + 2^8 + 2^7 + 2^6 + 2^4 + 2^3$ skal vi beregne $2^i P \modu{753161713}$ for $0 \leq i \leq 11$. Dette gøres med additionsformlen og vi opsummerer vores resultater i tabellen nedenfor:

%	Større eksempel, men som ikke er praktisk udskrevet her:
%	n =  1688955439703788849,
\begin{center}
\begin{tabular}{c c c c}
$i$ & $2^i P \modu{753161713}$ & $i$ & $2^i P \modu{753161713}$ \\ 
\hline 
0 & $(0, 1)$ & 6 & $(743238772, 703386057)$  \\ 
1 & $(6724, 752610344)$ & 7 & $(309161840, 219780637)$  \\ 
2 & $(293427237, 450490340)$ & 8 & $(116974611, 722899047)$ \\ 
3 & $(468952095, 385687511)$ & 9 & $(329743899, 182819134)$ \\ 
4 & $(288125200, 446796094)$ & 10 & $(163952469, 456288424)$ \\ 
5 & $(106753239, 115973502)$ & 11 & $(15710788, 301760412)$
\end{tabular} 
\end{center}
Vi kan nu addere disse punkter igen vha. additionsformlerne, hvor vi stadigvæk regner modulo $n$:
\begin{align*}
	(2^3 + 2^4)P &= (606730980, 447512524). 
\end{align*}
Algoritmen giver os en faktor netop når additionen bryder sammen, hvilket kan ske da $\mathbb{Z}/n \mathbb{Z}$ ikke er et legeme. Dette problem viser sig i dette eksempel allerede ved den næste addition, hvor vi forsøger at udregne
\begin{align*}
	(2^3 + 2^4 + 2^6) P = (743238772, 703386057) \\ + (606730980, 447512524) \modu{n}.
\end{align*}
For at denne addition skal kunne lade sig gøre, skal differensen af deres $x$-koordinater have en invers modulo $n$. Dette er kun tilfældet, hvis $\gcd(x_2 - x_1, n) = 1$ (se appendiks, sætning k). Men vi ser, at
\begin{align*}
	\gcd(606730980 - 743238772, 753161713) = 19259,
\end{align*}
så der findes altså ikke en invers, men vi har i stedet fundet en faktor i $n$. Dermed har vi faktoriseringen
\begin{align*}
	753161713 = 19259 \cdot 39107.
\end{align*}
Nu kan det virke til, at det var spild da vi lavede hele tabellen, men i beregningerne af $2^i P \modu{753161713}$ ville vi også kunne have løbet ind i et element, som ikke havde en invers og som dermed kunne give os en faktor.
\end{example}



