\chapter{Indledning}

Denne opgave omhandler elliptiske kurver, en specifik form for algebraiske kurver, som har nogle interessante egenskaber. Vi viser at gruppen af punkter på en elliptisk kurve, sammen med et såkaldt punkt i uendelig, under en geometrisk defineret addition giver os en abelsk gruppe. Dette resultat åbner for en rig teori, hvor vi først undersøger endomorfier for elliptiske kurver. Specielt ser vi på Frobenius endomorfien, som vi benytter til at bevise Hasses sætning, der begrænser antallet af punkter på en elliptisk kurve over et endeligt legeme $\mathbb{F}_q$. Vi viser desuden en sætning, som gør det muligt for os, at bestemme ordenen af $E(\mathbb{F}_{q^n})$ hvis blot vi kender ordenen af $E(\mathbb{F}_q)$.

Vi ser på en praktisk anvendelse af elliptiske kurver over endelige legemer idet, at man kan udnytte dem til primtalsfaktorisering. Vi beskriver først Pollards $p-1$ algoritme da den var inspirationen til den anden algoritme vi ser, nemlig Lenstras elliptiske kurve algoritme. 
Der er blevet udviklet et program, som demonstrerer disse algoritmer som kan findes på:
\begin{center}
	\url{http://orhoj.com/bachelor/factorization.zip}
\end{center}
I filen er vedlagt kildekoden og en eksekverbar fil \texttt{factorization.jar}, som skulle kunne afvikles direkte, hvis den nyeste version af Java er installeret. Man kan passende bruge programmet sideløbende med eksemplerne i kapitel \ref{chap:kapitel_faktorisering}. I tilfælde af at ovenstående link ikke virker findes en kopi også på
\begin{center}
	\url{http://daimi.au.dk/~jakkrej/bachelor/factorization.zip}
\end{center}
Ellers kan man ved henvendelse til \texttt{jakobii@msn.com} få tilsendt en kopi.

\vspace*{\fill}
\textit{Inden vi for alvor går i gang vil jeg gerne takke Johan P. Hansen for både god vejledning og hyggelige samtaler under hele forløbet.}