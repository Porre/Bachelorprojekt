\chapter{Faktoriseringsalgoritmer}
Vi vil i dette kapitel introducere nogle forskellige algoritmer
der kan benyttes til faktorisering. Først introduceres Pollards
$p-1$ algoritme, hvilket er den algoritme som var inspirationen
for Lenstras algoritme, der benytter elliptiske kurver.

Motivationen for disse hurtigere metoder er, at en naiv tilgang
er meget langsom. Antag at $n$ er et sammensat tal, som vi 
ønsker at faktorisere. Hvis $n$ faktoriseres som $n=n_1 n_2$ er 
det klart, at $\min \{n_1, n_2 \} \leq \sqrt{n}$. Vi kan da finde
en faktor ved at undersøge om først $2 \mid n$, dernæst om 
$3 \mid n$ osv. Vi vil da finde en faktor senest når vi kommer
til $\sqrt{n}$.

\section{Pollards $p-1$ metode}
Før vi beskriver algoritmen skal vi først bruge følgende 
definition.

\begin{definition}[$B$-potensglat]
Lad $B \in \mathbb{Z}^{+}$. Hvis $n \in \mathbb{Z}^{+}$ har 
primtalsfaktoriseringen $n = \prod {p_i}^{e_i}$, da siges $n$ 
at være $B$-potensglat hvis ${p_i}^{e_i} \leq B$ for alle $i$.
\end{definition}

\begin{example}
Da $50 = 2 \cdot 5^2$ følger det, at $50$ er $25$-potensglat. 
Bemærk, at den netop ikke er $5$-potensglat.
\end{example}

Med disse detaljer på plads er vi nu klar til, at beskrive Pollards
$p-1$ algoritme. Antag at det sammensatte tal $n$, som vi ønsker at
faktorisere, har en primfaktor $p$ sådan, at $p-1$ har mange små
primtalsfaktorer. Fra Fermats lille sætning ved vi, at 
\begin{align*}
	a^{p-1} \equiv 1 \modu{p},
\end{align*}
hvilket betyder at $p \mid \gcd(a^{p-1} - 1, n)$. Men da vi ikke kender
$p$ (det er jo den faktor vi leder efter)

\begin{algorithm}[Pollards $p-1$ algoritme]
Lad $n \geq 2$ være et sammensat tal, som er tallet vi ønsker at 
finde en faktor for.
\begin{enumerate}
	\item Vælg et tal $k \in \mathbb{Z}^+$ sådan, at $k$ er et produkt
	af mange små primtal opløftet i små potenser. F.eks. kan $k$ vælges
	som
	\begin{align*}
		k = \lcm [1, 2, \ldots, K],
	\end{align*}
	for $K \in \mathbb{Z}^+$.
	\item Vælg et heltal $a$ sådan, at $1 < a <n$.
	\item Udregn $\gcd(a, n)$. Hvis $\gcd(a, n) > 1$ har vi fundet en
	ikke-triviel faktor for $n$ og vi er færdige. Ellers fortsæt til næste
	trin.
	\item Udregn $D = \gcd(a^k - 1, n)$. Hvis $1 < D < n$ er $D$ en ikke-triviel
	faktor for $n$ og vi er færdige. Hvis $D = 1$ gå da tilbage til trin 1
	og vælg et større $k$. Hvis $D = n$ gå da til trin 2 og vælg et nyt $a$. 
\end{enumerate}


\end{algorithm}



Følgende er et eksempel på anvendelsen af Pollards algoritme,
hvor det går godt, altså hvor $p-1$ har små primfaktorer.

\begin{example}
Vi vil forsøge at faktorisere 
\begin{align*}
	N = 30042491.
\end{align*}
Vi ser at $2^{N-1} = 2^{30042490} \equiv 25171326 \ (\textrm{mod}\ 30042491)$,
så $N$ er ikke et primtal. Vi vælger som beskrevet i algoritmen
\begin{align*}
	a = 2 \quad \text{og} \quad k = \mathrm{LCM}(1,2,3,4,5,6,7) = 420.
\end{align*}
Da $420 = 2^2 + 2^5 + 2^7 + 2^8$ skal vi udregne $2^{2^i}$ for 
$0 \leq i \leq 8$. Vi springer de første par udregninger over.

\begin{center}
\begin{tabular}{|c|c|}
\hline 
$i$ & $2^{2^i} \ (\mathrm{mod} \ N)$ \\ 
\hline 
5 & 28933574 \\ 
6 & 27713768 \\ 
7 & 10802810 \\ 
8 & 16714289 \\ 
\hline 
\end{tabular} 
\end{center}

Denne tabel gør det forholdsvist let for os, at bestemme
\begin{align*}
	2^{420} &= 2^{2^2 + 2^5 + 2^7 + 2^8} \\
	&\equiv 16 \cdot 28933574 \cdot 10802810 \cdot 16714289 \ 
	(\mathrm{mod} \ 30042491) \\
	&\equiv 27976515 \ (\mathrm{mod} \ 30042491)
\end{align*}

Ved anvendelse af den euklidiske algoritme finder vi dernæst, at
\begin{align*}
	\mathrm{gcd}(2^{420} - 1, N) = \mathrm{gcd}(27976514, 30042491) = 1.
\end{align*}
Her fejler testen altså og vi er nået frem til, at $N$ ikke har nogle 
primtalsfaktorer $p$ sådan, at $p-1$ deler $420$.

\end{example}