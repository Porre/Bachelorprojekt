\section{Lenstras elliptiske kurve metode}

Vi vil nu se på Lenstras metode til at avende
elliptiske kurver til at faktorisere heltal. 
Idéerne til denne algoritme bygger videre på
Pollards $p-1$ metode, men den har den fordel, at 
hvor vi før kun havde en gruppe, $\mathbb{Z} / n \mathbb{Z}$,
at arbejde over, kan vi nu skifte imellem en masse.


\begin{example}
Vi vil nu give et eksempel for anvendelsen af Lenstras algoritme.
\end{example}


Vi har nu demonstreret, hvordan Lenstras faktoriseringsalgoritme 
anvendes i praksis og vil nu se på, hvorfor den er en effektiv algoritme
til dette problem. Vi vil vise Hasses sætning, som giver en grænse 
for antallet af punkter på en elliptisk kurve over et endeligt
legeme.


\begin{theorem}[Hasses sætning]
Lad $q = p^m$, hvor et primtal så $p \neq 2, 3$ og lad 
$A, B \in \mathbb{F}_q$ med $\Delta = 4a^3 + 27b^2 \neq 0$.
Hvis vi lader $\#E(\mathbb{F}_q)$ være antallet af løsninger
til 
\begin{align}
	y^2 = x^3 + Ax + B,
\end{align}
over $\mathbb{F}_q$, da vil
\begin{align}
	| \#E(\mathbb{F}_q) - q | \leq 2 \sqrt{q}.
\end{align}
\end{theorem}
Lad $E$ være en elliptisk kurve, som i (2.1). Til denne elliptiske
kurve tilknytter vi endnu en elliptisk kurve $E_\lambda$. For polynomiet 
\begin{align*}
	\lambda(t) = t^3 + At + B,
\end{align*}	
i $\mathbb{F}_q[t]$ lader vi $E_{\lambda}$ være en elliptisk kurve
over $\mathbb{F}_q(t)$ givet ved
\begin{align}
	\lambda y^2 = x^3 + Ax + B.
\end{align}
Da dette er ikke den sædvanlige form skal vi justere additionsformlerne
for, at vi stadigvæk opnår en gruppestruktur. Det kan ses, at 
$(t, 1)$ og $-(t, 1)=(t, -1) \in E_{\lambda}(K)$ ved indsættelse i (2.3).


Vi får altså en veldefineret funktion 
\begin{align*}
	d : \mathbb{Z} \to \{0, 1, 2, \ldots \},
\end{align*}
som er givet ved
\begin{align*}
	d(n) = d_n = \left\{ \begin{array}{rl}
		0 &\mbox{ hvis $P_n = \infp$} \\
		\deg f_n &\mbox{ ellers.}
	\end{array} \right.
\end{align*}


Forbindelsen mellem denne funktion $d(n)$ og Hasses sætning er følgende forhold
mellem $d_n$ og $N_q$, som vi vil bevise:

\begin{lemma}
Der gælder for funktionen $d_n$ og antallet af punkter $N_q$ på $E$
følgende lighed
\begin{align}
d_{-1} - d_0 - 1 = N_q - q.
\end{align}
\end{lemma}
\begin{proof}
Bemærk først at $d_0 = q$ da $\deg t^q = q$. Vi mangler da at vise, at 
$d_{-1} = N_q + 1$. For at vise ligheden vil vi reducere $X_{-1}$, fra
de modificerede additionsformler finder vi, at 
\begin{align*}
	X_{-1} &= (t^3 + at +b) \left( \frac{(t^3+at+b)^{(q-1)/2}+1}{t^q - t} \right)^2
	- (t^q + t) \\
	&= (t^3 + at + b) \frac{\big((t^3+at+b)^{(q-1)/2}+1 \big)^2}{(t^q - t)^2} 
	- (t^q + t) \\
	&= \frac{t^{2q+1} + R(t)}{(t^q - t)^2},
\end{align*}
hvor $R(t)$ er et polynomium hvor $\deg R(t) \leq 2q$. Vi bemærker, at over 
$\mathbb{F}_q$ kan vi skrive
\begin{align*}
	t^q - t = \prod_{\alpha \in \mathbb{F}_q} (t-\alpha) = t(t-1)\ldots (t-q+1).
\end{align*}
Der er kun to forskellige faktorer over $\mathbb{F}_q$, som kan udgå. Da en faktor
af den første type er relativt primisk til en faktor af den anden type har vi nu,
at 
\begin{align*}
	d_{-1} = \deg f_{-1} = 2q + 1 - 2m - n,
\end{align*}
hvor vi husker at for den første type er det en dobbeltrod. Idet vi husker, at 
$d_0 = q$ følger det, at 
\begin{align*}
	d_{-1} - d_0 = q + 1 - 2m - n.
\end{align*}
\end{proof}

Vi ønsker nu, at se nærmere på funktionen $d(n)$ og vi vil vise følgende lemma:

\begin{lemma}
For funktionen $d(n)$ gælder der, at 
\begin{align}
	d_n = n^2 - (d_{-1} - d_0 - 1)n + d_0.
\end{align}
\end{lemma}
\begin{proof}
Vi vil vise lemmaet vha. induktion. For $n=0$ og $n=-1$ ser vi først, at
\begin{align*}
	d_{-1} &= (-1)^2 + (d_{-1} - d_0 - 1) + d_0 = 1 + d_{-1} - d_0 -1 + d_0 = d_{-1}, \\
	d_0 &= d_0 = 0^2 - 0 \cdot (d_{-1} - d_0 - 1) + d_0 = d_0.
\end{align*}
Antag da, at $(2.5)$ er sandt for $n-1$ og $n$, hvor $n \geq 0$.
\end{proof}
























